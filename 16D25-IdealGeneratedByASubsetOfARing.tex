\documentclass[12pt]{article}
\usepackage{pmmeta}
\pmcanonicalname{IdealGeneratedByASubsetOfARing}
\pmcreated{2013-03-22 14:39:04}
\pmmodified{2013-03-22 14:39:04}
\pmowner{mathcam}{2727}
\pmmodifier{mathcam}{2727}
\pmtitle{ideal generated by a subset of a ring}
\pmrecord{9}{36242}
\pmprivacy{1}
\pmauthor{mathcam}{2727}
\pmtype{Definition}
\pmcomment{trigger rebuild}
\pmclassification{msc}{16D25}
\pmrelated{GeneratorsOfInverseIdeal}
\pmrelated{PrimeIdealsByKrullArePrimeIdeals}
\pmdefines{ideal generated by}
\pmdefines{left ideal generated by}
\pmdefines{right ideal generated by}
\pmdefines{generate as an ideal}
\pmdefines{generates as an ideal}
\pmdefines{generates}

\endmetadata

% this is the default PlanetMath preamble.  as your knowledge
% of TeX increases, you will probably want to edit this, but
% it should be fine as is for beginners.

% almost certainly you want these
\usepackage{amssymb}
\usepackage{amsmath}
\usepackage{amsfonts}
\usepackage{amsthm}

% used for TeXing text within eps files
%\usepackage{psfrag}
% need this for including graphics (\includegraphics)
%\usepackage{graphicx}
% for neatly defining theorems and propositions
%\usepackage{amsthm}
% making logically defined graphics
%%%\usepackage{xypic}

% there are many more packages, add them here as you need them

% define commands here

\newcommand{\mc}{\mathcal}
\newcommand{\mb}{\mathbb}
\newcommand{\mf}{\mathfrak}
\newcommand{\ol}{\overline}
\newcommand{\ra}{\rightarrow}
\newcommand{\la}{\leftarrow}
\newcommand{\La}{\Leftarrow}
\newcommand{\Ra}{\Rightarrow}
\newcommand{\nor}{\vartriangleleft}
\newcommand{\Gal}{\text{Gal}}
\newcommand{\GL}{\text{GL}}
\newcommand{\Z}{\mb{Z}}
\newcommand{\R}{\mb{R}}
\newcommand{\Q}{\mb{Q}}
\newcommand{\C}{\mb{C}}
\newcommand{\<}{\langle}
\renewcommand{\>}{\rangle}
\begin{document}
Let $X$ be a subset of a ring $R$.  Let $S=\{I_k\}$ be the collection of all left ideals of $R$ that contain $X$ (note that the set is nonempty since $X\subset R$ and $R$ is an ideal in itself).  The intersection
\begin{align*}
I=\bigcap_{I_k\in S} I_k
\end{align*}
is called the \emph{left ideal generated by $X$}, and is denoted by $(X)$.  We say that $X$ \emph{generates} $I$ as an ideal.

The definition is symmetrical for right ideals.

Alternatively, we can constructively form the set of elements that constitutes this ideal:  The left ideal $(X)$ consists of finite $R$-linear combinations of elements of $X$:
\begin{align*}
(X)=\left\{\sum_\lambda (r_\lambda a_\lambda + n_\lambda a_\lambda)\mid a_\lambda\in X, r_\lambda\in R, n_\lambda\in\Z\right\}.
\end{align*}
%%%%%
%%%%%
\end{document}

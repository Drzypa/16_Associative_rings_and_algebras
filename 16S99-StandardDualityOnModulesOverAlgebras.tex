\documentclass[12pt]{article}
\usepackage{pmmeta}
\pmcanonicalname{StandardDualityOnModulesOverAlgebras}
\pmcreated{2013-12-11 15:25:39}
\pmmodified{2013-12-11 15:25:39}
\pmowner{joking}{16130}
\pmmodifier{joking}{16130}
\pmtitle{standard duality on modules over algebras}
\pmrecord{5}{42209}
\pmprivacy{1}
\pmauthor{joking}{16130}
\pmtype{Theorem}
\pmcomment{trigger rebuild}
\pmclassification{msc}{16S99}
\pmclassification{msc}{20C99}
\pmclassification{msc}{13B99}

\endmetadata

% this is the default PlanetMath preamble.  as your knowledge
% of TeX increases, you will probably want to edit this, but
% it should be fine as is for beginners.

% almost certainly you want these
\usepackage{amssymb}
\usepackage{amsmath}
\usepackage{amsfonts}

% used for TeXing text within eps files
%\usepackage{psfrag}
% need this for including graphics (\includegraphics)
%\usepackage{graphicx}
% for neatly defining theorems and propositions
%\usepackage{amsthm}
% making logically defined graphics
%%\usepackage{xypic}

% there are many more packages, add them here as you need them

% define commands here

\begin{document}
Let $k$ be a field and let $A$ be an associative unital algebra. Throughout we will assume that all $A$-modules over $k$ are unital. If $M$ is a right $A$-module, then the space of all linear mappings
$$\mathrm{Hom}_{k}(M,k)$$
can be equipped with a left $A$-module structure as follows: for any $f\in\mathrm{Hom}_{k}(M,k)$ and $a\in A$ put
$$(af)(x)=f(xa).$$
Note that action direction need to be reversed, because
$$(abf)(x)=(bf)(xa)=f(xab).$$
Analogously $\mathrm{Hom}_{k}(-,k)$ takes left $A$-modules to right $A$-modules. Also this action is compatible with functoriality of $\mathrm{Hom}_{k}(-,k)$, which means that it takes $A$-homomorphisms to $A$-homomorphisms. In particular we obtain a (contravariant) functor from category of left (right) $A$-modules to category of right (left) $A$-modules. Obviously $\mathrm{Hom}$ does not change the dimension of spaces, so we have well defined functors
$$D:\mathrm{mod}A\to A\mathrm{mod}$$
$$D:A\mathrm{mod}\to \mathrm{mod}A$$
which are restrictions of $\mathrm{Hom}$ (here $\mathrm{mod}$ means finite dimensional modules left/right modules) and are known in literature as ,,standard dualities''.

\textbf{Proposition.} Both $D$'s are quasi inverse dualities of categories.

\textit{Proof.} Let $M$ be a finite dimensional $A$-module. We need to define a natural isomorphism between $M$ and $DD(M)$. Indeed, define
$$\tau_M:M\to DD(M);$$
$$\tau_M(m)(\alpha)=\alpha(m).$$
We will show that each $\tau$ is an isomorphism.
\begin{enumerate}
\item First we will show that $\tau$ is a monomorphism. Assume that $\tau_M(m)=0$ for nonzero $m\in M$. This is if and only if $\alpha(m)=0$ for every linear mapping $\alpha:M\to k$. But $m$ is nonzero, so there is a basis of $M$ (as linear space) which contains $m$. In particular there is a linear mapping $f:M\to k$ such that $f(m)=1$. Contradiction. Thus $m=0$, which completes this part.
\item $\tau$ is an epimorphism. Indeed, let $F:D(M)\to k$ be a linear mapping. We need to show, that there is $m\in M$ such that 
$$F(\alpha)=\alpha(m)$$
for any $\alpha\in D(M)$. Since $M$ is finite dimensional, then let $\{e_1,\ldots,e_n\}$ be a $k$-basis of $M$. Of course $\{e_1^*,\ldots,e_n^*\}$ is a basis of $D(M)$, where $e_i^*$ is given by $e_i^*(e_j)=1$ if $i=j$ and $e_i^*(e_j)=0$ otherwise. Define
$$\lambda_i=F(e_i^*)$$
and put
$$m=\sum_{i=1}^n\lambda_i\cdot e_i.$$
We leave it as a simple exercise, that $\tau(m)=F$.
\end{enumerate}
What remains is to prove, that $\tau$ is natural. Consider an $A$-homomorphism $f:X\to Y$. We need to show that the following diagram commutes:
$$\xymatrix{
X\ar[r]^f\ar[d]_{\tau_X} & Y\ar[d]^{\tau_Y}\\
DD(X)\ar[r]^{DD(f)} & DD(Y)
}$$
Indeed, if $x\in X$, then let $F=\tau_X(x)$. We have that 
$$DD(f)(F)=F\circ D(f)$$
and evaluating this at $\alpha\in D(M)$ we have
$$F(D(f)(\alpha))=F(\alpha\circ f)=(\alpha\circ f)(x)=\alpha(f(x)) = \tau_Y(f(x))(\alpha).$$
In particular we obtain that
$$DD(f)(\tau_X(x))=\tau_Y(f(x))$$
which means that
$$DD(f)\circ\tau_X=\tau_Y\circ f$$
which completes the proof. $\square$
%%%%%
%%%%%
\end{document}

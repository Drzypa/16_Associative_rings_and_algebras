\documentclass[12pt]{article}
\usepackage{pmmeta}
\pmcanonicalname{ModuleHomomorphism}
\pmcreated{2013-03-22 19:22:54}
\pmmodified{2013-03-22 19:22:54}
\pmowner{CWoo}{3771}
\pmmodifier{CWoo}{3771}
\pmtitle{module homomorphism}
\pmrecord{8}{42336}
\pmprivacy{1}
\pmauthor{CWoo}{3771}
\pmtype{Definition}
\pmcomment{trigger rebuild}
\pmclassification{msc}{16D20}
\pmclassification{msc}{15-00}
\pmclassification{msc}{13C10}
\pmclassification{msc}{16D10}
\pmdefines{bimodule homomorphism}

\endmetadata

\usepackage{amssymb,amscd}
\usepackage{amsmath}
\usepackage{amsfonts}
\usepackage{mathrsfs}

% used for TeXing text within eps files
%\usepackage{psfrag}
% need this for including graphics (\includegraphics)
%\usepackage{graphicx}
% for neatly defining theorems and propositions
\usepackage{amsthm}
% making logically defined graphics
%%\usepackage{xypic}
\usepackage{pst-plot}

% define commands here
\newcommand*{\abs}[1]{\left\lvert #1\right\rvert}
\newtheorem{prop}{Proposition}
\newtheorem{thm}{Theorem}
\newtheorem{ex}{Example}
\newcommand{\real}{\mathbb{R}}
\newcommand{\pdiff}[2]{\frac{\partial #1}{\partial #2}}
\newcommand{\mpdiff}[3]{\frac{\partial^#1 #2}{\partial #3^#1}}

\begin{document}
Let $R$ be a ring and $M,N$ left modules over $R$.  A function $f:M\to N$ is said to be a \emph{left module homomorphism} (over $R$) if 
\begin{enumerate}
\item $f$ is additive: $f(u+v)=f(u)+f(v)$, and
\item $f$ preserves left scalar multiplication: $f(rv)=rf(v)$, for any $r\in R$.
\end{enumerate}
If $M,N$ are right $R$-modules, then $f:M\to N$ is a \emph{right module homomorphism} provided that $f$ is additive and preserves right scalar multiplication: $f(vr)=f(v)r$ for any $r\in R$.  If $R$ is commutative, any left module homomorphism $f$ is a right module homomorphism, and vice versa, and we simply call $f$ a \emph{module homomorphism}$.

For example, any group homomorphism between abelian groups is a module homomorphism (over $\mathbb{Z}$) and vice versa, as any abelian group is a $\mathbb{Z}$-module (and vice versa).

If $R,S$ are rings, and $M,N$ are $(R,S)$-bimodules, then a function $f:M\to N$ is a \emph{bimodule homomorphism} if $f$ is a left $R$-module homomorphism from left $R$-module $M$ to left $R$-module $N$, and a right $S$-module homomorphism from right $S$-module $M$ to right $S$-module $N$.

Any group homomorphism between two abelian groups is a $(\mathbb{Z},\mathbb{Z})$-bimodule homomorphism.  Also, any left $R$-module homomorphism is an $(R,\mathbb{Z})$-bimodule homomorphism, and any right $S$-module homomorphism is a $(\mathbb{Z},S)$-bimodule homomorphism.  The converses are also true in all three cases.

%%%%%
%%%%%
\end{document}

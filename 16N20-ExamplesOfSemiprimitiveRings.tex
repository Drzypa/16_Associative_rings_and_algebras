\documentclass[12pt]{article}
\usepackage{pmmeta}
\pmcanonicalname{ExamplesOfSemiprimitiveRings}
\pmcreated{2013-03-22 12:50:39}
\pmmodified{2013-03-22 12:50:39}
\pmowner{yark}{2760}
\pmmodifier{yark}{2760}
\pmtitle{examples of semiprimitive rings}
\pmrecord{12}{33172}
\pmprivacy{1}
\pmauthor{yark}{2760}
\pmtype{Example}
\pmcomment{trigger rebuild}
\pmclassification{msc}{16N20}

\endmetadata

\usepackage{amssymb}
\usepackage{amsmath}
\usepackage{amsfonts}
\begin{document}
{\large \bf Examples of semiprimitive rings:}\\

{\bf The integers $\mathbb{Z}$:}\\
Since $\mathbb{Z}$ is commutative, any left ideal is two-sided.  So the maximal left ideals of $\mathbb{Z}$ are the maximal ideals of $\mathbb{Z}$, which are the ideals $p\mathbb{Z}$ for $p$ prime.
So $J(\mathbb{Z})= \bigcap_{p} p\mathbb{Z} = (0)$,
as there are infinitely many primes.\\

{\bf A matrix ring $M_n(D)$ over a division ring $D$:}\\
The ring $M_n(D)$ is simple, so the only proper ideal is $(0)$.  Thus $J(M_n(D))=(0)$.\\

{\bf A polynomial ring $R[x]$ over an integral domain $R$:}\\
Take $a \in J(R[x])$ with $a \neq 0$.
Then $ax \in J(R[x])$, since $J(R[x])$ is an ideal, and $\deg(ax) \geq 1$.
By one of the alternate characterizations of the Jacobson radical,
$1-ax$ is a unit.
But $\deg(1-ax)=\max\{\deg(1), \deg(ax)\} \geq 1$.
So $1-ax$ is not a unit, and by this contradiction we see that $J(R[x])=(0)$.
%%%%%
%%%%%
\end{document}

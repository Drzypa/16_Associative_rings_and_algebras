\documentclass[12pt]{article}
\usepackage{pmmeta}
\pmcanonicalname{TheMultiplicativeIdentityOfACyclicRingMustBeAGenerator}
\pmcreated{2013-03-22 15:56:59}
\pmmodified{2013-03-22 15:56:59}
\pmowner{Wkbj79}{1863}
\pmmodifier{Wkbj79}{1863}
\pmtitle{the multiplicative identity of a cyclic ring must be a generator}
\pmrecord{16}{37961}
\pmprivacy{1}
\pmauthor{Wkbj79}{1863}
\pmtype{Theorem}
\pmcomment{trigger rebuild}
\pmclassification{msc}{16U99}
\pmclassification{msc}{13F10}
\pmclassification{msc}{13A99}
\pmrelated{CyclicRing3}
\pmrelated{CriterionForCyclicRingsToBePrincipalIdealRings}
\pmrelated{CyclicRingsOfBehaviorOne}

\usepackage{amssymb}
\usepackage{amsmath}
\usepackage{amsfonts}

\usepackage{psfrag}
\usepackage{graphicx}
\usepackage{amsthm}
%%\usepackage{xypic}

\newtheorem*{thm*}{Theorem}
\begin{document}
\begin{thm*}
Let $R$ be a cyclic ring with multiplicative identity $u$.  Then $u$ \PMlinkname{generates}{Generator} the additive group of $R$.
\end{thm*}

\begin{proof}
Let $k$ be the behavior of $R$.  Then there exists a \PMlinkname{generator}{Generator} $r$ of the additive group of $R$ such that $r^2=kr$.  Let $a \in \mathbb{Z}$ with $u=ar$.  Then $r=ur=(ar)r=ar^2=a(kr)=(ak)r$.  If $R$ is infinite, then $ak=1$, causing $a=k=1$ since $k$ is a nonnegative integer.  If $R$ is finite, then $ak \equiv 1 \operatorname{mod} |R|$.  Thus, $\gcd(k,|R|)=1$.  Since $k$ divides $|R|$, $k=1$.  Therefore, $a \equiv 1 \operatorname{mod} |R|$.  In either case, $u=r$.
\end{proof}

Note that it was also proven that, if a cyclic ring has a multiplicative identity, then it has behavior one.  Its converse is also true.  See \PMlinkname{this theorem}{CyclicRingsOfBehaviorOne} for more details.
%%%%%
%%%%%
\end{document}

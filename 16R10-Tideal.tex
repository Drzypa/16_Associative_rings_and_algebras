\documentclass[12pt]{article}
\usepackage{pmmeta}
\pmcanonicalname{Tideal}
\pmcreated{2013-03-22 14:21:12}
\pmmodified{2013-03-22 14:21:12}
\pmowner{CWoo}{3771}
\pmmodifier{CWoo}{3771}
\pmtitle{T-ideal}
\pmrecord{7}{35831}
\pmprivacy{1}
\pmauthor{CWoo}{3771}
\pmtype{Definition}
\pmcomment{trigger rebuild}
\pmclassification{msc}{16R10}

% this is the default PlanetMath preamble.  as your knowledge
% of TeX increases, you will probably want to edit this, but
% it should be fine as is for beginners.

% almost certainly you want these
\usepackage{amssymb,amscd}
\usepackage{amsmath}
\usepackage{amsfonts}

% used for TeXing text within eps files
%\usepackage{psfrag}
% need this for including graphics (\includegraphics)
%\usepackage{graphicx}
% for neatly defining theorems and propositions
%\usepackage{amsthm}
% making logically defined graphics
%%%\usepackage{xypic}

% there are many more packages, add them here as you need them

% define commands here
\begin{document}
Let $R$ be a commutative ring and $R\langle X \rangle$ be a free algebra over $R$ on a set $X$ of \emph{non-commuting} variables.  A two-sided ideal $I$ of $R\langle X\rangle$ is called a $T$-\emph{ideal} if $\phi(I)\subseteq I$ for any $R$-endomorphism $\phi$ of $R\langle X\rangle$.

For example, let $A$ be a $R$-algebra.  Define $\mathcal{T}(A)$ to be the set of all \PMlinkname{polynomial identities}{PolynomialIdentityAlgebra} $f\in R\langle X\rangle$ for $A$.  Then $\mathcal{T}(A)$ is a $T$-ideal of $R\langle X\rangle$.  $\mathcal{T}(A)$ is called the $T$-\emph{ideal of \PMlinkescapetext{identities} of A}.
%%%%%
%%%%%
\end{document}

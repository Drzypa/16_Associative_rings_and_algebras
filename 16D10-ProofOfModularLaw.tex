\documentclass[12pt]{article}
\usepackage{pmmeta}
\pmcanonicalname{ProofOfModularLaw}
\pmcreated{2013-03-22 12:50:45}
\pmmodified{2013-03-22 12:50:45}
\pmowner{yark}{2760}
\pmmodifier{yark}{2760}
\pmtitle{proof of modular law}
\pmrecord{8}{33174}
\pmprivacy{1}
\pmauthor{yark}{2760}
\pmtype{Proof}
\pmcomment{trigger rebuild}
\pmclassification{msc}{16D10}
\pmrelated{FirstIsomorphismTheorem}

% this is the default PlanetMath preamble.  as your knowledge
% of TeX increases, you will probably want to edit this, but
% it should be fine as is for beginners.

% almost certainly you want these
\usepackage{amssymb}
\usepackage{amsmath}
\usepackage{amsfonts}

% used for TeXing text within eps files
%\usepackage{psfrag}
% need this for including graphics (\includegraphics)
%\usepackage{graphicx}
% for neatly defining theorems and propositions
%\usepackage{amsthm}
% making logically defined graphics
%%%\usepackage{xypic}

% there are many more packages, add them here as you need them

% define commands here
\begin{document}
First we show $C+(B \cap A) \subseteq B \cap (C+A)$:\\
Note that $C \subseteq B, B \cap A \subseteq B$, and therefore $C + (B \cap A) \subseteq B$.\\
Further, $C \subseteq C+A$, $B \cap A \subseteq C+A$, thus
$C+(B \cap A) \subseteq C+A$.\\

Next we show $B \cap (C+A) \subseteq C+(B \cap A)$:\\
Let $b \in B \cap (C+A)$.  Then $b=c+a$ for some $c \in C$ and $a \in A$.
Hence $a=b-c$, and so $a \in B$ since $b \in B$ and $c \in C \subseteq B$.\\
Hence $a \in B \cap A$, so $b = c+a \in C+(B \cap A)$.
%%%%%
%%%%%
\end{document}

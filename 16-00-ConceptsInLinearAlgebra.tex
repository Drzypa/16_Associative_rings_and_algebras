\documentclass[12pt]{article}
\usepackage{pmmeta}
\pmcanonicalname{ConceptsInLinearAlgebra}
\pmcreated{2013-03-22 14:13:30}
\pmmodified{2013-03-22 14:13:30}
\pmowner{matte}{1858}
\pmmodifier{matte}{1858}
\pmtitle{concepts in linear algebra}
\pmrecord{13}{35663}
\pmprivacy{1}
\pmauthor{matte}{1858}
\pmtype{Definition}
\pmcomment{trigger rebuild}
\pmclassification{msc}{16-00}
\pmclassification{msc}{13-00}
\pmclassification{msc}{20-00}
\pmclassification{msc}{15-00}

% this is the default PlanetMath preamble.  as your knowledge
% of TeX increases, you will probably want to edit this, but
% it should be fine as is for beginners.

% almost certainly you want these
\usepackage{amssymb}
\usepackage{amsmath}
\usepackage{amsfonts}
\usepackage{mathrsfs}


% used for TeXing text within eps files
%\usepackage{psfrag}
% need this for including graphics (\includegraphics)
%\usepackage{graphicx}
% for neatly defining theorems and propositions
%\usepackage{amsthm}
% making logically defined graphics
%%%\usepackage{xypic}

% there are many more packages, add them here as you need them

% define commands here

\newcommand{\sR}[0]{\mathbb{R}}
\newcommand{\sC}[0]{\mathbb{C}}
\newcommand{\sN}[0]{\mathbb{N}}
\newcommand{\sZ}[0]{\mathbb{Z}}

 \usepackage{bbm}
 \newcommand{\Z}{\mathbbmss{Z}}
 \newcommand{\C}{\mathbbmss{C}}
 \newcommand{\R}{\mathbbmss{R}}
 \newcommand{\Q}{\mathbbmss{Q}}



\newcommand*{\norm}[1]{\lVert #1 \rVert}
\newcommand*{\abs}[1]{| #1 |}
\begin{document}
\PMlinkescapeword{simple}
The aim of this entry is to present a list of the key
objects and operators used in linear algebra. Each entry in the
list links (or will link in the future) to the corresponding PlanetMath
entry where the object is presented in greater detail. For convenience, 
this list also presents the encouraged notation to use (at PlanetMath)
for these objects. 
%Where more than one notation is in common use, several options are listed; %PlanetMath authors are encouraged to use the first for consistency.

Some of this notation is simply an example of more general notation, either notation in set theory or notation for functions.  Some notation is also standard from category theory.

Suppose $V$ is a vector space over a field $K$.  Where the field $K$ is clear from context it is sometimes eliminated from the notation.  Let $L$ be a linear operator, or linear transformation, from $V$ to $W$, and $E$ be an endomorphism of $V$.
\begin{itemize}
\item $\mathscr{L}(V,W)$, the set of linear transformations between vector spaces $V$ and $W$ (also $\operatorname{Hom}_K(V,W)$),
\item basis of a vector space and the matrix associated with the basis,
\item $\dim_K V$, \PMlinkname{dimension}{Dimension2} of $V$,
\item $\operatorname{span}\{e_1,\ldots,e_n\}$ vector space spanned by vectors $\{e_i\}$ (note that the list of vectors need not be finite). Some other notations are $(e_1,e_2,\ldots,e_n)$ or $\langle e_1,e_2,\ldots,e_n\rangle$ (do not confuse last one with similar inner product notation),
\item $\det E$, determinant of a linear operator,
\item $\operatorname{tr} E$, trace of a linear operator (also $\operatorname{trace} E$),
\item $\operatorname{im} L$, image of a linear operator (also $\operatorname{img} L$ and $L(V)$),
\item $\operatorname{ker} L$, kernel of a linear operator,
\item for sets $A,B$ and  a point $x\in V$, the expressions $A+B$, $A-B$, $A+x$ 
are the \PMlinkname{Minkowski sums}{MinkowskiSum2} (not especially if $A$ and $B$ are subspaces, then their sum is a subspace, but the result may not be a direct sum),
\item $V^\ast$, dual space of a vector space $V$ (also $V^\vee$ or $\operatorname{Hom}_K(V,K)$), 
\item $A^\ast$, adjoint operator of a linear operator,
\item $V\oplus W$, direct sum of vector spaces $V$ and $W$ (both internal end external),
\item $V\otimes_K W$, tensor product of $V$ and $W$, and
\item $V\wedge W$, antisymmetrized tensor product (also called the wedge product),
\item bilinear forms and quadratic forms
\item generalizations of vector spaces over a field to vector spaces over a division ring to modules over a ring.
\end{itemize}
%%%%%
%%%%%
\end{document}

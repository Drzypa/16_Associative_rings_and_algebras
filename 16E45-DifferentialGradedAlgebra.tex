\documentclass[12pt]{article}
\usepackage{pmmeta}
\pmcanonicalname{DifferentialGradedAlgebra}
\pmcreated{2013-03-22 15:34:43}
\pmmodified{2013-03-22 15:34:43}
\pmowner{CWoo}{3771}
\pmmodifier{CWoo}{3771}
\pmtitle{differential graded algebra}
\pmrecord{8}{37488}
\pmprivacy{1}
\pmauthor{CWoo}{3771}
\pmtype{Definition}
\pmcomment{trigger rebuild}
\pmclassification{msc}{16E45}
\pmsynonym{DG Algebra}{DifferentialGradedAlgebra}

\endmetadata

% this is the default PlanetMath preamble.  as your knowledge
% of TeX increases, you will probably want to edit this, but
% it should be fine as is for beginners.

% almost certainly you want these
\usepackage{amssymb}
\usepackage{amsmath}
\usepackage{amsfonts}

% used for TeXing text within eps files
%\usepackage{psfrag}
% need this for including graphics (\includegraphics)
%\usepackage{graphicx}
% for neatly defining theorems and propositions
%\usepackage{amsthm}
% making logically defined graphics
%%%\usepackage{xypic}

% there are many more packages, add them here as you need them

% define commands here
\newcommand{\del}{\ensuremath{\partial}}
\begin{document}
Let $R$ be a commutative ring.  A \emph{differential graded algebra} (or \emph{DG algebra}) over $R$ is a complex $(A,\del^A)$ of $R$-modules with an element $1 \in A$ (the unit) and a degree zero chain map
$$A \otimes_R A \to A$$
that is unitary: $a1 = a = 1a$, and is associative: $a(bc) = (ab)c$.  We also will stipulate that a DG algebra is graded commutative; that is for each $x,y \in A$, we have
$$xy = (-1)^{|x||y|}yx$$
where $|x|$ means the degree of $x$.  Also, we assume that $A_i = 0$ for $i < 0$.    Without these final assumptions, we will say that $A$ is an \emph{associative} DG algebra.

The fact that the product is a chain map of degree zero is best described by the Leibniz Rule; that is, for each $x,y \in A$, we have
$$\del^A(xy) = \del^A(x)y + (-1)^{|x|}x\del^A(y).$$
%%%%%
%%%%%
\end{document}

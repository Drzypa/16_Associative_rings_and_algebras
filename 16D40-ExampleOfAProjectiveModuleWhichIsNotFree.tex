\documentclass[12pt]{article}
\usepackage{pmmeta}
\pmcanonicalname{ExampleOfAProjectiveModuleWhichIsNotFree}
\pmcreated{2013-03-22 18:49:55}
\pmmodified{2013-03-22 18:49:55}
\pmowner{joking}{16130}
\pmmodifier{joking}{16130}
\pmtitle{example of a projective module which is not free}
\pmrecord{6}{41636}
\pmprivacy{1}
\pmauthor{joking}{16130}
\pmtype{Example}
\pmcomment{trigger rebuild}
\pmclassification{msc}{16D40}

\endmetadata

% this is the default PlanetMath preamble.  as your knowledge
% of TeX increases, you will probably want to edit this, but
% it should be fine as is for beginners.

% almost certainly you want these
\usepackage{amssymb}
\usepackage{amsmath}
\usepackage{amsfonts}

% used for TeXing text within eps files
%\usepackage{psfrag}
% need this for including graphics (\includegraphics)
%\usepackage{graphicx}
% for neatly defining theorems and propositions
%\usepackage{amsthm}
% making logically defined graphics
%%%\usepackage{xypic}

% there are many more packages, add them here as you need them

% define commands here

\begin{document}
Let $R_1$ and $R_2$ be two nontrivial, unital rings and let $R=R_1\oplus R_2$. Furthermore let $\pi_i:R\to R_i$ be a projection for $i=1,2$. Note that in this case both $R_1$ and $R_2$ are (left) modules over $R$ via
$$\cdot:R\times R_i\to R_i;$$
$$(r,s)\cdot x = \pi_i(r,s)x,$$
where on the right side we have the multiplication in a ring $R_i$.

\textbf{Proposition.} Both $R_1$ and $R_2$ are projective $R$-modules, but neither $R_1$ nor $R_2$ is free.

\textit{Proof.} Obviously $R_1\oplus R_2$ is isomorphic (as a $R$-modules) with $R$ thus both $R_1$ and $R_2$ are projective as a direct summands of a free module.

Assume now that $R_1$ is free, i.e. there exists $\mathcal{B}=\{e_i\}_{i\in I}\subseteq R_1$ which is a basis. Take any $i_0\in I$. Both $R_1$ and $R_2$ are nontrivial and thus $1\neq 0$ in both $R_1$ and $R_2$. Therefore $(1,0)\neq (1,1)$ in $R$, but $$(1,1)\cdot e_{i_0}=\pi_1(1,1)e_{i_0}=1 e_{i_0}=\pi_1(1,0) e_{i_0}=(1,0)\cdot e_{i_0}.$$
This situation is impossible in free modules (linear combination is uniquely determined by scalars). Contradiction. Analogously we prove that $R_2$ is not free. $\square$
%%%%%
%%%%%
\end{document}

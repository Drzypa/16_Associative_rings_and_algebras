\documentclass[12pt]{article}
\usepackage{pmmeta}
\pmcanonicalname{IsomorphismSwappingZeroAndUnity}
\pmcreated{2013-03-22 19:17:16}
\pmmodified{2013-03-22 19:17:16}
\pmowner{pahio}{2872}
\pmmodifier{pahio}{2872}
\pmtitle{isomorphism swapping zero and unity}
\pmrecord{7}{42221}
\pmprivacy{1}
\pmauthor{pahio}{2872}
\pmtype{Example}
\pmcomment{trigger rebuild}
\pmclassification{msc}{16B99}
\pmclassification{msc}{20A05}
\pmclassification{msc}{16S50}
%\pmkeywords{isomorphism}
%\pmkeywords{zero}
%\pmkeywords{unity}
\pmrelated{RingHomomorphism}
\pmrelated{EpimorphismBetweenUnitaryRings}
\pmrelated{Null}
\pmrelated{TranslationAutomorphismOfAPolynomialRing}

% this is the default PlanetMath preamble.  as your knowledge
% of TeX increases, you will probably want to edit this, but
% it should be fine as is for beginners.

% almost certainly you want these
\usepackage{amssymb}
\usepackage{amsmath}
\usepackage{amsfonts}

% used for TeXing text within eps files
%\usepackage{psfrag}
% need this for including graphics (\includegraphics)
%\usepackage{graphicx}
% for neatly defining theorems and propositions
 \usepackage{amsthm}
% making logically defined graphics
%%%\usepackage{xypic}

% there are many more packages, add them here as you need them

% define commands here

\theoremstyle{definition}
\newtheorem*{thmplain}{Theorem}

\begin{document}
Let\, $(R,\,+,\,\cdot)$\, be a ring with unity 1.\, Define two new binary operations of $R$ as follows:
\begin{align}
a\!\oplus\!b \;=:\; a\!+\!b\!-\!1, \qquad a\odot\!b \;=:\; a\!+\!b\!-\!a\!\cdot\!b
\end{align}

Then we see that
\begin{align}
a\!\oplus\!1 \;=\; a \;=\; 1\!\oplus\!a, \qquad a\!\odot\!0 \;=\; a \;=\; 0\!\odot\!a.
\end{align}

But moreover, the algebraic system \,$(R,\,\oplus,\,\odot)$\, is a unitary ring, too, and isomorphic with the original ring.\\

In fact, we may define the bijective mapping
\begin{align}
f\!:\; x \,\mapsto\, 1\!-\!x
\end{align}
from $R$ to $R$ and verify that it is homomorphic:
$$f(a)\!\oplus\!f(b) \;=\; (1\!-\!a)\!\oplus\!(1\!-\!b) \;=\; (1\!-\!a)\!+\!(1\!-\!b)\!-\!1
\;=\; 1\!-\!a\!-\!b   \;=\; f(a\!+\!b),$$
$$f(a)\!\odot\!f(b) \;=\; (1\!-\!a)\!\odot\!(1\!-\!b) \;=\; 
(1\!-\!a)\!+\!(1\!-\!b)\!-(1\!-\!a)\!\cdot\!(1\!-\!b) \;=\; 1\!-\!a\!\cdot\!b \;=\; f(a\!\cdot\!b)$$
Thus\, $(R,\,\oplus,\,\odot)$\, as a \PMlinkname{homomorphic image}{HomomorphicImageOfGroup} of the ring\, $(R,\,+,\,\cdot)$\, is a ring, it's a question of two isomorphic rings.







%%%%%
%%%%%
\end{document}

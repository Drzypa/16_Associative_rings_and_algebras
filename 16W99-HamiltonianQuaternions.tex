\documentclass[12pt]{article}
\usepackage{pmmeta}
\pmcanonicalname{HamiltonianQuaternions}
\pmcreated{2013-03-22 12:35:42}
\pmmodified{2013-03-22 12:35:42}
\pmowner{mathcam}{2727}
\pmmodifier{mathcam}{2727}
\pmtitle{Hamiltonian quaternions}
\pmrecord{10}{32846}
\pmprivacy{1}
\pmauthor{mathcam}{2727}
\pmtype{Definition}
\pmcomment{trigger rebuild}
\pmclassification{msc}{16W99}
\pmsynonym{quaternion}{HamiltonianQuaternions}
\pmrelated{EulerFourSquareIdentity}
\pmrelated{QuaternionGroup}
\pmrelated{HyperkahlerManifold}
\pmrelated{MathematicalBiology}
\pmdefines{quaternion algebra}

\usepackage{amssymb}
\usepackage{amsmath}
\usepackage{amsfonts}
\begin{document}
\newcommand{\R}{\mathbb{R}}
\newcommand{\C}{\mathbb{C}}
\newcommand{\Q}{\mathbb{H}}
%\newcommand{\U}{\mathbb{U}}
\PMlinkescapeword{moment}
\PMlinkescapeword{relation}

\noindent
\textbf{Definition of $\Q$}

We define a unital associative algebra $\Q$ over $\R$, of dimension 4,
by the basis $\{\mathbf{1},\mathbf{i},\mathbf{j},\mathbf{k}\}$ and
the multiplication table
\begin{center}\begin{tabular}{rrrr}
$1$ & $i$ & $j$ & $k$ \\
$i$ & $-1$ & $k$ & $-j$ \\
$j$ & $-k$ & $-1$ & $i$ \\
$k$ & $j$ & $-i$ & $-1$
\end{tabular}\end{center}
(where the element in row $x$ and column $y$ is $xy$, not $yx$).
Thus an arbitrary element of $\Q$ is of the form
$$ a\mathbf{1} + b\mathbf{i} + c\mathbf{j} + d\mathbf{k},
\qquad a,b,c,d \in \R$$
(sometimes denoted by $\left< a, b, c, d\right>$
or by $a + \left<b,c,d\right>$) and the product of two elements
$\left<a,b,c,d\right>$ and $\left<\alpha,\beta,\gamma,\delta\right>$ (order matters)
is $\left<w,x,y,z\right>$ where
\begin{eqnarray*}
w&=& a\alpha - b\beta - c\gamma - d\delta \\
x&=& a\beta + b\alpha + c\delta - d\gamma \\
y&=& a\gamma - b\delta + c\alpha + d\beta \\
z&=& a\delta + b\gamma - c\beta + d\alpha
\end{eqnarray*}
The elements of $\Q$ are known as \emph{Hamiltonian quaternions}.

Clearly the subspaces of $\Q$ generated by $\{\mathbf{1}\}$
and by $\{\mathbf{1},\mathbf{i}\}$ are subalgebras isomorphic
to $\R$ and $\C$ respectively. $\R$ is customarily identified with
the corresponding subalgebra of $\Q$.
(We shall see in a moment that there are other and less obvious
embeddings of $\C$ in $\Q$.)
The real numbers commute with all the elements of $\Q$, and we have
$$\lambda \cdot \left<a,b,c,d\right>
= \left<\lambda a, \lambda b, \lambda c, \lambda d\right>$$
for $\lambda \in \R$ and $\left<a,b,c,d\right> \in \Q$.

\noindent
\textbf{Norm, conjugate, and inverse of a quaternion}

Like the complex numbers ($\C$), the quaternions have a
natural involution called the quaternion conjugate. If $q = a\mathbf{1}
+ b\mathbf{i} + c\mathbf{j} + d\mathbf{k}$, then the quaternion
conjugate of $q$, denoted $\overline{q}$, is simply $\overline{q}
= a\mathbf{1} - b\mathbf{i} - c\mathbf{j} - d\mathbf{k}$.

One can readily verify that if
$q = a\mathbf{1} + b\mathbf{i} + c\mathbf{j} + d\mathbf{k}$,
then $q\overline{q} = (a^2 + b^2 + c^2 + d^2)\mathbf{1}$.
(See Euler four-square identity.)
This product is used to form a norm $\|\cdot\|$ on the algebra
(or the ring) $\Q$: We define $\|q\| = \sqrt{s}$ where
$q\overline{q} = s\mathbf{1}$.

If $v,w \in \Q$ and $\lambda \in \R$, then
\begin{enumerate}
\item $\|v\| \geq 0$ with equality only if $v = \left<0,0,0,0\right> = 0$
\item $\|\lambda v\| = |\lambda| \|v\|$
\item $\|v+w\| \leq \|v\| + \|w\|$
\item $\|v \cdot w\| = \|v\| \cdot \|w\|$
\end{enumerate}
which means that $\Q$ qualifies as a normed algebra when we give it the
norm $\|\cdot\|$.

Because the norm of any nonzero quaternion $q$ is real and nonzero,
we have
$$ \frac{q\overline{q}}{\|q\|^2}
= \frac{\overline{q}q}{\|q\|^2}
= \left< 1, 0, 0, 0 \right> $$
which shows that any nonzero quaternion has an inverse:
$$q^{-1} = \frac{\overline{q}}{\|q\|^2}\;.$$

\noindent
\textbf{Other embeddings of $\C$ into $\Q$}

One can use any non-zero $q$ to define an embedding of
$\C$ into $\Q$.  If $\mathbf{n}(z)$
is a natural embedding of $z \in \C$
into $\Q$, then the embedding:
$$ z \rightarrow q \mathbf{n}(z) q^{-1} $$
is also an embedding into $\Q$.
Because $\Q$ is an associative algebra, it is obvious
that:
$$ ( q \mathbf{n}(a) q^{-1} )( q \mathbf{n}(b) q^{-1} ) =
q ( \mathbf{n}(a) \mathbf{n}(b) ) q^{-1} $$
and with the distributive laws, it is easy to check that
$$ ( q \mathbf{n}(a) q^{-1} ) + ( q \mathbf{n}(b) q^{-1} ) =
q ( \mathbf{n}(a) + \mathbf{n}(b) ) q^{-1} $$

\noindent
\textbf{Rotations in 3-space}

Let us write
$$U=\{q\in\Q:||q||=1\}$$
With multiplication, $U$ is a group.
Let us briefly sketch the relation between $U$ and the group
$SO(3)$ of rotations (about the origin) in 3-space.

An arbitrary element $q$ of $U$ can be expressed
$\cos\frac{\theta}{2} + \sin\frac{\theta}{2}
(a\mathbf{i} + b\mathbf{j} + c\mathbf{k})$,
for some real numbers $\theta,a,b,c$ such that $a^2+b^2+c^2=1$.
The permutation $v\mapsto qv$ of $U$ thus gives rise to a permutation
of the real sphere. It turns out that that permutation is a rotation.
Its axis is the line through $(0,0,0)$ and $(a,b,c)$, and the angle
through which it rotates the sphere is $\theta$.
If rotations $F$ and $G$ correspond to quaternions $q$ and $r$
respectively, then clearly the permutation $v\mapsto qrv$ corresponds
to the composite rotation $F\circ G$.
Thus this mapping of $U$ onto $SO(3)$ is a group homomorphism.
Its kernel is the subset $\{1,-1\}$ of $U$, and thus it comprises
a double cover of $SO(3)$. The kernel has a geometric interpretation
as well: two unit vectors in opposite directions determine the same
axis of rotation.

On the algebraic side, the quaternions provide an example of a division ring that is not a field.
%%%%%
%%%%%
\end{document}

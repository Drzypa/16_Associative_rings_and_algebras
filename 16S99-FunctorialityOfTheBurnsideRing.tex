\documentclass[12pt]{article}
\usepackage{pmmeta}
\pmcanonicalname{FunctorialityOfTheBurnsideRing}
\pmcreated{2013-03-22 18:08:06}
\pmmodified{2013-03-22 18:08:06}
\pmowner{joking}{16130}
\pmmodifier{joking}{16130}
\pmtitle{functoriality of the Burnside ring}
\pmrecord{5}{40687}
\pmprivacy{1}
\pmauthor{joking}{16130}
\pmtype{Derivation}
\pmcomment{trigger rebuild}
\pmclassification{msc}{16S99}

\endmetadata

% this is the default PlanetMath preamble.  as your knowledge
% of TeX increases, you will probably want to edit this, but
% it should be fine as is for beginners.

% almost certainly you want these
\usepackage{amssymb}
\usepackage{amsmath}
\usepackage{amsfonts}

% used for TeXing text within eps files
%\usepackage{psfrag}
% need this for including graphics (\includegraphics)
%\usepackage{graphicx}
% for neatly defining theorems and propositions
%\usepackage{amsthm}
% making logically defined graphics
%%%\usepackage{xypic}

% there are many more packages, add them here as you need them

% define commands here

\begin{document}
We wish to show how the Burnside ring $\Omega$ can be turned into a contravariant functor from the category of finite groups into the category of commutative, unital rings.\\ \\
Let $G$ and $H$ be finite groups. We already know how $\Omega$ acts on objects of the category of finite groups. Assume that $f:G\rightarrow H$ is a group homomorphism. Furthermore let $X$ be a $H$-set. Then $X$ can be naturally equiped with a $G$-set structure via function:
$$(g,x)\longmapsto f(g)x.$$
The set $X$ equiped with this group action will be denoted by $X_{f}$.\\ \\
Therefore a group homomorphism $f:G\rightarrow H$ induces a ring homomorphism $$\Omega(f):\Omega(H)\rightarrow\Omega(G)$$ such that $$\Omega(f)([X]-[Y])=[X_{f}]-[Y_{f}].$$ One can easily check that this turns $\Omega$ into a contravariant functor.
%%%%%
%%%%%
\end{document}

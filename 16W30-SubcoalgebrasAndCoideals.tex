\documentclass[12pt]{article}
\usepackage{pmmeta}
\pmcanonicalname{SubcoalgebrasAndCoideals}
\pmcreated{2013-03-22 18:49:19}
\pmmodified{2013-03-22 18:49:19}
\pmowner{joking}{16130}
\pmmodifier{joking}{16130}
\pmtitle{subcoalgebras and coideals}
\pmrecord{4}{41624}
\pmprivacy{1}
\pmauthor{joking}{16130}
\pmtype{Definition}
\pmcomment{trigger rebuild}
\pmclassification{msc}{16W30}

\endmetadata

% this is the default PlanetMath preamble.  as your knowledge
% of TeX increases, you will probably want to edit this, but
% it should be fine as is for beginners.

% almost certainly you want these
\usepackage{amssymb}
\usepackage{amsmath}
\usepackage{amsfonts}

% used for TeXing text within eps files
%\usepackage{psfrag}
% need this for including graphics (\includegraphics)
%\usepackage{graphicx}
% for neatly defining theorems and propositions
%\usepackage{amsthm}
% making logically defined graphics
%%%\usepackage{xypic}

% there are many more packages, add them here as you need them

% define commands here

\begin{document}
Let $(C,\Delta,\varepsilon)$ be a coalgebra over a field $k$.

\textbf{Definition.} Vector subspace $D\subseteq C$ is called \textit{subcoalgebra} iff $\Delta(D)\subseteq D\otimes D$.

\textbf{Definition.} Vector subspace $I\subseteq C$ is is called \textit{coideal} iff $\Delta(I)\subseteq I\otimes C + C\otimes I$ and $\varepsilon(I)=0$.

One can show that if $D\subseteq C$ is a subcoalgebra, then $(D, \Delta_{| D}, \varepsilon_{| D})$ is also a coalgebra. On the other hand, if $I\subseteq C$ is a coideal, then we can cannoicaly introduce a coalgebra structure on the quotient space $C/I$. More precisely, if $x\in C$ and $\Delta(x)=\sum a_i\otimes b_i$, then we define
$$\Delta':C/I\to (C/I)\otimes (C/I);$$
$$\Delta'(x+I)=\sum (a_i+I)\otimes (b_i+I)$$
and $\varepsilon':C/I\to k$ as $\varepsilon'(x+I)=\varepsilon(x)$. One can show that these two maps are well defined and $(C/I,\Delta',\varepsilon')$ is a coalgebra.
%%%%%
%%%%%
\end{document}

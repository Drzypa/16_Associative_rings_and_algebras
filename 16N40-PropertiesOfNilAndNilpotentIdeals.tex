\documentclass[12pt]{article}
\usepackage{pmmeta}
\pmcanonicalname{PropertiesOfNilAndNilpotentIdeals}
\pmcreated{2013-03-22 14:12:54}
\pmmodified{2013-03-22 14:12:54}
\pmowner{mclase}{549}
\pmmodifier{mclase}{549}
\pmtitle{properties of nil and nilpotent ideals}
\pmrecord{6}{35650}
\pmprivacy{1}
\pmauthor{mclase}{549}
\pmtype{Result}
\pmcomment{trigger rebuild}
\pmclassification{msc}{16N40}
\pmrelated{KoetheConjecture}
\pmrelated{NilIsARadicalProperty}

\endmetadata

% this is the default PlanetMath preamble.  as your knowledge
% of TeX increases, you will probably want to edit this, but
% it should be fine as is for beginners.

% almost certainly you want these
\usepackage{amssymb}
\usepackage{amsmath}
\usepackage{amsfonts}

% used for TeXing text within eps files
%\usepackage{psfrag}
% need this for including graphics (\includegraphics)
%\usepackage{graphicx}
% for neatly defining theorems and propositions
\usepackage{amsthm}
% making logically defined graphics
%%%\usepackage{xypic}

% there are many more packages, add them here as you need them

% define commands here


\newtheorem{lemma}{Lemma}
\newcommand{\famf}{\mathcal{F}}
\begin{document}
\PMlinkescapeword{lemma}

\begin{lemma}
Let $A \subset B$ be ideals of a ring $R$.  If $A$ is nil and $B/A$ is nil, then $B$ is nil.  If $A$ is nilpotent and $B/A$ is nilpotent, then $B$ is nilpotent.
\end{lemma}

\begin{proof}
Suppose that $A$ and $B/A$ are nil.  Let $x \in B$.  Then $x^n \in A$ for some $n$, since $B/A$ is nil.  But $A$ is nil, so there is an $m$ such that $x^{nm} = (x^n)^m = 0$.  Thus $B$ is nil.

Suppose that $A$ and $B/A$ are nilpotent.  Then there are natural numbers $n$ and $m$ such that $A^m  = 0$ and $B^n \subseteq A$.  Therefore, $B^{nm} = 0$.
\end{proof}


\begin{lemma}
The sum of an arbitrary family of nil ideals is nil.
\end{lemma}

\begin{proof}
Let $R$ be a ring, and let $\famf$ be a family of nil ideals of $R$.  Let $S = \sum_{I \in \famf} I$.  We must show that there is an $n$ with $x^n = 0$ for every $x \in S$.  Now, any such $x$ is actually in a sum of only finitely many of the ideals in $\famf$.  So it suffices to prove the lemma in the case that $\famf$ is finite.  By induction, it is enough to show that the sum of two nil ideals is nil.

Let $A$ and $B$ be nil ideals of a ring $R$.  Then $A \subset A + B$, and $A+B/A \cong B/(A \cap B)$, which is nil.  So by the first lemma, $A + B$ is nil.
\end{proof}

\begin{lemma}
The sum of a finite family of nilpotent left or right ideals is nilpotent.
\end{lemma}

\begin{proof}
We prove this for right ideals.  Again, by induction, it suffices to prove it for the case of two right ideals.

Let $A$ and $B$ be nilpotent right ideals of a ring $R$.  Then there are natural numbers $n$ and $m$ such that $A^n = 0$ and $b^m = 0$.

Let $k = n + m -1$.  Let $z_1, z_2, \dots, z_k$ be elements of $A + B$.  We may write $z_i = a_i + b_i$ for each $i$, with $a_i \in A$ and $b_i \in B$.  If we expand the product $z_1z_2 \dotsm z_k$ we get a sum of terms of the form
$x_1x_2 \dots x_k$ where each $x_i \in \{ a_i, b_i \}$.

Consider one of these terms $x_1x_2 \dotsm x_k$.  Then by our choice of $k$, it must contain at least $n$ of the $a_i$'s or at least $m$ of the $b_i$'s.  Without loss of generality, assume the former.
So there are indices $i_1 < i_2 < \dots < i_n$ with $x_{i_j} \in A$ for each $j$.
For $1 \le j \le n-1$, define $y_j = x_{i_j} x_{i_j+1} \dotsm x_{i_{j+1}-1}$,
and define $y_n = x_{i_n}x_{i_n+1} \dotsm x_k$.
Since $A$ is a right ideal, $y_j \in A$.

Then $x_1x_2 \dotsm x_k = x_1 x_2 \dotsm x_{i_1-1} y_1 y_2 \dotsm y_n \in x_1 x_2 \dotsm x_{i_1-1} A^n = 0$.

This is true for all choices of the $x_i$, and so $z_1 z_2 \dotsm z_k = 0$.  But this says that $(A+B)^k = 0$.
\end{proof}
%%%%%
%%%%%
\end{document}

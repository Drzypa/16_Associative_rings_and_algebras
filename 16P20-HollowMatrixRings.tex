\documentclass[12pt]{article}
\usepackage{pmmeta}
\pmcanonicalname{HollowMatrixRings}
\pmcreated{2013-03-22 17:42:21}
\pmmodified{2013-03-22 17:42:21}
\pmowner{Algeboy}{12884}
\pmmodifier{Algeboy}{12884}
\pmtitle{hollow matrix rings}
\pmrecord{10}{40148}
\pmprivacy{1}
\pmauthor{Algeboy}{12884}
\pmtype{Example}
\pmcomment{trigger rebuild}
\pmclassification{msc}{16P20}
\pmclassification{msc}{16W10}
\pmrelated{InvolutaryRing}
\pmrelated{Artinian}
\pmrelated{Noetherian2}

\endmetadata

\usepackage{latexsym}
\usepackage{amssymb}
\usepackage{amsmath}
\usepackage{amsfonts}
\usepackage{amsthm}

%%\usepackage{xypic}

%-----------------------------------------------------

%       Standard theoremlike environments.

%       Stolen directly from AMSLaTeX sample

%-----------------------------------------------------

%% \theoremstyle{plain} %% This is the default

\newtheorem{thm}{Theorem}

\newtheorem{coro}[thm]{Corollary}

\newtheorem{lem}[thm]{Lemma}

\newtheorem{lemma}[thm]{Lemma}

\newtheorem{prop}[thm]{Proposition}

\newtheorem{conjecture}[thm]{Conjecture}

\newtheorem{conj}[thm]{Conjecture}

\newtheorem{defn}[thm]{Definition}

\newtheorem{remark}[thm]{Remark}

\newtheorem{ex}[thm]{Example}

\newtheorem{claim}[thm]{Claim}


%\countstyle[equation]{thm}



%--------------------------------------------------

%       Item references.

%--------------------------------------------------


\newcommand{\exref}[1]{Example-\ref{#1}}

\newcommand{\thmref}[1]{Theorem-\ref{#1}}

\newcommand{\defref}[1]{Definition-\ref{#1}}

\newcommand{\eqnref}[1]{(\ref{#1})}

\newcommand{\secref}[1]{Section-\ref{#1}}

\newcommand{\lemref}[1]{Lemma-\ref{#1}}

\newcommand{\propref}[1]{Prop\-o\-si\-tion-\ref{#1}}

\newcommand{\corref}[1]{Cor\-ol\-lary-\ref{#1}}

\newcommand{\figref}[1]{Fig\-ure-\ref{#1}}

\newcommand{\conjref}[1]{Conjecture-\ref{#1}}


% Normal subgroup or equal.

\providecommand{\normaleq}{\unlhd}

% Normal subgroup.

\providecommand{\normal}{\lhd}

\providecommand{\rnormal}{\rhd}
% Divides, does not divide.

\providecommand{\divides}{\mid}

\providecommand{\ndivides}{\nmid}


\providecommand{\union}{\cup}

\providecommand{\bigunion}{\bigcup}

\providecommand{\intersect}{\cap}

\providecommand{\bigintersect}{\bigcap}










\begin{document}
\section{Definition}

\begin{defn}
Suppose that $R\subseteq S$ are both rings.  The \emph{hollow matrix ring} of $(R,S)$ 
is the ring of matrices:
\begin{equation*}
\begin{bmatrix} S & S \\ 0 & R\end{bmatrix} :=\left\{
\begin{bmatrix} s & t \\ 0 & r\end{bmatrix} : s,t\in S, r\in R\right\}.
\end{equation*}
\end{defn}
It is easy to check that this forms a ring under the usual matrix addition and
multiplication.
This definition is slightly simplified from the obvious higher dimensional examples
and the transpose of these matrices will also qualify as a hollow matrix ring.

The hollow matrix rings are highly counter-intuitive despite their simple definition.
In particular, they can be used to prove that in general a ring's left ideal
structure need not relate to its right ideal structure.  We highlight a few
examples of this.


\section{Left/Right Artinian and Noetherian}

We specialize to an example with the fields $\mathbb{Q}$ and $\mathbb{R}$, though
the same argument can be made in much more general settings.
\begin{equation}
  R := \begin{bmatrix} \mathbb{R} & \mathbb{R}\\ 0 & \mathbb{Q}\end{bmatrix}
     = \left\{\begin{bmatrix} a & b \\ 0 & c\end{bmatrix} : a,b\in \mathbb{R}, c\in\mathbb{Q}\right\}.
\end{equation}

\begin{claim}
 $R$ is left Artinian and left Noetherian.  
\end{claim}
\begin{proof}
Let $I$ be a left ideal of $R$ and suppose
that $r:=\begin{bmatrix} x & y \\ 0 & z\end{bmatrix}\in I$ for some
$x,y\in\mathbb{R}$ and $z\in\mathbb{Q}$.

Suppose that $z\neq 0$.  Hence, $s_q:=\begin{bmatrix} 0 & 0 \\ 0 & q/z\end{bmatrix}\in R$
for each $q\in\mathbb{Q}$ 
and so $s_qr=\begin{bmatrix} 0 & 0 \\ 0 & q\end{bmatrix}\in I$ for all $q\in\mathbb{Q}$.  In particular,
$\begin{bmatrix} x & y \\ 0 & 0 \end{bmatrix}=r-s_{1}r\in I$.  So in all cases it follows that 
$\begin{bmatrix} x & y\\ 0 & 0 \end{bmatrix} \in I$.  So now we take 
$r=\begin{bmatrix} x & y \\ 0 & 0 \end{bmatrix}$ and assume that $I$ does not contain any $r$ with $z\neq 0$.  
By observing matrix multiplication it follows that $I$ is now a left $\mathbb{R}$-vector space, and so any
chain of left $R$-modules is a chain of subspaces.  As $\dim_{\mathbb{R}} I\leq 2$,
it follows that such chains are finite.

Hence, there can be no infinite descending chain of distinct left ideals
and so $R$ is left Artinian and Noetherian.
\end{proof}

\begin{claim}
$R$ is not right Artinian nor right Noetherian.
\end{claim}
\begin{proof}
Using $\pi$ (the usual $3.14\dots$), or any other transcendental number, we define
\begin{equation}
   I_n := \begin{bmatrix} 0 & \mathbb{Q}[\pi;n] \\ 0 & 0 \end{bmatrix},
\end{equation}
where 
\begin{equation}
   \mathbb{Q}[\pi;n] := \{ q(\pi)\pi^{n} : q(x)\in\mathbb{Q}[x]\}.
\end{equation}
Since $\mathbb{Q}[\pi;n]$ properly contains $\mathbb{Q}[\pi;n+1]$ for all
$n\in\mathbb{Z}$, it follows that $\{I_n:n\in\mathbb{Z}\}$ is an infinite proper 
ascending and descending chain of right ideals.  Therefore, $R$ is neither right Artinian
nor right Noetherian.
\end{proof}


\begin{coro}
$R$ does not have a ring anti-isomorphism.  Thus $R$ is not an 
involutory ring.
\end{coro}
\begin{proof}
If $R$ is a ring with an anti-isomorphism, then the set of left ideals
is mapped to the set of right ideals, bijectively and order preserving.
This is not possible with $R$.
\end{proof}

%%%%%
%%%%%
\end{document}

\documentclass[12pt]{article}
\usepackage{pmmeta}
\pmcanonicalname{WedderburnArtinTheorem}
\pmcreated{2013-03-22 14:19:08}
\pmmodified{2013-03-22 14:19:08}
\pmowner{CWoo}{3771}
\pmmodifier{CWoo}{3771}
\pmtitle{Wedderburn-Artin theorem}
\pmrecord{14}{35785}
\pmprivacy{1}
\pmauthor{CWoo}{3771}
\pmtype{Theorem}
\pmcomment{trigger rebuild}
\pmclassification{msc}{16D70}
\pmsynonym{structure theorem on semisimple rings}{WedderburnArtinTheorem}
\pmsynonym{Artin-Wedderburn theorem}{WedderburnArtinTheorem}
\pmrelated{SemiprimitiveRing}

% this is the default PlanetMath preamble.  as your knowledge
% of TeX increases, you will probably want to edit this, but
% it should be fine as is for beginners.

% almost certainly you want these
\usepackage{amssymb}
\usepackage{amsmath}
\usepackage{amsfonts}

% used for TeXing text within eps files
%\usepackage{psfrag}
% need this for including graphics (\includegraphics)
%\usepackage{graphicx}
% for neatly defining theorems and propositions
%\usepackage{amsthm}
% making logically defined graphics
%%%\usepackage{xypic}

% there are many more packages, add them here as you need them

% define commands here
\begin{document}
If $R$ is a left semisimple ring, then
$$R \cong \mathbb{M}_{n_1}(D_1) \times \cdot\cdot\cdot \times \mathbb{M}_{n_r}(D_r)$$
where each $D_i$ is a division ring and $\mathbb{M}_{n_i}(D_i)$ is the matrix ring over $D_i$, $i = 1, 2, \ldots, r$.  The positive integer $r$ is unique, and so are the division rings (up to permutation).

Some immediate consequences of this theorem:
\begin{itemize}
\item
A \PMlinkname{simple}{SimpleRing} Artinian ring is isomorphic to a matrix ring over a division ring.
\item
A commutative semisimple ring is a finite direct product of fields.
\end{itemize}

This theorem is a special case of the more general \PMlinkescapetext{structure} theorem on semiprimitive rings.
%%%%%
%%%%%
\end{document}

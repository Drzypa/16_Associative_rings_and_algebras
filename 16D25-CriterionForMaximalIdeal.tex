\documentclass[12pt]{article}
\usepackage{pmmeta}
\pmcanonicalname{CriterionForMaximalIdeal}
\pmcreated{2013-03-22 19:10:40}
\pmmodified{2013-03-22 19:10:40}
\pmowner{pahio}{2872}
\pmmodifier{pahio}{2872}
\pmtitle{criterion for maximal ideal}
\pmrecord{6}{42085}
\pmprivacy{1}
\pmauthor{pahio}{2872}
\pmtype{Theorem}
\pmcomment{trigger rebuild}
\pmclassification{msc}{16D25}
\pmclassification{msc}{13A15}
%\pmkeywords{ring}
%\pmkeywords{unitary}
%\pmkeywords{maximality}
\pmrelated{MaximalIdealIsPrime}

% this is the default PlanetMath preamble.  as your knowledge
% of TeX increases, you will probably want to edit this, but
% it should be fine as is for beginners.

% almost certainly you want these
\usepackage{amssymb}
\usepackage{amsmath}
\usepackage{amsfonts}

% used for TeXing text within eps files
%\usepackage{psfrag}
% need this for including graphics (\includegraphics)
%\usepackage{graphicx}
% for neatly defining theorems and propositions
 \usepackage{amsthm}
% making logically defined graphics
%%%\usepackage{xypic}

% there are many more packages, add them here as you need them

% define commands here

\theoremstyle{definition}
\newtheorem*{thmplain}{Theorem}

\begin{document}
\PMlinkescapeword{maximal}

\textbf{Theorem.}\, In a commutative ring $R$ with non-zero unity, an ideal $\mathfrak{m}$ 
is maximal if and only if
\begin{align}
\forall a \in R\!\smallsetminus\!\mathfrak{m} \;\; \exists r \in R \quad\mbox{such that}\;\; 
1\!+\!ar \;\in\; \mathfrak{m}.
\end{align}



\emph{Proof.}\, $1^\circ$.\, Let first $\mathfrak{m}$ be a maximal ideal of $R$ and\, 
$a \in R\!\smallsetminus\!\mathfrak{m}$.\, Because\, $\mathfrak{m}\!+\!(a) = R$,\, there exist some elements\, 
$m \in \mathfrak{m}$\, and\, $-r \in R$\, such that\, $m\!-\!ar = 1$.\, Consequently,\, $1\!+\!ar = m \in \mathfrak{m}$.\\
$2^\circ$.\, Assume secondly that the ideal $\mathfrak{m}$ satisfies the condition (1).\, Now there must be a maximal ideal $\mathfrak{m}'$ of $R$ such that
$$\mathfrak{m} \;\subseteq\; \mathfrak{m}' \;\subset\; R.$$
Let us make the antithesis that $\mathfrak{m}'\!\smallsetminus\!\mathfrak{m}$ is non-empty.\, Choose an element
$$a \;\in\; \mathfrak{m}'\!\smallsetminus\!\mathfrak{m} \;\subset\; R\!\smallsetminus\!\mathfrak{m}.$$
By our assumption, we can choose another element $r$ of $R$ such that
$$s \;=\; 1\!+\!ar \;\in\; \mathfrak{m} \;\subset\; \mathfrak{m}'.$$
Then we have
$$1 \;=\; s\!-\!ar \;\in\; \mathfrak{m}'\!+\!\mathfrak{m}' \;=\; \mathfrak{m}'$$
which is impossible since with 1 the ideal $\mathfrak{m}'$ would contain the whole $R$.\, Thus the antithesis is wrong and\, $\mathfrak{m} = \mathfrak{m}'$\, is maximal.

%%%%%
%%%%%
\end{document}

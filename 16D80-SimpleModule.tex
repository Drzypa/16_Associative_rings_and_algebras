\documentclass[12pt]{article}
\usepackage{pmmeta}
\pmcanonicalname{SimpleModule}
\pmcreated{2013-03-22 12:01:40}
\pmmodified{2013-03-22 12:01:40}
\pmowner{bwebste}{988}
\pmmodifier{bwebste}{988}
\pmtitle{simple module}
\pmrecord{7}{31003}
\pmprivacy{1}
\pmauthor{bwebste}{988}
\pmtype{Definition}
\pmcomment{trigger rebuild}
\pmclassification{msc}{16D80}
\pmsynonym{irreducible module}{SimpleModule}
\pmrelated{JacobsonRadical}

\usepackage{amssymb}
\usepackage{amsmath}
\usepackage{amsfonts}
\usepackage{graphicx}
%%%\usepackage{xypic}
\begin{document}
Let $R$ be a ring, and let $M$ be an $R$-module.
We say that $M$ is a {\PMlinkescapetext {\it simple}} 
or {\PMlinkescapetext {\it irreducible}} {\PMlinkescapetext module }
if it contains no submodules 
other than itself and the zero module.

Typically, by convention, the zero module is not itself considered simple.
%%%%%
%%%%%
%%%%%
\end{document}

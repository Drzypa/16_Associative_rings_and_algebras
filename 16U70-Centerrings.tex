\documentclass[12pt]{article}
\usepackage{pmmeta}
\pmcanonicalname{Centerrings}
\pmcreated{2013-03-22 12:45:29}
\pmmodified{2013-03-22 12:45:29}
\pmowner{drini}{3}
\pmmodifier{drini}{3}
\pmtitle{center (rings)}
\pmrecord{6}{33065}
\pmprivacy{1}
\pmauthor{drini}{3}
\pmtype{Definition}
\pmcomment{trigger rebuild}
\pmclassification{msc}{16U70}
\pmsynonym{center}{Centerrings}
\pmrelated{GroupCentre}

% this is the default PlanetMath preamble.  as your knowledge
% of TeX increases, you will probably want to edit this, but
% it should be fine as is for beginners.

% almost certainly you want these
\usepackage{amssymb}
\usepackage{amsmath}
\usepackage{amsfonts}

% used for TeXing text within eps files
%\usepackage{psfrag}
% need this for including graphics (\includegraphics)
%\usepackage{graphicx}
% for neatly defining theorems and propositions
%\usepackage{amsthm}
% making logically defined graphics
%%%\usepackage{xypic} 

% there are many more packages, add them here as you need them

% define commands here
\begin{document}
If $A$ is a ring, the center of $A$, sometimes denoted $\operatorname{Z}(A)$, is the set of all elements in $A$ that commute with all other elements of $A$. That is,
$$\operatorname{Z}(A) = \{ a \in A \mid ax = xa \text{} \forall x \in A \}$$

Note that $0 \in \operatorname{Z}(A)$ so the center is non-empty. If we assume that $A$ is a ring with a multiplicative unity $1$, then $1$ is in the center as well. The center of $A$ is also a subring of $A$.
%%%%%
%%%%%
\end{document}

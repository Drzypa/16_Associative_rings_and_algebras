\documentclass[12pt]{article}
\usepackage{pmmeta}
\pmcanonicalname{InjectiveHull}
\pmcreated{2013-03-22 12:10:05}
\pmmodified{2013-03-22 12:10:05}
\pmowner{mclase}{549}
\pmmodifier{mclase}{549}
\pmtitle{injective hull}
\pmrecord{7}{31378}
\pmprivacy{1}
\pmauthor{mclase}{549}
\pmtype{Definition}
\pmcomment{trigger rebuild}
\pmclassification{msc}{16D50}
\pmsynonym{injective envelope}{InjectiveHull}

\usepackage{amssymb}
\usepackage{amsmath}
\usepackage{amsfonts}
\usepackage{graphicx}
%%%\usepackage{xypic}
\begin{document}
Let $X$ and $Q$ be modules.
We say that $Q$ is an {\it injective hull} or {\it injective envelope} of $X$
if $Q$ is both an injective module and an essential extension of $X$.

Equivalently, $Q$ is an injective hull of $X$
if $Q$ is injective,
and $X$ is a submodule of $Q$,
and if $g : X \to Q'$ is a monomorphism
from $X$ to an injective module $Q'$,
then there exists a monomorphism $h : Q \to Q'$
such that $h(x) = g(x)$ for all $x \in X$.
$$
\xymatrix{
  &
  0
        \ar[d]
  \\
  0
        \ar[r]
  &
  X
        \ar[r]^i
        \ar[d]_g
  &
  Q
        \ar@{-->}[dl]^h
  \\
  &
  Q'
}
$$

Every module $X$ has an injective hull, which is unique up to isomorphism.  The injective hull of $X$ is sometimes denoted $E(X)$.
%%%%%
%%%%%
%%%%%
\end{document}

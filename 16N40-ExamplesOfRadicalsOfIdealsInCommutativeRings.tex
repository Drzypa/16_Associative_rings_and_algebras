\documentclass[12pt]{article}
\usepackage{pmmeta}
\pmcanonicalname{ExamplesOfRadicalsOfIdealsInCommutativeRings}
\pmcreated{2013-03-22 19:04:34}
\pmmodified{2013-03-22 19:04:34}
\pmowner{joking}{16130}
\pmmodifier{joking}{16130}
\pmtitle{examples of radicals of ideals in commutative rings}
\pmrecord{5}{41962}
\pmprivacy{1}
\pmauthor{joking}{16130}
\pmtype{Example}
\pmcomment{trigger rebuild}
\pmclassification{msc}{16N40}
\pmclassification{msc}{14A05}
\pmclassification{msc}{13-00}

% this is the default PlanetMath preamble.  as your knowledge
% of TeX increases, you will probably want to edit this, but
% it should be fine as is for beginners.

% almost certainly you want these
\usepackage{amssymb}
\usepackage{amsmath}
\usepackage{amsfonts}

% used for TeXing text within eps files
%\usepackage{psfrag}
% need this for including graphics (\includegraphics)
%\usepackage{graphicx}
% for neatly defining theorems and propositions
%\usepackage{amsthm}
% making logically defined graphics
%%%\usepackage{xypic}

% there are many more packages, add them here as you need them

% define commands here

\begin{document}
Let $R$ be a commutative ring. Recall, that ideals $I,J$ in $R$ are called \textit{coprime} iff $I+J=R$. It can be shown, that if $I,J$ are coprime, then $IJ=I\cap J$. Elements $x_1,\ldots,x_n\in R$ are called \textit{pairwise coprime} iff $(x_i)+(x_j)=R$ for $i\neq j$. It follows by induction, that for pairwise coprime $x_1,\ldots,x_n\in R$ we have $(x_1\cdots x_n)=(x_1)\cap\cdots\cap(x_n)$,

Let $x\in R$ be such that $$x=p_1^{\alpha_1}\cdots p_n^{\alpha_n},$$
for some prime elements $p_i\in R$, $\alpha_i\in\mathbb{N}$ and assume that $p_1,\ldots,p_n$ are coprime. Denote by
$$\overline{x}=p_1\cdots p_n.$$

We shall denote by $r(I)$ the radical of an ideal $I\subseteq R$.

\textbf{Proposition.} $r\big((x)\big)=(\overline{x})$.

\textit{Proof.} ,,$\supseteq$'' Let $\alpha=\mathrm{max}(\alpha_1,\ldots,\alpha_n)$. Then we have 
$$\overline{x}^{\alpha}=(p_1\cdots p_n)^{\alpha}=p_1^{\alpha}\cdots p_n^{\alpha}=p_1^{\alpha-\alpha_1}\cdots p_n^{\alpha-\alpha_n}p_1^{\alpha_1}\cdots p_n^{\alpha_n}=yx$$
and thus $\overline{x}^{\alpha}\in (x)$. This shows the first inclusion.

,,$\subseteq$'' Assume that $y\in r\big((x)\big)$ and $y\neq 0$. Then there is $n\in\mathbb{N}$ such that $y^n\in (x)$. Thus $x$ divides $y^n$. Of course for any $i\in\{1,\ldots,n\}$ we have that $p_i$ divides $x$. Thus $p_i$ divides $y^n$ and since $p_i$ is prime, we obtain that $p_i$ divides $y$. Now for $i\neq j$ elements $p_i$ and $p_j$ are coprime, thus $\overline{x}$ divides $y$ and therefore $y\in (\overline{x})$, which completes the proof. $\square$

\textbf{Remark.} If we assume that $R$ is a PID (and thus UFD), then the previous proposition gives us the full characterization of radicals of ideals in $R$. In particular an ideal in PID is radical if and only if it is generated by an element of the form $p_1\cdots p_n$, where for $i\neq j$ elements $p_i$ and $p_j$ are not associated primes.

\textbf{Examples.} Consider ring of integers $\mathbb{Z}$. Then we have:
$$r\big((12)\big)=(6);$$
$$r\big((9)\big)=(3);$$
$$r\big((7)\big)=(7);$$
$$r\big((1125)\big)=(15).$$
%%%%%
%%%%%
\end{document}

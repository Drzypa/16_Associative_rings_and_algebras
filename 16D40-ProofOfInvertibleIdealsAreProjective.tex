\documentclass[12pt]{article}
\usepackage{pmmeta}
\pmcanonicalname{ProofOfInvertibleIdealsAreProjective}
\pmcreated{2013-03-22 18:35:51}
\pmmodified{2013-03-22 18:35:51}
\pmowner{gel}{22282}
\pmmodifier{gel}{22282}
\pmtitle{proof of invertible ideals are projective}
\pmrecord{5}{41327}
\pmprivacy{1}
\pmauthor{gel}{22282}
\pmtype{Proof}
\pmcomment{trigger rebuild}
\pmclassification{msc}{16D40}
\pmclassification{msc}{13A15}
%\pmkeywords{fractional ideal}
%\pmkeywords{invertible ideal}
%\pmkeywords{projective module}

\endmetadata

% this is the default PlanetMath preamble.  as your knowledge
% of TeX increases, you will probably want to edit this, but
% it should be fine as is for beginners.

% almost certainly you want these
\usepackage{amssymb}
\usepackage{amsmath}
\usepackage{amsfonts}

% used for TeXing text within eps files
%\usepackage{psfrag}
% need this for including graphics (\includegraphics)
%\usepackage{graphicx}
% for neatly defining theorems and propositions
\usepackage{amsthm}
% making logically defined graphics
%%%\usepackage{xypic}

% there are many more packages, add them here as you need them

% define commands here
\newtheorem*{theorem*}{Theorem}
\newtheorem*{lemma*}{Lemma}
\newtheorem*{corollary*}{Corollary}
\newtheorem{theorem}{Theorem}
\newtheorem{lemma}{Lemma}
\newtheorem{corollary}{Corollary}


\begin{document}
\PMlinkescapeword{invertible}
\PMlinkescapeword{right inverse}
\PMlinkescapeword{generating set}
\PMlinkescapeword{freely generate}
\PMlinkescapeword{fixed}
\PMlinkescapeword{inverse}
\PMlinkescapeword{generated by}

We show that a nonzero fractional ideal $\mathfrak{a}$ of an integral domain $R$ is invertible if and only if it is \PMlinkname{projective}{ProjectiveModule} as an $R$-module.

Let $\mathfrak{a}$ be an invertible fractional ideal and $f\colon M\rightarrow \mathfrak{a}$ be an epimorphism of $R$-modules. We need to show that $f$ has a right inverse.
Letting $\mathfrak{a}^{-1}$ be the inverse ideal of $\mathfrak{a}$, there exists $a_1,\ldots,a_n\in\mathfrak{a}$ and $b_1,\ldots,b_n\in\mathfrak{a}^{-1}$ such that
\begin{equation*}
a_1b_1+\cdots+a_nb_n=1
\end{equation*}
and, as $f$ is onto, there exist $e_k\in M$ such that $f(e_k)=a_k$. For any $x\in\mathfrak{a}$, $xb_k\in\mathfrak{a}\mathfrak{a}^{-1}=R$, so we can define $g\colon\mathfrak{a}\rightarrow M$ by
\begin{equation*}
g(x)\equiv (xb_1)e_1+\cdots+(xb_n)e_n.
\end{equation*}
Then
\begin{equation*}
f\circ g(x)= (xb_1)f(e_1)+\cdots+(xb_n)f(e_n)=x(b_1a_1+\cdots b_na_n)=x,
\end{equation*}
so $g$ is indeed a right inverse of $f$, and $\mathfrak{a}$ is projective.

Conversely, suppose that $\mathfrak{a}$ is projective and let $(a_i)_{i\in I}$ generate $\mathfrak{a}$ (this always exists, as we can let $a_i$ include every element of $\mathfrak{a}$). Then let $M$ be a module with free basis $(e_i)_{i\in I}$ and define $f\colon M\rightarrow\mathfrak{a}$ by $f(e_i)=a_i$. As $\mathfrak{a}$ is projective, $f$ has a right inverse $g\colon\mathfrak{a}\rightarrow M$. As $e_i$ freely generate $M$, we can uniquely define $g_i\colon\mathfrak{a}\rightarrow R$ by
\begin{equation*}
g(x)=\sum_{i\in I} g_i(x)e_i,
\end{equation*}
noting that all but finitely many $g_i(x)$ must be zero for any given $x$. Choosing any fixed nonzero $a\in\mathfrak{a}$, we can set $b_i=a^{-1}g_i(a)$ so that
\begin{equation*}
g_i(x)=a^{-1}g_i(ax)=a^{-1}xg_i(a)=b_ix
\end{equation*}
for all $x\in\mathfrak{a}$, and $b_i$ must equal zero for all but finitely many $i$. So, we can let $\mathfrak{b}$ be the fractional ideal generated by the $b_i$ and, noting that $xb_i=g_i(x)\in R$ we get $\mathfrak{a}\mathfrak{b}\subseteq R$. Furthermore, for any $x\in R$,
\begin{equation*}
x=a^{-1}f\circ g(ax)=\sum_ia^{-1}g_i(ax)f(e_i)=\sum_i xb_if(e_i)\in\mathfrak{b}\mathfrak{a}
\end{equation*}
so that $R\subseteq\mathfrak{a}\mathfrak{b}$, and $\mathfrak{b}$ is the inverse of $\mathfrak{a}$ as required.

%%%%%
%%%%%
\end{document}

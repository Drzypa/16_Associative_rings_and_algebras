\documentclass[12pt]{article}
\usepackage{pmmeta}
\pmcanonicalname{ModularIdeal}
\pmcreated{2013-03-22 17:31:47}
\pmmodified{2013-03-22 17:31:47}
\pmowner{CWoo}{3771}
\pmmodifier{CWoo}{3771}
\pmtitle{modular ideal}
\pmrecord{7}{39926}
\pmprivacy{1}
\pmauthor{CWoo}{3771}
\pmtype{Definition}
\pmcomment{trigger rebuild}
\pmclassification{msc}{16D25}

\endmetadata

\usepackage{amssymb,amscd}
\usepackage{amsmath}
\usepackage{amsfonts}
\usepackage{mathrsfs}

% used for TeXing text within eps files
%\usepackage{psfrag}
% need this for including graphics (\includegraphics)
%\usepackage{graphicx}
% for neatly defining theorems and propositions
\usepackage{amsthm}
% making logically defined graphics
%%\usepackage{xypic}
\usepackage{pst-plot}
\usepackage{psfrag}

% define commands here
\newtheorem{prop}{Proposition}
\newtheorem{thm}{Theorem}
\newtheorem{ex}{Example}
\newcommand{\real}{\mathbb{R}}
\newcommand{\pdiff}[2]{\frac{\partial #1}{\partial #2}}
\newcommand{\mpdiff}[3]{\frac{\partial^#1 #2}{\partial #3^#1}}
\begin{document}
Let $R$ be a ring.  A left ideal $I$ of $R$ is said to be \emph{modular} if there is an $e\in R$ such that $re-r\in I$ for all $r\in R$.  In other words, $e$ acts as a right identity element modulo $I$: $$re\equiv r\pmod I.$$
A right modular ideal is defined similarly, with $e$ be a left identity modulo $I$.  

\textbf{Remark}.  If an ideal $I$ is modular both as a left ideal as well as a right ideal in $R$, then $R/I$ is a unital ring.  Furthermore, every (left, right, two-sided) ideal in a unital ring is modular, implying that the notion of modular ideals is only interesting in rings without $1$.

\begin{thebibliography}{8}
\bibitem{pc} P. M. Cohn, {\em Further Algebra and Applications}, Springer (2003).
\end{thebibliography}
%%%%%
%%%%%
\end{document}

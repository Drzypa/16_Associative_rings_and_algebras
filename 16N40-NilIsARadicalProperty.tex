\documentclass[12pt]{article}
\usepackage{pmmeta}
\pmcanonicalname{NilIsARadicalProperty}
\pmcreated{2013-03-22 14:12:58}
\pmmodified{2013-03-22 14:12:58}
\pmowner{mclase}{549}
\pmmodifier{mclase}{549}
\pmtitle{nil is a radical property}
\pmrecord{5}{35651}
\pmprivacy{1}
\pmauthor{mclase}{549}
\pmtype{Proof}
\pmcomment{trigger rebuild}
\pmclassification{msc}{16N40}
\pmrelated{PropertiesOfNilAndNilpotentIdeals}

% this is the default PlanetMath preamble.  as your knowledge
% of TeX increases, you will probably want to edit this, but
% it should be fine as is for beginners.

% almost certainly you want these
\usepackage{amssymb}
\usepackage{amsmath}
\usepackage{amsfonts}

% used for TeXing text within eps files
%\usepackage{psfrag}
% need this for including graphics (\includegraphics)
%\usepackage{graphicx}
% for neatly defining theorems and propositions
%\usepackage{amsthm}
% making logically defined graphics
%%%\usepackage{xypic}

% there are many more packages, add them here as you need them

% define commands here

\newcommand{\nilrad}{\mathcal{N}}
\begin{document}
We must show that the nil property, $\nilrad$, is a radical property, that is that it satisfies the following conditions:

\begin{enumerate}
\item The class of $\nilrad$-rings is closed under homomorphic images.
\item Every ring $R$ has a largest $\nilrad$-ideal, which contains all other $\nilrad$-ideals of $R$. This ideal is written $\nilrad(R)$.
\item $\nilrad(R/\nilrad(R)) = 0$.
\end{enumerate}


It is easy to see that the homomorphic image of a nil ring is nil, for if $f \colon R \to S$ is a homomorphism and $x^n = 0$, then $f(x)^n = f(x^n) = 0$.

The sum of all nil ideals is nil (see proof \PMlinkid{here}{5650}), so this sum is the largest nil ideal in the ring.

Finally, if $N$ is the largest nil ideal in $R$, and $I$ is an ideal of $R$ containing $N$ such that $I/N$ is nil, then $I$ is also nil (see proof \PMlinkid{here}{5650}).  So $I \subseteq N$ by definition of $N$.  Thus $R/N$ contains no nil ideals.
%%%%%
%%%%%
\end{document}

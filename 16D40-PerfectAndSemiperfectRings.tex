\documentclass[12pt]{article}
\usepackage{pmmeta}
\pmcanonicalname{PerfectAndSemiperfectRings}
\pmcreated{2013-03-22 19:17:56}
\pmmodified{2013-03-22 19:17:56}
\pmowner{joking}{16130}
\pmmodifier{joking}{16130}
\pmtitle{perfect and semiperfect rings}
\pmrecord{4}{42234}
\pmprivacy{1}
\pmauthor{joking}{16130}
\pmtype{Definition}
\pmcomment{trigger rebuild}
\pmclassification{msc}{16D40}

\endmetadata

% this is the default PlanetMath preamble.  as your knowledge
% of TeX increases, you will probably want to edit this, but
% it should be fine as is for beginners.

% almost certainly you want these
\usepackage{amssymb}
\usepackage{amsmath}
\usepackage{amsfonts}

% used for TeXing text within eps files
%\usepackage{psfrag}
% need this for including graphics (\includegraphics)
%\usepackage{graphicx}
% for neatly defining theorems and propositions
%\usepackage{amsthm}
% making logically defined graphics
%%%\usepackage{xypic}

% there are many more packages, add them here as you need them

% define commands here

\begin{document}
A ring $R$ is called \textbf{left/right perfect} if for any left/right $R$-module $M$ there exists a projective cover $p:P\to M$.

A ring $R$ is called \textbf{left/right semiperfect} if for any left/right finitely-generated $R$-module $M$ there exists a projective cover $p:P\to M$.

It can be shown that there are rings which are left perfect, but not right perfect. However being semiperfect is left-right symmetric property.

Some examples of semiperfect rings include:
\begin{enumerate}
\item perfect rings;
\item left/right Artinian rings;
\item finite-dimensional algebras over a field $k$.
\end{enumerate}
%%%%%
%%%%%
\end{document}

\documentclass[12pt]{article}
\usepackage{pmmeta}
\pmcanonicalname{SemihereditaryRing}
\pmcreated{2013-03-22 14:48:55}
\pmmodified{2013-03-22 14:48:55}
\pmowner{CWoo}{3771}
\pmmodifier{CWoo}{3771}
\pmtitle{semihereditary ring}
\pmrecord{5}{36475}
\pmprivacy{1}
\pmauthor{CWoo}{3771}
\pmtype{Definition}
\pmcomment{trigger rebuild}
\pmclassification{msc}{16D80}
\pmclassification{msc}{16E60}
\pmdefines{semihereditary module}

% this is the default PlanetMath preamble.  as your knowledge
% of TeX increases, you will probably want to edit this, but
% it should be fine as is for beginners.

% almost certainly you want these
\usepackage{amssymb,amscd}
\usepackage{amsmath}
\usepackage{amsfonts}

% used for TeXing text within eps files
%\usepackage{psfrag}
% need this for including graphics (\includegraphics)
%\usepackage{graphicx}
% for neatly defining theorems and propositions
%\usepackage{amsthm}
% making logically defined graphics
%%%\usepackage{xypic}

% there are many more packages, add them here as you need them

% define commands here
\begin{document}
Let $R$ be a ring.  A right (left) $R$-module $M$ is called right (left) \emph{semihereditary} if every finitely generated submodule of $M$ is projective over $R$.

A ring $R$ is said to be a right (left) \emph{semihereditary ring} if all of its finitely generated right (left) ideals are projective as modules over $R$.  If $R$ is both left and right semihereditary, then $R$ is simply called a semihereditary ring.

\textbf{Remarks.}
\begin{itemize}
\item A hereditary ring is clearly semihereditary.
\item A ring that is left (right) semiheridtary is not necessarily right (left) semihereditary.
\item If $R$ is hereditary, then every finitely generated submodule of a free $R$-modules is a projective module.
\item A semihereditary integral domain is a Pr\"ufer domain, and conversely.
\end{itemize}
%%%%%
%%%%%
\end{document}

\documentclass[12pt]{article}
\usepackage{pmmeta}
\pmcanonicalname{BurnsideRing}
\pmcreated{2013-03-22 18:08:02}
\pmmodified{2013-03-22 18:08:02}
\pmowner{joking}{16130}
\pmmodifier{joking}{16130}
\pmtitle{Burnside ring}
\pmrecord{10}{40685}
\pmprivacy{1}
\pmauthor{joking}{16130}
\pmtype{Definition}
\pmcomment{trigger rebuild}
\pmclassification{msc}{16S99}

\endmetadata

% this is the default PlanetMath preamble.  as your knowledge
% of TeX increases, you will probably want to edit this, but
% it should be fine as is for beginners.

% almost certainly you want these
\usepackage{amssymb}
\usepackage{amsmath}
\usepackage{amsfonts}

% used for TeXing text within eps files
%\usepackage{psfrag}
% need this for including graphics (\includegraphics)
%\usepackage{graphicx}
% for neatly defining theorems and propositions
%\usepackage{amsthm}
% making logically defined graphics
%%%\usepackage{xypic}

% there are many more packages, add them here as you need them

% define commands here

\begin{document}
Let $G$ be a finite group. Recall that by $G$\textit{-set} we understand a pair $(X,\circ)$, where $X$ is a set and $\circ:G\times X\rightarrow X$ is a group action of $G$ on $X$. For short notation the pair notation will be omitted and $G$-sets will be simply denoted by capital letters.\\ \\
Recall that for each subgroup $H\subseteq G$ we have canonical $G$-set $G/H=\{gH;\ g\in G\}$ where group action is defined as follows: for any $g,k\in G$ we have $(g,kH)\longmapsto gkH$.\\ \\
Let $X$ and $Y$ be $G$-sets. Recall that by $G$\textit{-map} from $X$ to $Y$ we understand any function $F:X\rightarrow Y$ such that for any $g\in G$ and $x\in X$ we have $F(gx)=gF(x)$.\\ \\
It is easy to see that family of all $G$-sets and $G$-maps forms a category (with standard comoposition). We shall denote this category by $G-\mathbb{S}$. Moreover, by $G-\mathbb{S}_{0}$ we shall denote full subcategory of $G-\mathbb{S}$ whose objects are all finite $G$-sets.\\ \\
From $G$-sets $X$ and $Y$ one can construct another $G$-set in two interesting (from our point of view) ways, i.e. by taking disjoint union $X\sqcup Y$ with obvious group action and by taking product $X\times Y$ with group action as follows: $(g,(x,y))\longmapsto (gx,gy).$ Moreover it is clear that when $X$ and $Y$ are finite, so are $X\sqcup Y$ and $X\times Y$.\\ \\
Consider a finite $G$-set $X$. Then there exist a natural number $n\in\mathbb{N}$, finite family $\{H_{i}\}_{i=1}^{n}$ of subgroups of $G$ and an isomorphism (in $G-\mathbb{S}_{0}$ category) $$X\simeq\coprod_{i=1}^{n} G/H_{i}.$$
Therefore (since $G$ is finite) family of isomorphism classes of $G-\mathbb{S}_{0}$ forms a countable set.\\ \\
Denote by $\Omega^{+}(G)=\{[X];\ X\in G-\mathbb{S}_{0}\}$ the set of isomorphism classes of category $G-\mathbb{S}_{0}$. Then one can turn $\Omega^{+}(G)$ into a semiring as follows: for any finite $G$-sets $X$ and $Y$ define
$$[X]+[Y]=[X\sqcup Y];$$
$$[X][Y]=[X\times Y].$$
Note that here we treat the empty set as a $G$-set (with one and unique group action), therefore $\Omega^{+}(G)$ has zero element $[\emptyset]$ (the other way is to formally add the zero to $\Omega^{+}(G)$ - this is just technical thing).\\ \\
Define by $\Omega(G)=K(\Omega^{+}(G))$ the Grothendieck group of $(\Omega^{+}(G),+)$. If $A$ is an abelian semigroup and $f:A\times A\rightarrow A$ is a bilinear map, then it can be uniquely extended to a bilinear map $K(f):K(A)\times K(A)\rightarrow K(A)$, therefore $\Omega(G)$ can be uniquely turned into a ring from $\Omega^{+}(G)$. This ring is called \textit{the Burnside ring} of $G$.\\ \\
Some properties:\\ \\
$\mathrm{(0)}$ each element of $\Omega(G)$ can be expressed as a formal diffrence $[X]-[Y]$;\\ \\
$\mathrm{(1)}$ $\Omega(G)$ is a commutative, unital ring, where $[G/G]$ is the unity of $\Omega(G)$;\\ \\
$\mathrm{(2)}$ $\Omega$ can be turned into a contravariant functor from the category of finite groups to the category of commutative, unital rings;\\ \\
$\mathrm{(3)}$ $(\Omega^{+}(G),+)$ is a cancellative semigroup, therefore it embedds into $\Omega(G)$;\\ \\
$\mathrm{(4)}$ for the trivial group $E$ there is a ring isomorphism $\Omega(E)\simeq\mathbb{Z}$;\\ \\
$\mathrm{(5)}$ for any group $G$ there is a ring monomorphism $\varphi:\Omega(G)\rightarrow\bigoplus_{i=1}^{n}\mathbb{Z}$ for some natural number $n\in\mathbb{N}$; this is called the characteristic embedding;\\ \\
$\mathrm{(6)}$ for any two groups $G,H$ we have: if $\Omega(G)$ and $\Omega(H)$ are isomorphic (as a rings), then $|G|=|H|$; generally $G$ need not be isomorphic to $H$.
%%%%%
%%%%%
\end{document}

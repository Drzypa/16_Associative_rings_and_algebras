\documentclass[12pt]{article}
\usepackage{pmmeta}
\pmcanonicalname{GrothendieckGroup}
\pmcreated{2013-03-22 13:38:24}
\pmmodified{2013-03-22 13:38:24}
\pmowner{mhale}{572}
\pmmodifier{mhale}{572}
\pmtitle{Grothendieck group}
\pmrecord{11}{34290}
\pmprivacy{1}
\pmauthor{mhale}{572}
\pmtype{Definition}
\pmcomment{trigger rebuild}
\pmclassification{msc}{16E20}
\pmclassification{msc}{13D15}
\pmclassification{msc}{18F30}
\pmsynonym{group completion}{GrothendieckGroup}
\pmrelated{AlgebraicKTheory}
\pmrelated{KTheory}
\pmrelated{AlgebraicTopology}
\pmrelated{GrothendieckCategory}

\endmetadata

\usepackage{amssymb}
\usepackage{amsmath}
\usepackage{amsfonts}
\usepackage{amsthm}

% used for TeXing text within eps files
%\usepackage{psfrag}
% need this for including graphics (\includegraphics)
%\usepackage{graphicx}
% making logically defined graphics
%%%\usepackage{xypic}

% my maths package

\newcommand*{\Nset}{\mathbb{N}}
\newcommand*{\Zset}{\mathbb{Z}}
\newcommand*{\Qset}{\mathbb{Q}}
\newcommand*{\Rset}{\mathbb{R}}
\newcommand*{\Cset}{\mathbb{C}}
\newcommand*{\Hset}{\mathbb{H}}
\newcommand*{\Oset}{\mathbb{O}}
\newcommand*{\Bset}{\mathbb{B}}
\newcommand*{\Kset}{\mathbb{K}}
\newcommand*{\Sset}{\mathbb{S}}
\newcommand*{\Tset}{\mathbb{T}}
\newcommand*{\GLgrp}{\mathrm{GL}}
\newcommand*{\SLgrp}{\mathrm{SL}}
\newcommand*{\Ogrp}{\mathrm{O}}
\newcommand*{\SOgrp}{\mathrm{SO}}
\newcommand*{\Ugrp}{\mathrm{U}}
\newcommand*{\SUgrp}{\mathrm{SU}}
\newcommand*{\e}{\mathop{\mathrm{e}}\nolimits}
\newcommand*{\im}{\mathord{\mathrm{i}}}
\newcommand*{\identity}{\mathord{\mathrm{1\!\!\!\:I}}}
\newcommand*{\tr}{\mathop{\mathrm{tr}}}
\newcommand*{\Tr}{\mathop{\mathrm{Tr}}}
\renewcommand*{\d}{\mathrm{d}}
\newcommand*{\deriv}[2]{\frac{\d #1}{\d #2}}
\newcommand*{\pderiv}[2]{\frac{\partial #1}{\partial #2}}
\newcommand*{\fderiv}[2]{\frac{\delta #1}{\delta #2}}

% my noncommutative geometry package

\newcommand*{\algebra}[1][A]{\mathord{\mathcal{#1}}}
\newcommand*{\hilbert}[1][H]{\mathord{\mathcal{#1}}}
\newcommand*{\hilbmod}[1][E]{\mathord{\mathcal{#1}}}
\newcommand*{\Matrix}[2]{\mathord{\mathrm{M}_{#1}(#2)}}
\newcommand*{\dixmier}{\mathop{\mathrm{Tr}_\omega}}
\newcommand*{\Res}{\mathop{\mathrm{Res}}}
\newcommand*{\Wres}{\mathop{\mathrm{Wres}}}
\newcommand*{\Aut}{\mathop{\mathrm{Aut}}\nolimits}
\newcommand*{\Inn}{\mathop{\mathrm{Inn}}\nolimits}
\newcommand*{\Out}{\mathop{\mathrm{Out}}\nolimits}
\newcommand*{\Diff}{\mathop{\mathrm{Diff}}\nolimits}
\newcommand*{\Ker}{\mathop{\mathrm{Ker}}\nolimits}
\newcommand*{\Coker}{\mathop{\mathrm{Coker}}\nolimits}
\newcommand*{\Img}{\mathop{\mathrm{Im}}\nolimits}
\newcommand*{\End}{\mathop{\mathrm{End}}\nolimits}
\newcommand*{\spin}{\mathop{\mathrm{spin}}\nolimits}
\newcommand*{\Ind}{\mathop{\mathrm{Ind}}\nolimits}
\newcommand*{\KK}{\mathit{KK}}
\newcommand*{\HH}{\mathit{HH}}
\newcommand*{\HC}{\mathit{HC}}
\newcommand*{\ch}{\mathop{\mathrm{ch}}\nolimits}

% my category theory package

\newcommand*{\mathcat}[1]{\mathord{\mathbf{#1}}}
\newcommand*{\id}{\mathrm{id}}
\newcommand*{\op}{\mathrm{op}}
\newcommand*{\boxprod}{\mathbin{\square}}

% my environments

\newtheoremstyle{inlinedefn}{}{0pt}{}{}{\bfseries}{.}{0.5em}{}
\theoremstyle{inlinedefn}
\newtheorem{definition}{Definition}

\newtheoremstyle{break}{\baselineskip}{\baselineskip}{\itshape}{}{\bfseries}{}{\newline}{}
\theoremstyle{break}
\newtheorem{example}{Example}

% misc commands

\newcommand*{\defn}[1]{\textbf{#1}}
\begin{document}
Let $S$ be an abelian semigroup.
The \textbf{Grothendieck group} of $S$ is $K(S) = S\times S/\mathord{\sim}$,
where $\sim$ is the equivalence relation:
$(s,t) \sim (u,v)$ if there exists $r \in S$ such that $s+v+r = t+u+r$.
This is indeed an abelian group with zero element $(s,s)$ (any $s \in S$), inverse $-(s,t) = (t,s)$ and addition given by
$(s,t)+(u,v) = (s+u, t+v)$.
It is common to use the suggestive notation $t-s$ for $(t,s)$.

The Grothendieck group construction is a functor from the category of abelian semigroups to the category of abelian groups.
A morphism $f\colon S \to T$ induces a morphism $K(f)\colon K(S) \to K(T)$
which sends an element $(s^+,s^-) \in K(S)$ to $(f(s^+),f(s^-)) \in K(T)$.

\begin{example}
Let $(\Nset,+)$ be the semigroup of natural numbers with composition given by addition.
Then, $K(\Nset,+) = \Zset$.
\end{example}

\begin{example}
Let $(\Zset-\lbrace 0 \rbrace,\times)$ be the semigroup of non-zero integers with composition given by multiplication.
Then, $K(\Zset-\lbrace 0 \rbrace,\times) = (\Qset-\lbrace 0 \rbrace,\times)$.
\end{example}

\begin{example}
Let $G$ be an abelian group, then $K(G) \cong G$ via the identification $(g,h) \leftrightarrow g-h$
(or $(g,h) \leftrightarrow gh^{-1}$ if $G$ is multiplicative).
\end{example}

Let $C$ be a (essentially small) symmetric monoidal category.
Its Grothendieck group is $K([C])$,
i.e.\ the Grothendieck group of the isomorphism classes of objects of $C$.
%%%%%
%%%%%
\end{document}

\documentclass[12pt]{article}
\usepackage{pmmeta}
\pmcanonicalname{DivisionRing}
\pmcreated{2013-03-22 11:48:46}
\pmmodified{2013-03-22 11:48:46}
\pmowner{djao}{24}
\pmmodifier{djao}{24}
\pmtitle{division ring}
\pmrecord{10}{30356}
\pmprivacy{1}
\pmauthor{djao}{24}
\pmtype{Definition}
\pmcomment{trigger rebuild}
\pmclassification{msc}{16K99}
\pmclassification{msc}{81P05}
\pmsynonym{skew field}{DivisionRing}

\endmetadata

\usepackage{amssymb}
\usepackage{amsmath}
\usepackage{amsfonts}
\usepackage{graphicx}
%%%%\usepackage{xypic}
\begin{document}
A \emph{division ring} is a ring $D$ with identity such that
\begin{itemize}
\item $1 \neq 0$
\item For all nonzero $a \in D$, there exists $b \in D$ with $a \cdot b = b \cdot a = 1$
\end{itemize}
Every field is a commutative division ring. The Hamiltonian quaternions are an example of a division ring which is not a field.

%%%%%
%%%%%
%%%%%
%%%%%
\end{document}

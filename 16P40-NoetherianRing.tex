\documentclass[12pt]{article}
\usepackage{pmmeta}
\pmcanonicalname{NoetherianRing}
\pmcreated{2013-03-22 11:44:52}
\pmmodified{2013-03-22 11:44:52}
\pmowner{archibal}{4430}
\pmmodifier{archibal}{4430}
\pmtitle{noetherian ring}
\pmrecord{18}{30187}
\pmprivacy{1}
\pmauthor{archibal}{4430}
\pmtype{Definition}
\pmcomment{trigger rebuild}
\pmclassification{msc}{16P40}
\pmclassification{msc}{18-00}
\pmclassification{msc}{18E05}
\pmsynonym{noetherian}{NoetherianRing}
\pmrelated{Artinian}
\pmrelated{NoetherianModule}
\pmrelated{Noetherian2}
\pmdefines{left noetherian}
\pmdefines{right noetherian}
\pmdefines{left noetherian ring}
\pmdefines{right noetherian ring}

\usepackage{amssymb}
\usepackage{amsmath}
\usepackage{amsfonts}
%\usepackage{graphicx}
%%%%%\usepackage{xypic}
\begin{document}
\PMlinkescapeword{property}
\PMlinkescapeword{simple}
\PMlinkescapeword{natural}
\PMlinkescapeword{extension}
A ring $R$ is \emph{right noetherian} if it is a \PMlinkname{right noetherian module}{NoetherianModule}, considered as a right module over itself in the natural way (that is, an element $r$ acts by $x\mapsto xr$).  Similarly, $R$ is \emph{left noetherian} if it is a left noetherian module over itself (equivalently, if the opposite ring of $R$ is right noetherian). We say that $R$ is \emph{noetherian} if it is both left noetherian and right noetherian. 

Examining the definition, it is relatively easy to show that $R$ is right noetherian if and only if the three equivalent conditions hold:
\begin{enumerate}
\item right ideals are finitely generated,
\item the ascending chain condition holds on right ideals, or
\item every nonempty family of right ideals has a maximal element.
\end{enumerate}

Examples of Noetherian rings include:
\begin{itemize}
\item any field (as the only ideals are 0 and the whole ring),
\item the ring $\mathbb{Z}$ of integers (each ideal is generated by a single integer, the greatest common divisor of the elements of the ideal),
\item the \PMlinkname{$p$-adic integers}{PAdicIntegers}, $\mathbb{Z}_p$ for any prime $p$, where every ideal is generated by a multiple of $p$, and
\item the ring of complex polynomials in two variables, where some ideals (the ideal generated by $X$ and $Y$, for example) are not principal, but all ideals are finitely generated.
\end{itemize}
The Hilbert Basis Theorem says that a ring $R$ is noetherian if and only if the polynomial ring $R[x]$ is.

A ring can be right noetherian but not left noetherian.

The word noetherian is used in a number of other places.  A topology can be \PMlinkname{noetherian}{NoetherianTopologicalSpace}; although this is not related in a simple way to the property for rings, the definition is based on an ascending chain condition.  A site can also be noetherian; this is a generalization of the notion of noetherian for topological space.

Noetherian rings (and by extension most other uses of the word noetherian) are named after Emmy Noether (see \PMlinkexternal{Wikipedia}{http://en.wikipedia.org/wiki/Emmy_Noether} for a short biography) who made many contributions to algebra.  Older references tend to capitalize the word (Noetherian) but in some fields, such as algebraic geometry, the word has come into such common use that the capitalization is dropped (noetherian).  A few other objects with proper names have undergone this process (abelian, for example) while others have not (Galois groups, for example).  Any particular work should of course choose one convention and use it consistently.
%%%%%
%%%%%
%%%%%
%%%%%
\end{document}

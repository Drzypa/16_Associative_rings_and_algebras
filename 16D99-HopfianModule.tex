\documentclass[12pt]{article}
\usepackage{pmmeta}
\pmcanonicalname{HopfianModule}
\pmcreated{2013-03-22 14:20:21}
\pmmodified{2013-03-22 14:20:21}
\pmowner{CWoo}{3771}
\pmmodifier{CWoo}{3771}
\pmtitle{Hopfian module}
\pmrecord{9}{35810}
\pmprivacy{1}
\pmauthor{CWoo}{3771}
\pmtype{Definition}
\pmcomment{trigger rebuild}
\pmclassification{msc}{16D99}
\pmrelated{HopfianGroup}
\pmdefines{cohopfian module}

% this is the default PlanetMath preamble.  as your knowledge
% of TeX increases, you will probably want to edit this, but
% it should be fine as is for beginners.

% almost certainly you want these
\usepackage{amssymb,amscd}
\usepackage{amsmath}
\usepackage{amsfonts}

% used for TeXing text within eps files
%\usepackage{psfrag}
% need this for including graphics (\includegraphics)
%\usepackage{graphicx}
% for neatly defining theorems and propositions
%\usepackage{amsthm}
% making logically defined graphics
%%%\usepackage{xypic}

% there are many more packages, add them here as you need them

% define commands here
\begin{document}
\PMlinkescapeword{hopfian}
\PMlinkescapeword{cohopfian}

A left (right) module $M$ over a ring $R$ is \emph{Hopfian} if every surjective $R$-endomorphism of $M$ is an automorphism.  Dually, a left (right) $R$-module $M$ is \emph{cohopfian} if every injective $R$-endomorphism of $M$ is an automorphism.

\begin{thebibliography}{8}
\bibitem{lam1} T. Y. Lam, {\em Lectures on Modules and Rings}, Springer-Verlag, New York (1999).
\end{thebibliography}
%%%%%
%%%%%
\end{document}

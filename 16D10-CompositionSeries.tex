\documentclass[12pt]{article}
\usepackage{pmmeta}
\pmcanonicalname{CompositionSeries}
\pmcreated{2013-03-22 14:04:13}
\pmmodified{2013-03-22 14:04:13}
\pmowner{mclase}{549}
\pmmodifier{mclase}{549}
\pmtitle{composition series}
\pmrecord{6}{35428}
\pmprivacy{1}
\pmauthor{mclase}{549}
\pmtype{Definition}
\pmcomment{trigger rebuild}
\pmclassification{msc}{16D10}

% this is the default PlanetMath preamble.  as your knowledge
% of TeX increases, you will probably want to edit this, but
% it should be fine as is for beginners.

% almost certainly you want these
\usepackage{amssymb}
\usepackage{amsmath}
\usepackage{amsfonts}

% used for TeXing text within eps files
%\usepackage{psfrag}
% need this for including graphics (\includegraphics)
%\usepackage{graphicx}
% for neatly defining theorems and propositions
%\usepackage{amsthm}
% making logically defined graphics
%%%\usepackage{xypic}

% there are many more packages, add them here as you need them

% define commands here
\begin{document}
Let $R$ be a ring and let $M$ be a (right or left) $R$-module.
A series of submodules
$$M = M_0 \supset M_1 \supset M_2 \supset \dots \supset M_n = 0$$
in which each quotient $M_i/M_{i+1}$ is simple is called a composition series for $M$.

A module need not have a composition series.  For example, the ring of integers, $\mathbb{Z}$, considered as a module over itself, does not have a composition series.

A necessary and sufficient condition for a module to have a composition series is that it is both Noetherian and Artinian.

If a module does have a composition series, then all composition series are the same length.
This length (the number $n$ above) is called the \emph{composition length} of the module.

If $R$ is a semisimple Artinian ring, then $R_R$ and ${}_RR$ always have composition series.
%%%%%
%%%%%
\end{document}

\documentclass[12pt]{article}
\usepackage{pmmeta}
\pmcanonicalname{PrimeIdeal}
\pmcreated{2013-03-22 11:50:54}
\pmmodified{2013-03-22 11:50:54}
\pmowner{djao}{24}
\pmmodifier{djao}{24}
\pmtitle{prime ideal}
\pmrecord{15}{30409}
\pmprivacy{1}
\pmauthor{djao}{24}
\pmtype{Definition}
\pmcomment{trigger rebuild}
\pmclassification{msc}{16D99}
\pmclassification{msc}{13C99}
\pmrelated{MaximalIdeal}
\pmrelated{Ideal}
\pmrelated{PrimeElement}

\usepackage{amssymb}
\usepackage{amsmath}
\usepackage{amsfonts}
\usepackage{graphicx}
%%%%\usepackage{xypic}
\begin{document}
Let $R$ be a ring. A two-sided proper ideal $\mathfrak{p}$ of a ring $R$ is called a prime ideal if the following equivalent conditions are met:

\begin{enumerate}
\item If $I$ and $J$ are left ideals and the product of ideals $IJ$ satisfies $IJ \subset \mathfrak{p}$, then $I \subset \mathfrak{p}$ or $J \subset \mathfrak{p}$.
\item If $I$ and $J$ are right ideals with $IJ \subset \mathfrak{p}$, then $I \subset \mathfrak{p}$ or $J \subset \mathfrak{p}$.
\item If $I$ and $J$ are two-sided ideals with $IJ \subset \mathfrak{p}$, then $I \subset \mathfrak{p}$ or $J\subset \mathfrak{p}$.
\item If $x$ and $y$ are elements of $R$ with $xRy \subset \mathfrak{p}$, then $x \in \mathfrak{p}$ or $y \in \mathfrak{p}$.
\end{enumerate}

$R/\mathfrak{p}$ is a prime ring if and only if $\mathfrak{p}$ is a prime ideal. When $R$ is commutative with identity, a proper ideal $\mathfrak{p}$ of $R$ is prime if and only if for any $a,b \in R$, if $a\cdot b \in \mathfrak{p}$ then either $a \in \mathfrak{p}$ or $b \in \mathfrak{p}$. One also has in this case that $\mathfrak{p} \subset R$ is prime if and only if the quotient ring $R/\mathfrak{p}$ is an integral domain.
%%%%%
%%%%%
%%%%%
%%%%%
\end{document}

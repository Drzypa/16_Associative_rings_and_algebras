\documentclass[12pt]{article}
\usepackage{pmmeta}
\pmcanonicalname{ProductOfLeftAndRightIdeal}
\pmcreated{2013-03-22 17:38:09}
\pmmodified{2013-03-22 17:38:09}
\pmowner{pahio}{2872}
\pmmodifier{pahio}{2872}
\pmtitle{product of left and right ideal}
\pmrecord{7}{40057}
\pmprivacy{1}
\pmauthor{pahio}{2872}
\pmtype{Theorem}
\pmcomment{trigger rebuild}
\pmclassification{msc}{16D25}
\pmrelated{ProductOfIdeals}
\pmrelated{Intersection}
\pmrelated{IdealMultiplicationLaws}

\endmetadata

% this is the default PlanetMath preamble.  as your knowledge
% of TeX increases, you will probably want to edit this, but
% it should be fine as is for beginners.

% almost certainly you want these
\usepackage{amssymb}
\usepackage{amsmath}
\usepackage{amsfonts}

% used for TeXing text within eps files
%\usepackage{psfrag}
% need this for including graphics (\includegraphics)
%\usepackage{graphicx}
% for neatly defining theorems and propositions
 \usepackage{amsthm}
% making logically defined graphics
%%%\usepackage{xypic}

% there are many more packages, add them here as you need them

% define commands here

\theoremstyle{definition}
\newtheorem*{thmplain}{Theorem}

\begin{document}
Let $\mathfrak{a}$ and $\mathfrak{b}$ be ideals of a ring $R$.\, Denote by\, $\mathfrak{ab}$\, the subset of $R$ formed by all finite sums of products $ab$ with\, $a \in \mathfrak{a}$\, and\, 
$b \in \mathfrak{b}$.\, It is straightforward to verify the following facts:
\begin{itemize}
\item If $\mathfrak{a}$ is a \PMlinkname{left}{Ideal} and $\mathfrak{b}$ a right ideal, $\mathfrak{ab}$\, is a two-sided ideal of $R$.
\item If both $\mathfrak{a}$ and $\mathfrak{b}$ are two-sided ideals, then\,
 $\mathfrak{ab} \subseteq \mathfrak{a}\cap\mathfrak{b}$.
\end{itemize}
%%%%%
%%%%%
\end{document}

\documentclass[12pt]{article}
\usepackage{pmmeta}
\pmcanonicalname{CharacteristicOfFiniteRing}
\pmcreated{2013-03-22 19:10:19}
\pmmodified{2013-03-22 19:10:19}
\pmowner{pahio}{2872}
\pmmodifier{pahio}{2872}
\pmtitle{characteristic of finite ring}
\pmrecord{7}{42079}
\pmprivacy{1}
\pmauthor{pahio}{2872}
\pmtype{Theorem}
\pmcomment{trigger rebuild}
\pmclassification{msc}{16B99}
%\pmkeywords{characteristic of ring}
%\pmkeywords{order of ring}
\pmrelated{Multiple}
\pmrelated{IdealOfElementsWithFiniteOrder}

% this is the default PlanetMath preamble.  as your knowledge
% of TeX increases, you will probably want to edit this, but
% it should be fine as is for beginners.

% almost certainly you want these
\usepackage{amssymb}
\usepackage{amsmath}
\usepackage{amsfonts}

% used for TeXing text within eps files
%\usepackage{psfrag}
% need this for including graphics (\includegraphics)
%\usepackage{graphicx}
% for neatly defining theorems and propositions
 \usepackage{amsthm}
% making logically defined graphics
%%%\usepackage{xypic}

% there are many more packages, add them here as you need them

% define commands here

\theoremstyle{definition}
\newtheorem*{thmplain}{Theorem}

\begin{document}
\PMlinkescapeword{characteristic}

The \PMlinkname{characteristic}{Characteristic} of the residue class ring $\mathbb{Z}/m\mathbb{Z}$, which contains $m$ elements, is $m$, too.\, More generally, one has the


\textbf{Theorem.}\, The characteristic of a finite ring divides the number of the elements of the ring.\\

\emph{Proof.}\,Let $n$ be the characteristic of the ring $R$ with $m$ elements.\, Since $m$ is the \PMlinkname{order}{OrderGroup} of the group 
\,$(R,\,+)$,\, the Lagrange's theorem implies that
$$ma \;=\; 0 \quad \forall a \in R.$$
Let\, $m = qn\!+\!r$\, where\, $0 \leqq r < n$.\, Because
$$ra \;=\; (m\!-\!qn)a \;=\; ma\!-\!q(na) \;=\; 0\!-\!0 \;=\; 0 \quad \forall a \in R$$
and $n$ is the least positive integer $\nu$ making all\, $\nu a = 0$,\, the number $r$ must vanish.\, Therefore,\, 
$m \,=\, qn$,\, i.e.\, $n \mid m$.\\


\textbf{Remark.}\, A ring $R$, the polynomial ring $R[X]$ and the ring $R[[X]]$ of formal power series have always the same characteristic.

%%%%%
%%%%%
\end{document}

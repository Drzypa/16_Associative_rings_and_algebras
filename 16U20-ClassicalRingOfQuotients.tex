\documentclass[12pt]{article}
\usepackage{pmmeta}
\pmcanonicalname{ClassicalRingOfQuotients}
\pmcreated{2013-03-22 14:03:01}
\pmmodified{2013-03-22 14:03:01}
\pmowner{mclase}{549}
\pmmodifier{mclase}{549}
\pmtitle{classical ring of quotients}
\pmrecord{5}{35402}
\pmprivacy{1}
\pmauthor{mclase}{549}
\pmtype{Definition}
\pmcomment{trigger rebuild}
\pmclassification{msc}{16U20}
\pmclassification{msc}{16S90}
\pmsynonym{left classical ring of quotients}{ClassicalRingOfQuotients}
\pmsynonym{right classical ring of quotients}{ClassicalRingOfQuotients}
\pmrelated{OreCondition}
\pmrelated{ExtensionByLocalization}
\pmrelated{FiniteRingHasNoProperOverrings}
\pmdefines{regular}

% this is the default PlanetMath preamble.  as your knowledge
% of TeX increases, you will probably want to edit this, but
% it should be fine as is for beginners.

% almost certainly you want these
\usepackage{amssymb}
\usepackage{amsmath}
\usepackage{amsfonts}

% used for TeXing text within eps files
%\usepackage{psfrag}
% need this for including graphics (\includegraphics)
%\usepackage{graphicx}
% for neatly defining theorems and propositions
%\usepackage{amsthm}
% making logically defined graphics
%%%\usepackage{xypic}

% there are many more packages, add them here as you need them

% define commands here
\begin{document}
Let $R$ be a ring.  An element of $R$ is called \emph{regular} if it is not a right zero divisor or a left zero divisor in $R$.

A ring $Q \supset R$ is a \emph{left classical ring of quotients} for $R$
(resp. \emph{right classical ring of quotients} for $R$) if it satisifies:
\begin{itemize}
\item every regular element of $R$ is invertible in $Q$
\item every element of $Q$ can be written in the form $x^{-1}y$ (resp. $yx^{-1}$) with $x, y \in R$
and $x$ regular.
\end{itemize}

If a ring $R$ has a left or right classical ring of quotients, then it is unique up to isomorphism.

If $R$ is a commutative integral domain, then the left and right classical rings of quotients always exist -- they are the field of fractions of $R$.

For non-commutative rings, necessary and sufficient conditions are given by Ore's Theorem.

Note that the goal here is to construct a ring which is not too different from $R$, but in which more elements are invertible.  The first condition says which elements we want to be invertible.  The second condition says that $Q$ should \PMlinkescapetext{contain} just enough extra elements to make the regular elements invertible.

Such rings are called classical rings of quotients, because there are other rings of quotients.  These all attempt to enlarge $R$ somehow to make more elements invertible (or sometimes to make ideals invertible).

Finally, note that a ring of quotients is not the same as a quotient ring.
%%%%%
%%%%%
\end{document}

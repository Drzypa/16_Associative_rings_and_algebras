\documentclass[12pt]{article}
\usepackage{pmmeta}
\pmcanonicalname{InvertibleIdealsAreProjective}
\pmcreated{2013-03-22 18:35:47}
\pmmodified{2013-03-22 18:35:47}
\pmowner{gel}{22282}
\pmmodifier{gel}{22282}
\pmtitle{invertible ideals are projective}
\pmrecord{5}{41326}
\pmprivacy{1}
\pmauthor{gel}{22282}
\pmtype{Theorem}
\pmcomment{trigger rebuild}
\pmclassification{msc}{16D40}
\pmclassification{msc}{13A15}
%\pmkeywords{projective module}
%\pmkeywords{fractional ideal}
%\pmkeywords{invertible ideal}
\pmrelated{ProjectiveModule}
\pmrelated{FractionalIdeal}

% this is the default PlanetMath preamble.  as your knowledge
% of TeX increases, you will probably want to edit this, but
% it should be fine as is for beginners.

% almost certainly you want these
\usepackage{amssymb}
\usepackage{amsmath}
\usepackage{amsfonts}

% used for TeXing text within eps files
%\usepackage{psfrag}
% need this for including graphics (\includegraphics)
%\usepackage{graphicx}
% for neatly defining theorems and propositions
\usepackage{amsthm}
% making logically defined graphics
%%%\usepackage{xypic}

% there are many more packages, add them here as you need them

% define commands here
\newtheorem*{theorem*}{Theorem}
\newtheorem*{lemma*}{Lemma}
\newtheorem*{corollary*}{Corollary}
\newtheorem{theorem}{Theorem}
\newtheorem{lemma}{Lemma}
\newtheorem{corollary}{Corollary}


\begin{document}
\PMlinkescapeword{right inverse}
\PMlinkescapeword{invertible}
\PMlinkescapeword{equivalent}

If $R$ is a ring and $f\colon M\rightarrow N$ is a homomorphism of $R$-modules, then a right inverse of $f$ is a  homomorphism $g\colon N\rightarrow M$ such that $f\circ g$ is the identity map on $N$. For a right inverse to exist, it is clear that $f$ must be an epimorphism. If a right inverse exists for every such epimorphism and all modules $M$, then $N$ is said to be a projective module.

For fractional ideals over an integral domain $R$, the property of being projective as an $R$-module is equivalent to being an invertible ideal.

\begin{theorem*}
Let $R$ be an integral domain. Then a fractional ideal over $R$ is invertible if and only if it is projective as an $R$-module.
\end{theorem*}

In particular, every fractional ideal over a Dedekind domain is invertible, and is therefore projective.
%%%%%
%%%%%
\end{document}

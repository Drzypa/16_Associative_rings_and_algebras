\documentclass[12pt]{article}
\usepackage{pmmeta}
\pmcanonicalname{GradedAlgebra}
\pmcreated{2013-03-22 12:45:47}
\pmmodified{2013-03-22 12:45:47}
\pmowner{mhale}{572}
\pmmodifier{mhale}{572}
\pmtitle{graded algebra}
\pmrecord{8}{33071}
\pmprivacy{1}
\pmauthor{mhale}{572}
\pmtype{Definition}
\pmcomment{trigger rebuild}
\pmclassification{msc}{16W50}
\pmrelated{GradedModule}
\pmrelated{SuperAlgebra}
\pmrelated{LieSuperalgebra}
\pmrelated{LieSuperalgebra3}

% this is the default PlanetMath preamble.  as your knowledge
% of TeX increases, you will probably want to edit this, but
% it should be fine as is for beginners.

% almost certainly you want these
\usepackage{amssymb}
\usepackage{amsmath}
\usepackage{amsfonts}

% used for TeXing text within eps files
%\usepackage{psfrag}
% need this for including graphics (\includegraphics)
%\usepackage{graphicx}
% for neatly defining theorems and propositions
%\usepackage{amsthm}
% making logically defined graphics
%%%\usepackage{xypic} 

% there are many more packages, add them here as you need them

% define commands here
\begin{document}
An algebra $A$ over a graded ring $B$ is \emph{graded} if it is itself a graded ring and a graded module over $B$ such that
$$A^p \cdot A^q \subseteq A^{p+q}$$
where $A^i$, $i \in \mathbb{N}$, are submodules of $A$.
More generally, one can replace $\mathbb{N}$ by a monoid or semigroup $G$.
In which case, $A$ is called a $G$-graded algebra.
A graded algebra then is the same thing as an $\mathbb{N}$-graded algebra.

Examples of graded algebras include the polynomial ring $k[X]$ being an $\mathbb{N}$-graded $k$-algebra, and the exterior algebra.
%%%%%
%%%%%
\end{document}

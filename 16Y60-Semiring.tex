\documentclass[12pt]{article}
\usepackage{pmmeta}
\pmcanonicalname{Semiring}
\pmcreated{2013-03-22 12:27:46}
\pmmodified{2013-03-22 12:27:46}
\pmowner{mps}{409}
\pmmodifier{mps}{409}
\pmtitle{semiring}
\pmrecord{11}{32617}
\pmprivacy{1}
\pmauthor{mps}{409}
\pmtype{Definition}
\pmcomment{trigger rebuild}
\pmclassification{msc}{16Y60}
%\pmkeywords{partial order}
%\pmkeywords{poset}
\pmrelated{Ring}
\pmrelated{KleeneAlgebra}

% this is the default PlanetMath preamble.  as your knowledge
% of TeX increases, you will probably want to edit this, but
% it should be fine as is for beginners.

% almost certainly you want these
\usepackage{amssymb}
\usepackage{amsmath}
\usepackage{amsfonts}

% used for TeXing text within eps files
%\usepackage{psfrag}
% need this for including graphics (\includegraphics)
%\usepackage{graphicx}
% for neatly defining theorems and propositions
%\usepackage{amsthm}
% making logically defined graphics
%%%\usepackage{xypic}

% there are many more packages, add them here as you need them

% define commands here
\begin{document}
\PMlinkescapeword{algebra}
\PMlinkescapeword{constants}
\PMlinkescapephrase{right annihilator}
\PMlinkescapeword{cycle}

%A \emph{semiring} is an algebra $(A, \cdot, +, 0, 1)$ over a set $A$, 
%where 0 and 1 are constants, $(A, \cdot, 1)$ is a monoid, $(A, +, 0)$ is a %commutative monoid, $\cdot$ \PMlinkname{distributes}{Distributivity} over $+$ %from the left and right, and 0 is both a left and right annihilator ($0a = a0 = %0$).
%Often $a\cdot b$ is written simply as $ab$, and the semiring $(A, \cdot, +, 0, %1)$ as simply $A$.

A \emph{semiring} is a set $A$ with two operations, $+$ and $\cdot$, such that
$0\in A$ makes $(A,+)$ into a commutative monoid, $1\in A$ makes $(A,\cdot)$
into a monoid, the operation $\cdot$ \PMlinkname{distributes}{Distributivity}
over $+$, and for any $a\in A$, $0\cdot a=a\cdot 0=0$.  Usually, $a\cdot b$
is instead written $ab$.

A ring $(R,+,\cdot)$, can be described as a semiring for which $(R,+)$ is
required to be a group.  Thus every ring is a semiring.  
The natural numbers
$\mathbb{N}$ form a semiring, but not a ring, with the usual multiplication and addition.

Every semiring $A$ has a quasiorder $\preceq$ 
given by $a\preceq b$ if and only if there exists some $c\in A$ such that $a+c=b$.  Any element $a\in A$ with an additive inverse is smaller than
any other element.  Thus if $A$ has a nonzero element $a$ with an additive 
inverse, then the elements $-a$, $0$, $a$ form a cycle with respect to $\preceq$.
If $+$ is an \PMlinkname{idempotent}{Idempotency} operation,
then $\preceq$ is a partial order.
Addition and (left and right) multiplication are 
\PMlinkname{order-preserving operators}{Poset}.
%%%%%
%%%%%
\end{document}

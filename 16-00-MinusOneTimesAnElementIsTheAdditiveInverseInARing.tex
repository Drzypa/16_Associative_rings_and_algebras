\documentclass[12pt]{article}
\usepackage{pmmeta}
\pmcanonicalname{MinusOneTimesAnElementIsTheAdditiveInverseInARing}
\pmcreated{2013-03-22 14:14:00}
\pmmodified{2013-03-22 14:14:00}
\pmowner{alozano}{2414}
\pmmodifier{alozano}{2414}
\pmtitle{minus one times an element is the additive inverse in a ring}
\pmrecord{9}{35674}
\pmprivacy{1}
\pmauthor{alozano}{2414}
\pmtype{Theorem}
\pmcomment{trigger rebuild}
\pmclassification{msc}{16-00}
\pmclassification{msc}{13-00}
\pmclassification{msc}{20-00}
\pmsynonym{$(-1)\cdot a= -a$}{MinusOneTimesAnElementIsTheAdditiveInverseInARing}
\pmrelated{0cdotA0}

\endmetadata

% this is the default PlanetMath preamble.  as your knowledge
% of TeX increases, you will probably want to edit this, but
% it should be fine as is for beginners.

% almost certainly you want these
\usepackage{amssymb}
\usepackage{amsmath}
\usepackage{amsthm}
\usepackage{amsfonts}

% used for TeXing text within eps files
%\usepackage{psfrag}
% need this for including graphics (\includegraphics)
%\usepackage{graphicx}
% for neatly defining theorems and propositions
%\usepackage{amsthm}
% making logically defined graphics
%%%\usepackage{xypic}

% there are many more packages, add them here as you need them

% define commands here

\newtheorem{thm}{Theorem}
\newtheorem{defn}{Definition}
\newtheorem{prop}{Proposition}
\newtheorem{lemma}{Lemma}
\newtheorem{cor}{Corollary}

% Some sets
\newcommand{\Nats}{\mathbb{N}}
\newcommand{\Ints}{\mathbb{Z}}
\newcommand{\Reals}{\mathbb{R}}
\newcommand{\Complex}{\mathbb{C}}
\newcommand{\Rats}{\mathbb{Q}}
\begin{document}
\begin{lemma}
Let $R$ be a ring (with unity $1$) and let $a$ be an element of $R$. Then 
$$(-1)\cdot a = -a$$
where $-1$ is the additive inverse of $1$ and $-a$ is the additive inverse of $a$.
\end{lemma}
\begin{proof}
Note that for any $a$ in $R$ there exists a unique ``$-a$'' by the uniqueness of additive inverse in a ring. We check that $(-1)\cdot a$ equals the additive inverse of $a$.
\begin{eqnarray*}
a+(-1)\cdot a &=& 1\cdot a + (-1)\cdot a, \quad \text{ by the definition of }1\\
&=& (1+ (-1))\cdot a, \quad \text{ by the distributive law}\\
&=& 0\cdot a,\quad \text{ by the definition of }-1\\
&=& 0, \quad \text{ as a result of the properties of zero} 
\end{eqnarray*}
Hence $(-1)\cdot a$ is ``an'' additive inverse for $a$, and by uniqueness $(-1)\cdot a = -a$, {\it the} additive inverse of $a$. Analogously, we can prove that $a\cdot (-1) = -a$ as well.
\end{proof}
%%%%%
%%%%%
\end{document}

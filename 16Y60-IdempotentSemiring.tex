\documentclass[12pt]{article}
\usepackage{pmmeta}
\pmcanonicalname{IdempotentSemiring}
\pmcreated{2013-03-22 15:52:12}
\pmmodified{2013-03-22 15:52:12}
\pmowner{CWoo}{3771}
\pmmodifier{CWoo}{3771}
\pmtitle{idempotent semiring}
\pmrecord{8}{37866}
\pmprivacy{1}
\pmauthor{CWoo}{3771}
\pmtype{Definition}
\pmcomment{trigger rebuild}
\pmclassification{msc}{16Y60}
\pmsynonym{i-semiring}{IdempotentSemiring}
\pmsynonym{dioid}{IdempotentSemiring}

\endmetadata

\usepackage{amssymb,amscd}
\usepackage{amsmath}
\usepackage{amsfonts}

% used for TeXing text within eps files
%\usepackage{psfrag}
% need this for including graphics (\includegraphics)
%\usepackage{graphicx}
% for neatly defining theorems and propositions
%\usepackage{amsthm}
% making logically defined graphics
%%%\usepackage{xypic}

% define commands here
\begin{document}
A semiring $S$ is called an \emph{idempotent semiring}, or \emph{i-semiring} for short, if, addition $+$ is an idempotent binary operation:
$$a+a=a,\qquad\mbox{ for all }a\in S.$$

\textbf{Some properties of an i-semiring $S$}.
\begin{enumerate}
\item If we define a binary relation $\le$ on $S$ by 
$$a\le b\qquad\mbox{ iff }\qquad a+b=b$$
then $\le$ becomes a partial order on $S$.  Indeed, for $a+a=a$ implies $a\le a$; if $a\le b$ and $b\le a$, then $b=a+b=a$; and finally, if $a\le b$ and $b\le c$, then $a+c=a+(b+c)=(a+b)+c=b+c=c$ so $a\le c$.
\item $0\le a$ for any $a\in S$, because $0+a=a$.
\item Define $a\vee b$ as the supremum of $a$ and $b$ (with respect to $\le$).  Then $a\vee b$ exists and $$a\vee b=a+b.$$ To see this, we have $a+(a+b)=(a+a)+b=a+b$, so $a\le a+b$.  Similarly $b\le a+b$.  If $a\le c$ and $b\le c$, then $(a+b)+c=a+(b+c)=a+c=c$.  So $a+b\le c$.
\item Collecting all the information above, we see that $(S,+)$ is an upper semilattice with $+$ as the join operation on $S$ and $0$ the bottom element.
\item Additon and multiplication respect partial ordering: suppose $a\le b$, then for any $c\in S$, $(c+a)+(c+b)=(c+c)+(a+b)=c+b$, hence $c+a\le c+b$; also, $cb=c(a+b)=ca+cb$ implies $ca\le cb$.
\end{enumerate}

\textbf{Remark}.  $S$ in general is not a lattice, and $1$ is not the top element of $S$.

The main example of an i-semiring is a Kleene algebra used in the theory of computations.
%%%%%
%%%%%
\end{document}

\documentclass[12pt]{article}
\usepackage{pmmeta}
\pmcanonicalname{DivisionAlgebra}
\pmcreated{2013-03-22 16:52:03}
\pmmodified{2013-03-22 16:52:03}
\pmowner{Algeboy}{12884}
\pmmodifier{Algeboy}{12884}
\pmtitle{division algebra}
\pmrecord{6}{39117}
\pmprivacy{1}
\pmauthor{Algeboy}{12884}
\pmtype{Definition}
\pmcomment{trigger rebuild}
\pmclassification{msc}{16K99}
\pmrelated{octonion}
\pmrelated{Octonion}
\pmdefines{division algebra}

\endmetadata

\usepackage{latexsym}
\usepackage{amssymb}
\usepackage{amsmath}
\usepackage{amsfonts}
\usepackage{amsthm}

\usepackage{enumerate}

%%\usepackage{xypic}

%-----------------------------------------------------

%       Standard theoremlike environments.

%       Stolen directly from AMSLaTeX sample

%-----------------------------------------------------

%% \theoremstyle{plain} %% This is the default

\newtheorem{thm}{Theorem}

\newtheorem{coro}[thm]{Corollary}

\newtheorem{lem}[thm]{Lemma}

\newtheorem{lemma}[thm]{Lemma}

\newtheorem{prop}[thm]{Proposition}

\newtheorem{conjecture}[thm]{Conjecture}

\newtheorem{conj}[thm]{Conjecture}

\newtheorem{defn}[thm]{Definition}

\newtheorem{remark}[thm]{Remark}

\newtheorem{ex}[thm]{Example}



%\countstyle[equation]{thm}



%--------------------------------------------------

%       Item references.

%--------------------------------------------------


\newcommand{\exref}[1]{Example-\ref{#1}}

\newcommand{\thmref}[1]{Theorem-\ref{#1}}

\newcommand{\defref}[1]{Definition-\ref{#1}}

\newcommand{\eqnref}[1]{(\ref{#1})}

\newcommand{\secref}[1]{Section-\ref{#1}}

\newcommand{\lemref}[1]{Lemma-\ref{#1}}

\newcommand{\propref}[1]{Prop\-o\-si\-tion-\ref{#1}}

\newcommand{\corref}[1]{Cor\-ol\-lary-\ref{#1}}

\newcommand{\figref}[1]{Fig\-ure-\ref{#1}}

\newcommand{\conjref}[1]{Conjecture-\ref{#1}}


% Normal subgroup or equal.

\providecommand{\normaleq}{\unlhd}

% Normal subgroup.

\providecommand{\normal}{\lhd}

\providecommand{\rnormal}{\rhd}
% Divides, does not divide.

\providecommand{\divides}{\mid}

\providecommand{\ndivides}{\nmid}


\providecommand{\union}{\cup}

\providecommand{\bigunion}{\bigcup}

\providecommand{\intersect}{\cap}

\providecommand{\bigintersect}{\bigcap}










\begin{document}
Let $K$ be a unital ring and $A$ a $K$-algebra.  Defining ``division''
requires special considerations when the algebras are non-associative
so we introduce the definition in stages.

\section{Associative division algebras}

If $A$ is an 
associative algebra then we say $A$ is a \emph{division algebra}
if
\begin{enumerate}[(i)]
\item $A$ is unital with identity $1$.  So for all $a\in A$,
\[a1=1a=a.\]
\item Also every non-zero element of $A$ has an inverse.  That is
$a\in A$, $a\neq 0$, then there exists a $b\in A$ such that
\[ab=1=ba.\]
We denote $b$ by $a^{-1}$ and we may prove $a^{-1}$ is unique to $a$.
\end{enumerate}

The standard examples of associative division algebras are fields, which
are commutative, and the non-split quaternion algebra: $\alpha,\beta\in K$,
\[\left(\frac{\alpha,\beta}{K}\right)=\left\{
a_1 1+a_2 i+a_3 j+a_4 k : i^2=\alpha 1, j^2=\beta 1, k^2=-\alpha \beta 1, ij=k=-ji.\right\}\]
where $x^2-\alpha$ and $x^2-\beta$ are irreducible over $K$.

\section{Non-associative division algebras}

For non-associative algebras $A$, the notion of an inverse is not immediate.
We use $x.y$ for the product of $x,y\in A$.

\textbf{Invertible as endomorphisms:}  Let $a\in A$.  Then define $L_a:x\mapsto a.x$
and $R_a:x\mapsto x.a$.  As the product of $A$ is distributive, both $L_a$ an $R_a$
are additive endomorphisms of $A$.  If $L_a$ is invertible then we may call $a$
``left invertible'' and similarly, when $R_a$ is invertible we may call $a$
``right invertible'' and ``invertible'' if both $L_a$ and $R_a$ are invertible.

In this model of invertible, $A$ is a \emph{division algebra} if, and only if,
for each non-zero $a\in A$, both $L_a$ and $R_a$ invertible.
Equivalently: the equations $a.x=b$ and $y.a=b$ have unique solutions
for nonzero $a,b\in A$.  However, $x$ and $y$ need not be equal.

A common method to produce non-associative division algebras of this sort is
through Schur's Lemma.  

\textbf{Invertible in the product:}
In some instances, the notion of invertible via endomorphisms is not 
sufficient.  Instead, assume $A$ has an identity, that is, an element $1\in A$ such
that for all $a\in A$,
\[1.a=a=a.1.\]

Next if $a\in A$, we say $a$ is \emph{invertible} if there exists a $b\in A$
such that 
\begin{equation}\label{eq:inv}
a.b=1=b.a
\end{equation}
and furthermore that for all $x\in A$,
\begin{equation}\label{eq:inv-non-a}
b.(a.x)=x=(x.a).b.
\end{equation}
Evidently (\ref{eq:inv}) can be inferred from (\ref{eq:inv-non-a}).
This added assumption substitutes for the need of associativity in the
proofs of uniqueness of inverses and in solving equations with non-associative
products.  

\begin{prop}
If $A$ is a finite dimensional algebra over a field, then
invertible in this sense forces both $L_a$ and $R_a$ to be invertible as well.
\end{prop}
\begin{proof}
Let $x\in A$.  Then $xL_1=1.x=x=b.(a.x)=x L_a L_b$.  So $L_1=L_a L_b$.  As
$L_1$ is the identity map, $L_a$ is injective and $L_b$ is surjective.
As $A$ is finite dimensional, injective and surjective endomorphisms are
bijective.
\end{proof}


In this model, a non-associative algebra is a division algebra $A$ if it is
unital and every non-zero element is invertible.

\section{Alternative division algebras}

The standard examples of non-associative division algebras are actually 
alternative alegbras, specfically, the composition algebras of fields,
non-split quaternions and non-split octonions -- only the latter are
actually not associative.  Invertible in the octonions is interpreted
in the second stronger form.

\begin{thm}[Bruck-Klienfeld]
Every alternative division algebra is either associative or a non-split
octonion.
\end{thm}

This result is usually followed by two useful results which serve to omit
the need to consider non-associative examples.

\begin{thm}[Artin-Zorn, Wedderburn]
A finite alternative division algebra is associative and commutative, so 
it is a finite field.
\end{thm}

\begin{thm}
An alternative division algebra over an algebraically closed field is
the field itself.
\end{thm}

%%%%%
%%%%%
\end{document}

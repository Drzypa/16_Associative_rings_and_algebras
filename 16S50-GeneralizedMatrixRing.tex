\documentclass[12pt]{article}
\usepackage{pmmeta}
\pmcanonicalname{GeneralizedMatrixRing}
\pmcreated{2013-03-22 14:39:14}
\pmmodified{2013-03-22 14:39:14}
\pmowner{mclase}{549}
\pmmodifier{mclase}{549}
\pmtitle{generalized matrix ring}
\pmrecord{4}{36245}
\pmprivacy{1}
\pmauthor{mclase}{549}
\pmtype{Definition}
\pmcomment{trigger rebuild}
\pmclassification{msc}{16S50}

\endmetadata

% this is the default PlanetMath preamble.  as your knowledge
% of TeX increases, you will probably want to edit this, but
% it should be fine as is for beginners.

% almost certainly you want these
\usepackage{amssymb}
\usepackage{amsmath}
\usepackage{amsfonts}

% used for TeXing text within eps files
%\usepackage{psfrag}
% need this for including graphics (\includegraphics)
%\usepackage{graphicx}
% for neatly defining theorems and propositions
%\usepackage{amsthm}
% making logically defined graphics
%%%\usepackage{xypic}

% there are many more packages, add them here as you need them

% define commands here
\begin{document}
Let $I$ be an indexing set.  A {\em ring of $I \times I$ generalized matrices}
is a ring $R$ with a decompostion (as an additive group)
$$R = \bigoplus_{i,j \in I} R_{ij}, $$
such that $R_{ij} R_{kl} \subseteq R_{il}$ if $j = k$ and
$R_{ij} R_{kl} = 0$ if $j \neq k$.

If $I$ is finite, then we usually replace it by its cardinal $n$
and speak of a ring of $n \times n$ generalized matrices with components
$R_{ij}$ for $i \le i,j \le n$.

If we arrange the components $R_{ij}$ as follows:

\begin{displaymath}
  \begin{pmatrix}
    R_{11} & R_{12} & \dots & R_{1n} \\
    R_{21} & R_{22} & \dots & R_{2n} \\
    \vdots & \vdots & \ddots & \vdots \\
    R_{n1} & R_{n2} & \dots & R_{nn}
  \end{pmatrix}
\end{displaymath}

and we write elements of $R$ in the same fashion, then the multiplication in $R$ follows
the same pattern as ordinary matrix multiplication.

Note that $R_{ij}$ is an $R_{ii}$-$R_{jj}$-bimodule,
and the multiplication of elements induces homomorphisms
$R_{ij} \otimes_{R_{jj}} R_{jk} \to R_{ik}$ for all $i, j, k$.

Conversely, given a collection of rings $R_i$,
and for each $i \neq j$ an $R_i$-$R_j$-bimodule $R_{ij}$,
and for each $i, j,  k$ with $i \neq j$ and $j \neq k$
a homomorphism $R_{ij} \otimes_{R_{j}} R_{jk} \to R_{ik}$,
we can construct a generalized matrix ring structure on
$$R = \bigoplus_{i,j} R_{ij},$$
where we take $R_{ii} = R_i$.
%%%%%
%%%%%
\end{document}

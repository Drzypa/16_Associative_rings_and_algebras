\documentclass[12pt]{article}
\usepackage{pmmeta}
\pmcanonicalname{IdealMultiplicationLaws}
\pmcreated{2014-05-11 17:05:52}
\pmmodified{2014-05-11 17:05:52}
\pmowner{pahio}{2872}
\pmmodifier{pahio}{2872}
\pmtitle{ideal multiplication laws}
\pmrecord{16}{37711}
\pmprivacy{1}
\pmauthor{pahio}{2872}
\pmtype{Definition}
\pmcomment{trigger rebuild}
\pmclassification{msc}{16D25}
\pmsynonym{laws of ideal product}{IdealMultiplicationLaws}
%\pmkeywords{ideal product}
%\pmkeywords{ideal multiplication}
\pmrelated{DivisibilityInRings}
\pmrelated{ProductOfLeftAndRightIdeal}
\pmrelated{InvertibilityOfRegularlyGeneratedIdeal}

% this is the default PlanetMath preamble.  as your knowledge
% of TeX increases, you will probably want to edit this, but
% it should be fine as is for beginners.

% almost certainly you want these
\usepackage{amssymb}
\usepackage{amsmath}
\usepackage{amsfonts}

% used for TeXing text within eps files
%\usepackage{psfrag}
% need this for including graphics (\includegraphics)
%\usepackage{graphicx}
% for neatly defining theorems and propositions
 \usepackage{amsthm}
% making logically defined graphics
%%%\usepackage{xypic}

% there are many more packages, add them here as you need them

% define commands here

\theoremstyle{definition}
\newtheorem*{thmplain}{Theorem}

\begin{document}
The \PMlinkname{multiplication}{ProductOfIdeals} of the (two-sided) ideals of any ring $R$ has following properties:
\begin{enumerate}
\item $(0)\mathfrak{a = a}(0) = (0)$
\item $\mathfrak{(ab)c = a(bc)}$
\item $\mathfrak{a(b+c) = ab+ac, \quad (a+b)c = ac+bc}$
\item If $R$ has a unity, then\, $R\mathfrak{a} = \mathfrak{a}R = \mathfrak{a}$
\item If $R$ is commutative, then\, $\mathfrak{ab = ba}$
\item $\mathfrak{ab \subseteq a\cap b}$
\item $\mathfrak{a(b\cap c) \subseteq ab\cap ac}$
\item $\mathfrak{a\subseteq b\quad\Rightarrow\quad ac\subseteq bc}$\end{enumerate}


\textbf{Remark.}\, The properties 1, 2, 3, 4 together with the properties
$$\mathfrak{(a+b)+c = a+(b+c),\qquad a+b = b+a,\qquad a}+(0) = \mathfrak{a}$$
of the ideal addition make the set $A$ of all ideals of $R$ to a semiring\, $(A,\,+,\,\cdot)$.\, It is not a ring, since no non-zero ideal of $R$ has the \PMlinkname{additive inverse}{Ring}.

\begin{thebibliography}{9}
\bibitem{Larsen & McCarthy} {\sc M. Larsen \& P. McCarthy}: {\it Multiplicative theory of ideals}.\, Academic Press, New York (1971).
\end{thebibliography}

%%%%%
%%%%%
\end{document}

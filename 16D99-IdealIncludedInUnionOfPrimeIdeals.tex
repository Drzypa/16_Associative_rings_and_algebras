\documentclass[12pt]{article}
\usepackage{pmmeta}
\pmcanonicalname{IdealIncludedInUnionOfPrimeIdeals}
\pmcreated{2013-03-22 16:53:14}
\pmmodified{2013-03-22 16:53:14}
\pmowner{polarbear}{3475}
\pmmodifier{polarbear}{3475}
\pmtitle{ideal included in union of prime ideals}
\pmrecord{10}{39142}
\pmprivacy{1}
\pmauthor{polarbear}{3475}
\pmtype{Result}
\pmcomment{trigger rebuild}
\pmclassification{msc}{16D99}
\pmclassification{msc}{13C99}
\pmsynonym{prime avoidance lemma}{IdealIncludedInUnionOfPrimeIdeals}
\pmrelated{IdealsContainedInAUnionOfIdeals}

\endmetadata

% this is the default PlanetMath preamble.  as your knowledge
% of TeX increases, you will probably want to edit this, but
% it should be fine as is for beginners.

% almost certainly you want these
\usepackage{amssymb}
\usepackage{amsmath}
\usepackage{amsfonts}

% used for TeXing text within eps files
%\usepackage{psfrag}
% need this for including graphics (\includegraphics)
%\usepackage{graphicx}
% for neatly defining theorems and propositions
\usepackage{amsthm}
% making logically defined graphics
%%\usepackage{xypic}

% there are many more packages, add them here as you need them
\newtheorem{proposition}{Proposition}
\newtheorem{corollary}{Corollary}
% define commands here

\begin{document}
 In the following $R$ is a commutative ring with unity.
\begin{proposition} Let $I$ be an ideal of the ring $R$ and $P_1, P_2, ... ,P_n$ be prime ideals of $R$. If $I\not\subseteq P_i$, for all $i$, then $I \not\subseteq \cup P_i$.\end{proposition}\begin{proof} We will prove by induction on $n$. For $n=1$ the proof is trivial. Assume now that the result is true for $n-1$. That implies the existence, for each $i$, of an element $s_i$ such that $s_i\in I$ and $s_i \not\in \bigcup_{j\ne i}P_j$. If for some $i$, $s_i\not\in P_i$ then we are done. Thus, we may consider only the case $s_i \in P_i$, for all $i$.\newline
 Let $a_i = r_1...r_{i-1}r_{i+1}...r_n$. Since $P_i$ is prime then $a_i\not\in P_i$, for all $i$. Moreover, for $j \ne i$, the element $a_i \in P_j$. Consider the element $a = \sum a_j \in I$. Since $a_i = a - \sum_{i\ne j} a_j$ and $\sum_{i\ne j} a_j\;\in P_i$, it follows that $a \not\in P_i$, otherwise $a_i\in P_i$, contradiction. The existence of the element $a$ proves the proposition.\end{proof}
\begin{corollary} Let $I$ be an ideal of the ring $R$ and $P_1,P_2,...,P_n$ be prime ideals of $R$. If $I \subseteq \cup P_i$, then $I\subseteq P_i$, for some $i$.\end{corollary}    
%%%%%
%%%%%
\end{document}

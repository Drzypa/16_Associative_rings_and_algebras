\documentclass[12pt]{article}
\usepackage{pmmeta}
\pmcanonicalname{AnArtinianIntegralDomainIsAField}
\pmcreated{2013-03-22 12:49:37}
\pmmodified{2013-03-22 12:49:37}
\pmowner{yark}{2760}
\pmmodifier{yark}{2760}
\pmtitle{an Artinian integral domain is a field}
\pmrecord{13}{33150}
\pmprivacy{1}
\pmauthor{yark}{2760}
\pmtype{Theorem}
\pmcomment{trigger rebuild}
\pmclassification{msc}{16P20}
\pmclassification{msc}{13G05}
\pmrelated{AFiniteIntegralDomainIsAField}

\endmetadata

\usepackage{amssymb}
\usepackage{amsmath}
\usepackage{amsfonts}

\newcommand{\N}{\mathbb{N}}
\begin{document}
\PMlinkescapeword{multiplicative}

Let $R$ be an integral domain, and assume that $R$ is Artinian.

Let $a \in R$ with $a \neq 0$.
Then $R \supseteq aR \supseteq a^2R \supseteq \cdots$.

As $R$ is Artinian, there is some $n\in\N$ such that $a^nR=a^{n+1}R$.
There exists $r \in R$ such that $a^n=a^{n+1}r$,
that is, $a^n1=a^n(ar)$.
But $a^n \neq 0$ (as $R$ is an integral domain), so we have $1=ar$.
Thus $a$ is a unit.

Therefore, every Artinian integral domain is a field.
%%%%%
%%%%%
\end{document}

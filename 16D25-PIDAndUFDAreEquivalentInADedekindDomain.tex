\documentclass[12pt]{article}
\usepackage{pmmeta}
\pmcanonicalname{PIDAndUFDAreEquivalentInADedekindDomain}
\pmcreated{2013-03-22 17:53:45}
\pmmodified{2013-03-22 17:53:45}
\pmowner{rm50}{10146}
\pmmodifier{rm50}{10146}
\pmtitle{PID and UFD are equivalent in a Dedekind domain}
\pmrecord{7}{40384}
\pmprivacy{1}
\pmauthor{rm50}{10146}
\pmtype{Theorem}
\pmcomment{trigger rebuild}
\pmclassification{msc}{16D25}
\pmclassification{msc}{13G05}
\pmclassification{msc}{13A15}
\pmclassification{msc}{11N80}

% this is the default PlanetMath preamble.  as your knowledge
% of TeX increases, you will probably want to edit this, but
% it should be fine as is for beginners.

% almost certainly you want these
\usepackage{amssymb}
\usepackage{amsmath}
\usepackage{amsfonts}

% used for TeXing text within eps files
%\usepackage{psfrag}
% need this for including graphics (\includegraphics)
%\usepackage{graphicx}
% for neatly defining theorems and propositions
%\usepackage{amsthm}
% making logically defined graphics
%%%\usepackage{xypic}

% there are many more packages, add them here as you need them

% define commands here

\begin{document}
This article shows that if $A$ is a Dedekind domain, then $A$ is a UFD if and only if it is a PID. Note that this result implies the more specific result given in the article unique factorization and ideals in ring of integers.

Since any PID is a UFD, we need only prove the other direction. So assume $A$ is a UFD, let $\mathfrak{p}$ be a nonzero (proper) prime ideal, and choose $0\neq x\in\mathfrak{p}$. Note that $x$ is a nonunit since $\mathfrak{p}$ is a proper ideal. Since $A$ is a UFD, we may write $x$ uniquely (up to units) as $x=p_1^{a_1}\cdots p_k^{a_k}$ where the $p_i$ are distinct irreducibles in $A$, the $a_i$ are positive integers, and $k>0$ since $x$ is not a unit. Since $\mathfrak{p}$ is prime and $x\in\mathfrak{p}$, it follows that some $p_i$, say $p_1$, is in $\mathfrak{p}$. Then $(p_1)\subset\mathfrak{p}$. But $(p_1)$ is prime since clearly in a UFD any ideal generated by an irreducible is prime. Since $A$ is Dedekind and thus has Krull dimension 1, it must be that $(p_1)=\mathfrak{p}$ and thus $\mathfrak{p}$ is principal.

%%%%%
%%%%%
\end{document}

\documentclass[12pt]{article}
\usepackage{pmmeta}
\pmcanonicalname{FriedrichsTheorem}
\pmcreated{2013-03-22 16:51:16}
\pmmodified{2013-03-22 16:51:16}
\pmowner{Algeboy}{12884}
\pmmodifier{Algeboy}{12884}
\pmtitle{Friedrichs' theorem}
\pmrecord{6}{39101}
\pmprivacy{1}
\pmauthor{Algeboy}{12884}
\pmtype{Theorem}
\pmcomment{trigger rebuild}
\pmclassification{msc}{16S30}
\pmclassification{msc}{17B35}

\endmetadata

\usepackage{latexsym}
\usepackage{amssymb}
\usepackage{amsmath}
\usepackage{amsfonts}
\usepackage{amsthm}

%%\usepackage{xypic}

%-----------------------------------------------------

%       Standard theoremlike environments.

%       Stolen directly from AMSLaTeX sample

%-----------------------------------------------------

%% \theoremstyle{plain} %% This is the default

\newtheorem{thm}{Theorem}

\newtheorem{coro}[thm]{Corollary}

\newtheorem{lem}[thm]{Lemma}

\newtheorem{lemma}[thm]{Lemma}

\newtheorem{prop}[thm]{Proposition}

\newtheorem{conjecture}[thm]{Conjecture}

\newtheorem{conj}[thm]{Conjecture}

\newtheorem{defn}[thm]{Definition}

\newtheorem{remark}[thm]{Remark}

\newtheorem{ex}[thm]{Example}



%\countstyle[equation]{thm}



%--------------------------------------------------

%       Item references.

%--------------------------------------------------


\newcommand{\exref}[1]{Example-\ref{#1}}

\newcommand{\thmref}[1]{Theorem-\ref{#1}}

\newcommand{\defref}[1]{Definition-\ref{#1}}

\newcommand{\eqnref}[1]{(\ref{#1})}

\newcommand{\secref}[1]{Section-\ref{#1}}

\newcommand{\lemref}[1]{Lemma-\ref{#1}}

\newcommand{\propref}[1]{Prop\-o\-si\-tion-\ref{#1}}

\newcommand{\corref}[1]{Cor\-ol\-lary-\ref{#1}}

\newcommand{\figref}[1]{Fig\-ure-\ref{#1}}

\newcommand{\conjref}[1]{Conjecture-\ref{#1}}


% Normal subgroup or equal.

\providecommand{\normaleq}{\unlhd}

% Normal subgroup.

\providecommand{\normal}{\lhd}

\providecommand{\rnormal}{\rhd}
% Divides, does not divide.

\providecommand{\divides}{\mid}

\providecommand{\ndivides}{\nmid}


\providecommand{\union}{\cup}

\providecommand{\bigunion}{\bigcup}

\providecommand{\intersect}{\cap}

\providecommand{\bigintersect}{\bigcap}










\begin{document}
Fix a commutative unital ring $K$ of characteristic 0.  Let $X$ be a finite
set and $K\langle X\rangle$ the free associative algebra on $X$.  Then define
the map $\delta:K\langle X\rangle\rightarrow K\langle X\rangle\otimes K\langle X\rangle$ by $x\mapsto x\otimes 1+1\otimes x$.

\begin{thm}[Friedrichs]\cite[Thm V.9]{Jacobson}
An element $a\in K\langle X\rangle$ is a Lie element if and only if
$a\delta=a\otimes 1+1\otimes a$.
\end{thm}

The term Lie element applies only when an element is taken from the universal
enveloping algebra of a Lie algebra.  Here the Lie algebra in question is
the free Lie algebra on $X$, $FL\langle X\rangle$ whose universal enveloping 
algebra is $K\langle X\rangle$ by a theorem of Witt.

This characterization of Lie elements is a primary means in modern proofs
of the Baker-Campbell-Hausdorff formula.

\bibliographystyle{amsplain}
\begin{thebibliography}{10}
\bibitem{Jacobson}
Nathan Jacobson \emph{Lie Algebras}, Interscience Publishers, New York, 1962.

\end{thebibliography}

%%%%%
%%%%%
\end{document}

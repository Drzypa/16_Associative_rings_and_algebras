\documentclass[12pt]{article}
\usepackage{pmmeta}
\pmcanonicalname{UpperNilradical}
\pmcreated{2013-03-22 17:29:06}
\pmmodified{2013-03-22 17:29:06}
\pmowner{CWoo}{3771}
\pmmodifier{CWoo}{3771}
\pmtitle{upper nilradical}
\pmrecord{4}{39872}
\pmprivacy{1}
\pmauthor{CWoo}{3771}
\pmtype{Definition}
\pmcomment{trigger rebuild}
\pmclassification{msc}{16N40}

\endmetadata

\usepackage{amssymb,amscd}
\usepackage{amsmath}
\usepackage{amsfonts}
\usepackage{mathrsfs}

% used for TeXing text within eps files
%\usepackage{psfrag}
% need this for including graphics (\includegraphics)
%\usepackage{graphicx}
% for neatly defining theorems and propositions
\usepackage{amsthm}
% making logically defined graphics
%%\usepackage{xypic}
\usepackage{pst-plot}
\usepackage{psfrag}

% define commands here
\newtheorem{prop}{Proposition}
\newtheorem{thm}{Theorem}
\newtheorem{ex}{Example}
\newcommand{\real}{\mathbb{R}}
\newcommand{\pdiff}[2]{\frac{\partial #1}{\partial #2}}
\newcommand{\mpdiff}[3]{\frac{\partial^#1 #2}{\partial #3^#1}}
\begin{document}
The \emph{upper nilradical} $\operatorname{Nil}^*(R)$ of $R$ is the \PMlinkname{sum}{SumOfIdeals} of all (two-sided) nil ideals in $R$.  In other words, $a\in Nil^* R$ iff $a$ can be expressed as a (finite) sum of nilpotent elements.

It is not hard to see that $\operatorname{Nil}^*(R)$ is the largest nil ideal in $R$.  Furthermore, we have that  $\operatorname{Nil}_*(R)\subseteq \operatorname{Nil}^*(R)\subseteq J(R)$, where $\operatorname{Nil}_*(R)$ is the lower radical or prime radical of $R$, and $J(R)$ is the Jacobson radical of $R$.

\textbf{Remarks}.  
\begin{itemize}
\item
If $R$ is commutative, then $\operatorname{Nil}_*(R)=\operatorname{Nil}^*(R)=\operatorname{Nil}(R)$, the nilradical of $R$, consisting of all nilpotent elements.
\item
If $R$ is left (or right) artinian, then $\operatorname{Nil}_*(R)=\operatorname{Nil}^*(R)=J(R)$.
\end{itemize}
%%%%%
%%%%%
\end{document}

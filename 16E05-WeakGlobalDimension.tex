\documentclass[12pt]{article}
\usepackage{pmmeta}
\pmcanonicalname{WeakGlobalDimension}
\pmcreated{2013-03-22 19:18:42}
\pmmodified{2013-03-22 19:18:42}
\pmowner{joking}{16130}
\pmmodifier{joking}{16130}
\pmtitle{weak global dimension}
\pmrecord{4}{42249}
\pmprivacy{1}
\pmauthor{joking}{16130}
\pmtype{Definition}
\pmcomment{trigger rebuild}
\pmclassification{msc}{16E05}

\endmetadata

% this is the default PlanetMath preamble.  as your knowledge
% of TeX increases, you will probably want to edit this, but
% it should be fine as is for beginners.

% almost certainly you want these
\usepackage{amssymb}
\usepackage{amsmath}
\usepackage{amsfonts}

% used for TeXing text within eps files
%\usepackage{psfrag}
% need this for including graphics (\includegraphics)
%\usepackage{graphicx}
% for neatly defining theorems and propositions
%\usepackage{amsthm}
% making logically defined graphics
%%%\usepackage{xypic}

% there are many more packages, add them here as you need them

% define commands here

\begin{document}
Let $R$ be a ring. The \textbf{(right) weak global dimension} of $R$ is defined as
$$\mathrm{w.gl.dim}R=\mathrm{sup}\{\mathrm{wd}_RM\ |\ M\mbox{ is a right module}\}.$$

Unlike global dimension of $R$ the definition of the weak global dimension is left/right symmetric. This follows from the fact that for every left module $M$ and right module $N$ there is an isomorphism
$$\mathrm{Tor}_n^R(M,N)\simeq\mathrm{Tor}_n^R(N,M).$$
Thus we simply say that $R$ has the weak global dimension. Note that this does not hold for Ext functors, so (generally) the definition of global dimension is not left/right symmetric.

The following proposition is a simple consequence of the fact that every projective module is flat:

\textbf{Proposition.} For any ring $R$ we have
$$\mathrm{w.gl.dim}R\leqslant\mathrm{min}\ \{\ \mathrm{l.gl.dim}R,\ \mathrm{r.gl.dim}R\ \},$$
where $\mathrm{l.gl.dim}$ and $\mathrm{r.gl.dim}$ denote the left global dimension and right global dimension respectively.
%%%%%
%%%%%
\end{document}

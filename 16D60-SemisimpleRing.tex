\documentclass[12pt]{article}
\usepackage{pmmeta}
\pmcanonicalname{SemisimpleRing}
\pmcreated{2013-03-22 14:19:05}
\pmmodified{2013-03-22 14:19:05}
\pmowner{CWoo}{3771}
\pmmodifier{CWoo}{3771}
\pmtitle{semisimple ring}
\pmrecord{11}{35784}
\pmprivacy{1}
\pmauthor{CWoo}{3771}
\pmtype{Definition}
\pmcomment{trigger rebuild}
\pmclassification{msc}{16D60}
\pmrelated{SemiprimitiveRing}
\pmdefines{semisimple}

\endmetadata

% this is the default PlanetMath preamble.  as your knowledge
% of TeX increases, you will probably want to edit this, but
% it should be fine as is for beginners.

% almost certainly you want these
\usepackage{amssymb}
\usepackage{amsmath}
\usepackage{amsfonts}

% used for TeXing text within eps files
%\usepackage{psfrag}
% need this for including graphics (\includegraphics)
%\usepackage{graphicx}
% for neatly defining theorems and propositions
%\usepackage{amsthm}
% making logically defined graphics
%%%\usepackage{xypic}

% there are many more packages, add them here as you need them

% define commands here
\begin{document}
A ring $R$ is (left) \emph{semisimple} if it \PMlinkescapetext{satisfies} one of the following \PMlinkescapetext{equivalent} statements:
\begin{enumerate}
\item
All left $R$-modules are semisimple.
\item
All \PMlinkname{finitely-generated}{FinitelyGeneratedRModule} left $R$-modules are semisimple.
\item
All cyclic left $R$-modules are semisimple.
\item
The left regular $R$-module $_RR$ is semisimple.
\item
All short exact sequences of left $R$-modules \PMlinkname{split}{SplitShortExactSequence}.
\end{enumerate}

The last \PMlinkescapetext{equivalent} condition offers another homological characterization of a \emph{semisimple} ring:

\begin{itemize}
\item
A ring $R$ is (left) semisimple iff all of its left modules are \PMlinkname{projective}{ProjectiveModule}.
\end{itemize}

A more ring-theorectic characterization of a (left) semisimple ring is:

\begin{itemize}
\item
A ring is left semisimple iff it is semiprimitive and left artinian.
\end{itemize}

In some literature, a (left) semisimple ring is defined to be a ring that is semiprimitive without necessarily being (left) artinian.  Such a ring (semiprimitive) is called Jacobson semisimple, or J-semisimple, to remind us of the fact that its Jacobson radical is (0).

Relating to von Neumann regular rings, one has:

\begin{itemize}
\item
A ring is left semisimple iff it is von Neumann regular and left noetherian.
\end{itemize}

The famous Wedderburn-Artin Theorem \PMlinkescapetext{states} that a (left) semisimple ring is isomorphic to a finite direct product of matrix rings over division rings.

The theorem implies that a left semisimplicity is synonymous with right semisimplicity, so that it is safe to drop the word left or right when referring to semisimple rings.
%%%%%
%%%%%
\end{document}

\documentclass[12pt]{article}
\usepackage{pmmeta}
\pmcanonicalname{DualModule}
\pmcreated{2013-03-22 16:00:26}
\pmmodified{2013-03-22 16:00:26}
\pmowner{Mathprof}{13753}
\pmmodifier{Mathprof}{13753}
\pmtitle{dual module}
\pmrecord{10}{38037}
\pmprivacy{1}
\pmauthor{Mathprof}{13753}
\pmtype{Definition}
\pmcomment{trigger rebuild}
\pmclassification{msc}{16-00}
\pmrelated{Unimodular}
\pmdefines{linear functional}

% this is the default PlanetMath preamble.  as your knowledge
% of TeX increases, you will probably want to edit this, but
% it should be fine as is for beginners.

% almost certainly you want these
\usepackage{amssymb}
\usepackage{amsmath}
\usepackage{amsfonts}

% used for TeXing text within eps files
%\usepackage{psfrag}
% need this for including graphics (\includegraphics)
%\usepackage{graphicx}
% for neatly defining theorems and propositions
%\usepackage{amsthm}
% making logically defined graphics
%%%\usepackage{xypic}

% there are many more packages, add them here as you need them

% define commands here

\begin{document}
Let $R$ be a ring and $M$ be a left \PMlinkid{$R$-module}{365}. The {\it dual module} of $M$ is
the right \PMlinkid{$R$-module}{365} consisting of all module homomorphisms  from $M$ into $R$.



It is denoted by $M^\ast$. The elements of $M^\ast$ are called {\it linear functionals}.

The action of $R$ on $M^\ast$ is given by $(fr)(m) = (f(m))r$ for
$f \in M^\ast$, $m \in M$, and $r \in R$.

If $R$ is commutative, then every \PMlinkescapetext{$R$-module} $M$ is an \PMlinkid{$(R,R)$-bimodule}{987} with $rm = mr$ for all $r \in R$ and $m \in M$. Hence, it makes sense to ask whether $M$ and $M^\ast$ are isomorphic. Suppose that 
$b: M \times M \to R$ is a bilinear form. Then it is easy to check that for a fixed
$m \in M$, the function $b(m, -): M \to R$ is a module homomorphism,
so is an element of $M^\ast$. Then we have a  module homomorphism from $M$
to $M^\ast$ given by $m \mapsto b(m,-)$. 
%%%%%
%%%%%
\end{document}

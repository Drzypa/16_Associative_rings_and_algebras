\documentclass[12pt]{article}
\usepackage{pmmeta}
\pmcanonicalname{CornerOfARing}
\pmcreated{2013-03-22 15:43:56}
\pmmodified{2013-03-22 15:43:56}
\pmowner{CWoo}{3771}
\pmmodifier{CWoo}{3771}
\pmtitle{corner of a ring}
\pmrecord{9}{37682}
\pmprivacy{1}
\pmauthor{CWoo}{3771}
\pmtype{Definition}
\pmcomment{trigger rebuild}
\pmclassification{msc}{16S99}
\pmrelated{UnityOfSubring}

\endmetadata

\usepackage{amssymb,amscd}
\usepackage{amsmath}
\usepackage{amsfonts}

% used for TeXing text within eps files
%\usepackage{psfrag}
% need this for including graphics (\includegraphics)
%\usepackage{graphicx}
% for neatly defining theorems and propositions
%\usepackage{amsthm}
% making logically defined graphics
%%%\usepackage{xypic}

% define commands here
\begin{document}
Does there exist a subset $S$ of a ring $R$ which is a ring with a multiplicative identity, but not a subring of $R$?

Let $R$ be a ring without the assumption that $R$ has a multiplicative identity.  Further, assume that $e$ is an idempotent of $R$.  Then the subset of the form $eRe$ is called a \emph{corner} of the ring $R$.

It's not hard to see that $eRe$ is a ring with $e$ as its multiplicative identity:
\begin{enumerate}
\item $eae+ebe=e(a+b)e\in eRe$,
\item $0=e0e\in eRe$,
\item $e(-a)e$ is the additive inverse of $eae$ in $eRe$,
\item $(eae)(ebe)=e(aeb)e\in eRe$, and 
\item $e=ee=eee\in eRe$, with $e(eae)=eae=(eae)e$, for any $eae\in eRe$.
\end{enumerate}

If $R$ has no multiplicative identity, then any corner of $R$ is a proper subset of $R$ which is a ring and not a subring of $R$.  If $R$ has 1 as its multiplicative identity and if $e\neq 1$ is an idempotent, then the $eRe$ is not a subring of $R$ as they don't share the same multiplicative identity.  In this case, the corner $eRe$ is said to be \emph{proper}.  If we set $f=1-e$, then $fRf$ is also a proper corner of $R$.

\textbf{Remark.}  If $R$ has 1 with $e\neq 1$ an idempotent.  Then corners $S=eRe$ and $T=fRf$, where $f=1-e$, are direct summands (as modules over $\mathbb{Z}$) of $R$ via a Peirce decomposition.

\begin{thebibliography}{9}
\bibitem{edwards} I. Kaplansky, \emph{Rings of Operators}, W. A. Benjamin, Inc., New York, 1968.
\end{thebibliography}
%%%%%
%%%%%
\end{document}

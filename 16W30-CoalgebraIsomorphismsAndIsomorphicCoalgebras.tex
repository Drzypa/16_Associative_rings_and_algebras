\documentclass[12pt]{article}
\usepackage{pmmeta}
\pmcanonicalname{CoalgebraIsomorphismsAndIsomorphicCoalgebras}
\pmcreated{2013-03-22 18:49:28}
\pmmodified{2013-03-22 18:49:28}
\pmowner{joking}{16130}
\pmmodifier{joking}{16130}
\pmtitle{coalgebra isomorphisms and isomorphic coalgebras}
\pmrecord{4}{41627}
\pmprivacy{1}
\pmauthor{joking}{16130}
\pmtype{Definition}
\pmcomment{trigger rebuild}
\pmclassification{msc}{16W30}

% this is the default PlanetMath preamble.  as your knowledge
% of TeX increases, you will probably want to edit this, but
% it should be fine as is for beginners.

% almost certainly you want these
\usepackage{amssymb}
\usepackage{amsmath}
\usepackage{amsfonts}

% used for TeXing text within eps files
%\usepackage{psfrag}
% need this for including graphics (\includegraphics)
%\usepackage{graphicx}
% for neatly defining theorems and propositions
%\usepackage{amsthm}
% making logically defined graphics
%%%\usepackage{xypic}

% there are many more packages, add them here as you need them

% define commands here

\begin{document}
Let $(C,\Delta,\varepsilon)$ and $(D,\Delta',\varepsilon')$ be coalgebras.

\textbf{Definition.} We will say that coalgebra homomorphism $f:C\to D$ is a \textit{coalgebra isomorphism}, if there exists a coalgebra homomorphism $g:D\to C$ such that $f\circ g=\mathrm{id}_{D}$ and $g\circ f=\mathrm{id}_{C}$.

\textbf{Remark.} Of course every coalgebra isomorphism is a linear isomorphism, thus it is ,,one-to-one'' and ,,onto''. One can show that the converse also holds, i.e. if $f:C\to D$ is a coalgebra homomorphism such that $f$ is ,,one-to-one'' and ,,onto'', then $f$ is a coalgebra isomorphism.

\textbf{Definition.} We will say that coalgebras $(C,\Delta,\varepsilon)$ and $(D,\Delta',\varepsilon')$ are \textit{isomorphic} if there exists coalgebra isomorphism $f:C\to D$. In this case we often write $(C,\Delta,\varepsilon)\simeq (D,\Delta',\varepsilon')$ or simply $C\simeq D$ if structure maps are known from the context.

\textbf{Remarks.} Of course the relation ,,$\simeq$'' is an equivalence relation. Furthermore, (from the coalgebraic point of view) isomorphic coalgebras are the same, i.e. they share all coalgebraic properties.
%%%%%
%%%%%
\end{document}

\documentclass[12pt]{article}
\usepackage{pmmeta}
\pmcanonicalname{CriterionForCyclicRingsToBePrincipalIdealRings}
\pmcreated{2013-03-22 15:57:03}
\pmmodified{2013-03-22 15:57:03}
\pmowner{Wkbj79}{1863}
\pmmodifier{Wkbj79}{1863}
\pmtitle{criterion for cyclic rings to be principal ideal rings}
\pmrecord{15}{37962}
\pmprivacy{1}
\pmauthor{Wkbj79}{1863}
\pmtype{Theorem}
\pmcomment{trigger rebuild}
\pmclassification{msc}{16U99}
\pmclassification{msc}{13A99}
\pmclassification{msc}{13F10}
\pmrelated{CyclicRing3}
\pmrelated{PrincipalIdealRing}
\pmrelated{MultiplicativeIdentityOfACyclicRingMustBeAGenerator}
\pmrelated{CyclicRingsOfBehaviorOne}

\usepackage{amssymb}
\usepackage{amsmath}
\usepackage{amsfonts}

\usepackage{psfrag}
\usepackage{graphicx}
\usepackage{amsthm}
%%\usepackage{xypic}

\newtheorem*{thm*}{Theorem}
\begin{document}
\begin{thm*}
A cyclic ring is a principal ideal ring if and only if it has a multiplicative identity.
\end{thm*}

\begin{proof}
Let $R$ be a cyclic ring.  If $R$ has a multiplicative identity $u$, then $u$ \PMlinkname{generates}{Generator} the additive group of $R$.  Let $I$ be an ideal of $R$.  Since $\{0_R\}$ is principal, it may be assumed that $I$ contains a nonzero element.  Let $n$ be the smallest natural number such that $nu \in I$.  The inclusion $\langle nu \rangle \subseteq I$ is trivial.  Let $t \in I$.  Since $t \in R$, there exists $a \in \mathbb{Z}$ with $t=au$.  By the division algorithm, there exists $q,r \in \mathbb{Z}$ with $0 \le r < n$ such that $a=qn+r$.  Thus, $t=au=(qn+r)u=(qn)u+ru=q(nu)+ru$.  Since $ru=t-q(nu) \in I$, by choice of $n$, it must be the case that $r=0$.  Thus, $t=q(nu)$.  Hence, $\langle nu \rangle = I$, and $R$ is a principal ideal ring.

Conversely, if $R$ is a principal ideal ring, then $R$ is a principal ideal.  Let $k$ be the behavior of $R$ and $r$ be a \PMlinkname{generator}{Generator} of the additive group of $R$ such that $r^2=kr$.  Since $R$ is principal, there exists $s \in R$ such that $\langle s \rangle = R$.  Let $a \in \mathbb{Z}$ such that $s=ar$.  Since $r \in R = \langle s \rangle$, there exists $t \in R$ with $st=r$.  Let $b \in \mathbb{Z}$ such that $t=br$.  Then $r=st=(ar)(br)=(ab)r^2=(ab)(kr)=(abk)r$.  If $R$ is infinite, then $abk=1$, in which case $k=1$ since $k$ is nonnegative.  If $R$ is finite, then $abk \equiv 1 \operatorname{mod} |R|$, in which case $k=1$ since $k$ is a positive divisor of $|R|$.  In either case, $R$ has behavior one, and it follows that $R$ has a multiplicative identity.
\end{proof}
%%%%%
%%%%%
\end{document}

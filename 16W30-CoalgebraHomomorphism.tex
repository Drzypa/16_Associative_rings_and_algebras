\documentclass[12pt]{article}
\usepackage{pmmeta}
\pmcanonicalname{CoalgebraHomomorphism}
\pmcreated{2013-03-22 18:49:25}
\pmmodified{2013-03-22 18:49:25}
\pmowner{joking}{16130}
\pmmodifier{joking}{16130}
\pmtitle{coalgebra homomorphism}
\pmrecord{4}{41626}
\pmprivacy{1}
\pmauthor{joking}{16130}
\pmtype{Definition}
\pmcomment{trigger rebuild}
\pmclassification{msc}{16W30}

% this is the default PlanetMath preamble.  as your knowledge
% of TeX increases, you will probably want to edit this, but
% it should be fine as is for beginners.

% almost certainly you want these
\usepackage{amssymb}
\usepackage{amsmath}
\usepackage{amsfonts}

% used for TeXing text within eps files
%\usepackage{psfrag}
% need this for including graphics (\includegraphics)
%\usepackage{graphicx}
% for neatly defining theorems and propositions
%\usepackage{amsthm}
% making logically defined graphics
%%%\usepackage{xypic}

% there are many more packages, add them here as you need them

% define commands here

\begin{document}
Let $(C,\Delta,\varepsilon)$ and $(D,\Delta',\varepsilon')$ be coalgebras.

\textbf{Definition.} Linear map $f:C\to D$ is called \textit{coalgebra homomorphism} if $\Delta'\circ f=(f\otimes f)\circ\Delta$ and $\varepsilon'\circ f=\varepsilon$.

\textbf{Examples.}
$1)$ Of course, if $D$ is a subcoalgebra of $C$, then the inclusion $i:D\to C$ is a coalgebra homomorphism. In particular, the identity is a coalgebra homomorphism.

$2)$ If $(C,\Delta,\varepsilon)$ is a coalgebra and $I\subseteq C$ is a coideal, then we have canonical coalgebra structur on $C/I$ (please, see \PMlinkname{this entry}{SubcoalgebrasAndCoideals} for more details). Then the projection $\pi:C\to C/I$ is a coalgebra homomorphism. Furthermore, one can show that the canonical coalgebra structure on $C/I$ is a unique coalgebra structure such that $\pi$ is a coalgebra homomorphism.
%%%%%
%%%%%
\end{document}

\documentclass[12pt]{article}
\usepackage{pmmeta}
\pmcanonicalname{ACharacterizationOfTheRadicalOfAnIdeal}
\pmcreated{2013-03-22 18:04:50}
\pmmodified{2013-03-22 18:04:50}
\pmowner{CWoo}{3771}
\pmmodifier{CWoo}{3771}
\pmtitle{a characterization of the radical of an ideal}
\pmrecord{7}{40617}
\pmprivacy{1}
\pmauthor{CWoo}{3771}
\pmtype{Derivation}
\pmcomment{trigger rebuild}
\pmclassification{msc}{16N40}
\pmclassification{msc}{13-00}
\pmclassification{msc}{14A05}

\usepackage{amssymb,amscd}
\usepackage{amsmath}
\usepackage{amsfonts}
\usepackage{mathrsfs}

% used for TeXing text within eps files
%\usepackage{psfrag}
% need this for including graphics (\includegraphics)
%\usepackage{graphicx}
% for neatly defining theorems and propositions
\usepackage{amsthm}
% making logically defined graphics
%%\usepackage{xypic}
\usepackage{pst-plot}

% define commands here
\newcommand*{\abs}[1]{\left\lvert #1\right\rvert}
\newtheorem{prop}{Proposition}
\newtheorem{thm}{Theorem}
\newtheorem{ex}{Example}
\newcommand{\real}{\mathbb{R}}
\newcommand{\pdiff}[2]{\frac{\partial #1}{\partial #2}}
\newcommand{\mpdiff}[3]{\frac{\partial^#1 #2}{\partial #3^#1}}
\begin{document}
\begin{prop} Let $I$ be an ideal in a ring $R$, and $\sqrt{I}$ be its radical.  Then $\sqrt{I}$ is the intersection of all prime ideals containing $I$. \end{prop}

\begin{proof}  Suppose $x\in \sqrt{I}$, and $P$ is a prime ideal containing $I$.  Then $R-P$ is an \PMlinkname{$m$-system}{MSystem}.  If $x\in R-P$, then $(R-P)\cap I\ne \varnothing$, contradicting the assumption that $I\subseteq P$.  Therefore $x\notin R-P$.  In other words, $x\in P$, and we have one of the inclusions.

Conversely, suppose $x\notin \sqrt{I}$.  Then there is an $m$-system $S$ containing $x$ such that $S\cap I=\varnothing$.  Enlarge $I$ to a prime ideal $P$ disjoint from $S$, so that $x\notin P$ (we can do this; for a proof, see the second remark in \PMlinkname{this entry}{MSystem}).  By contrapositivity, we have the other inclusion.
\end{proof}

\textbf{Remark}.  This shows that every prime ideal is a radical ideal: for $\sqrt{P}$ is the intersection of all prime ideals containing $P$, and if $P$ is itself prime, then $P=\sqrt{P}$.
%%%%%
%%%%%
\end{document}

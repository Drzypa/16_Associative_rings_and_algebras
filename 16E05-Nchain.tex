\documentclass[12pt]{article}
\usepackage{pmmeta}
\pmcanonicalname{Nchain}
\pmcreated{2013-03-22 13:46:20}
\pmmodified{2013-03-22 13:46:20}
\pmowner{mps}{409}
\pmmodifier{mps}{409}
\pmtitle{$n$-chain}
\pmrecord{11}{34478}
\pmprivacy{1}
\pmauthor{mps}{409}
\pmtype{Definition}
\pmcomment{trigger rebuild}
\pmclassification{msc}{16E05}
\pmsynonym{chain}{Nchain}
%\pmkeywords{chain complex}
%\pmkeywords{cell}
\pmrelated{LongExactSequenceOfHomologyGroups}
\pmdefines{closed n-chain}
\pmdefines{exact n-chain}
\pmdefines{boundary map}

\endmetadata

% this is the default PlanetMath preamble.  as your knowledge
% of TeX increases, you will probably want to edit this, but
% it should be fine as is for beginners.

% almost certainly you want these
\usepackage{amssymb}
\usepackage{amsmath}
\usepackage{amsfonts}
\usepackage{amsthm}

% used for TeXing text within eps files
%\usepackage{psfrag}
% need this for including graphics (\includegraphics)
%\usepackage{graphicx}
% for neatly defining theorems and propositions
%\usepackage{amsthm}
% making logically defined graphics
%%\usepackage{xypic}

% there are many more packages, add them here as you need them

% define commands here

\newcommand{\mc}{\mathcal}
\newcommand{\mb}{\mathbb}
\newcommand{\mf}{\mathfrak}
\newcommand{\ol}{\overline}
\newcommand{\ra}{\rightarrow}
\newcommand{\la}{\leftarrow}
\newcommand{\La}{\Leftarrow}
\newcommand{\Ra}{\Rightarrow}
\newcommand{\nor}{\vartriangleleft}
\newcommand{\Gal}{\text{Gal}}
\newcommand{\GL}{\text{GL}}
\newcommand{\Z}{\mb{Z}}
\newcommand{\R}{\mb{R}}
\newcommand{\Q}{\mb{Q}}
\newcommand{\C}{\mb{C}}
\newcommand{\<}{\langle}
\renewcommand{\>}{\rangle}
\DeclareMathOperator{\im}{im}
\begin{document}
Let $X$ be a topological space and let $K$ be a simplicial approximation to $X$.  An \emph{$n$-chain} on $X$ is a finite formal sum of oriented $n$-simplices in $K$.  The group of such chains is denoted by $C_n(X)$ and is called the $n$th \emph{chain group} of $X$.  In other words, $C_n(X)$ is the free abelian group generated by the oriented $n$-simplices in $K$.

We have defined chain groups for simplicial homology.  Their definition is similar in singular homology and the homology of CW complexes.  For example, if $Y$ is a CW complex, then its $n$th chain group is the free abelian group on the cells of $Y^n$, the $n$-skeleton of $Y$.

The formal \emph{boundary} of an oriented $n$-simplex $\sigma=(v_0,\dots,v_n)$ is given by the alternating sum of the oriented $n$-simplices forming the topological boundary of $\sigma$, that is,
\[
\partial_n(\sigma) = \sum_{j=0}^n (-1)^j (v_0,\dots, v_{j-1},v_{j+1},\dots, v_n).
\]
The boundary of a $0$-simplex is $0$.

Since $n$-simplices form a basis for the chain group $C_n(X)$, this extends to give a group homomorphism $\partial_n\colon C_n(X)\to C_{n-1}(X)$, called the \emph{boundary map}.  An $n$-chain is \emph{closed} if its boundary is 0 and \emph{exact} if it is the boundary of some $(n+1)$-chain.  Closed $n$-chains are also called \emph{cycles}.  Every exact $n$-chain is also closed.  This implies that the sequence
\[\xymatrix{
\cdots \ar[r] & C_{n+1}(X)\ar[r]^{\partial_{n+1}} & C_n(X)\ar[r]^{\partial_n} & C_{n-1}\ar[r] & \cdots
}\]
is a complex of free abelian groups.  This complex is usually called the \emph{chain complex} of $X$ corresponding to the simplicial complex $K$.  Note that while the chain groups $C_n(X)$ depend on the choice of simplicial approximation $K$, the resulting homology groups
\[
H_n(X) = \frac{\ker\partial_n}{\im\,\partial_{n+1}}
\]
do not.

\PMlinkescapeword{alternating}
\PMlinkescapeword{boundary}
\PMlinkescapeword{boundaries}
\PMlinkescapeword{closed}
\PMlinkescapeword{cycle}
\PMlinkescapeword{cycles}
\PMlinkescapeword{words}
%%%%%
%%%%%
\end{document}

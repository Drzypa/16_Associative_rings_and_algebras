\documentclass[12pt]{article}
\usepackage{pmmeta}
\pmcanonicalname{SimpleRing}
\pmcreated{2013-03-22 11:51:07}
\pmmodified{2013-03-22 11:51:07}
\pmowner{antizeus}{11}
\pmmodifier{antizeus}{11}
\pmtitle{simple ring}
\pmrecord{9}{30414}
\pmprivacy{1}
\pmauthor{antizeus}{11}
\pmtype{Definition}
\pmcomment{trigger rebuild}
\pmclassification{msc}{16D60}

\usepackage{amssymb}
\usepackage{amsmath}
\usepackage{amsfonts}
\usepackage{graphicx}
%%%%\usepackage{xypic}
\begin{document}
A nonzero ring $R$ is said to be a {\it simple ring} if it has no (two-sided) ideal other then the zero ideal and $R$ itself.
\par
This is equivalent to saying that the zero ideal is a maximal ideal.
\par
If $R$ is a commutative ring with unit, then this is equivalent to being a field.
%%%%%
%%%%%
%%%%%
%%%%%
\end{document}

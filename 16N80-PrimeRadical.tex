\documentclass[12pt]{article}
\usepackage{pmmeta}
\pmcanonicalname{PrimeRadical}
\pmcreated{2013-03-22 12:01:29}
\pmmodified{2013-03-22 12:01:29}
\pmowner{CWoo}{3771}
\pmmodifier{CWoo}{3771}
\pmtitle{prime radical}
\pmrecord{7}{30995}
\pmprivacy{1}
\pmauthor{CWoo}{3771}
\pmtype{Definition}
\pmcomment{trigger rebuild}
\pmclassification{msc}{16N80}
\pmsynonym{lower nilradical}{PrimeRadical}
\pmrelated{RadicalOfAnIdeal}
\pmrelated{Nilradical}

\endmetadata

\usepackage{amssymb}
\usepackage{amsmath}
\usepackage{amsfonts}
\usepackage{graphicx}
%%%\usepackage{xypic}
\begin{document}
The {\it prime radical} or {\it lower nilradical} of a ring $R$ is the intersection of all the prime ideals of $R$.

Note that the prime radical is the smallest semiprime ideal of $R$, and that $R$ is a semiprime ring if and only if its prime radical is the zero ideal.
%%%%%
%%%%%
%%%%%
\end{document}

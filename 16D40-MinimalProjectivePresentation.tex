\documentclass[12pt]{article}
\usepackage{pmmeta}
\pmcanonicalname{MinimalProjectivePresentation}
\pmcreated{2013-03-22 19:18:00}
\pmmodified{2013-03-22 19:18:00}
\pmowner{joking}{16130}
\pmmodifier{joking}{16130}
\pmtitle{minimal projective presentation}
\pmrecord{4}{42235}
\pmprivacy{1}
\pmauthor{joking}{16130}
\pmtype{Definition}
\pmcomment{trigger rebuild}
\pmclassification{msc}{16D40}

\endmetadata

% this is the default PlanetMath preamble.  as your knowledge
% of TeX increases, you will probably want to edit this, but
% it should be fine as is for beginners.

% almost certainly you want these
\usepackage{amssymb}
\usepackage{amsmath}
\usepackage{amsfonts}

% used for TeXing text within eps files
%\usepackage{psfrag}
% need this for including graphics (\includegraphics)
%\usepackage{graphicx}
% for neatly defining theorems and propositions
%\usepackage{amsthm}
% making logically defined graphics
%%\usepackage{xypic}

% there are many more packages, add them here as you need them

% define commands here

\begin{document}
Let $R$ be a ring and $M$ a (right) module over $R$. A short exact sequence of modules
$$\xymatrix{
P_1\ar[r]^{p_1} & P_0\ar[r]^{p_0} & M\ar[r] & 0
}$$
is called a \textbf{minimal projective presentation} of $M$ if both $p_0:P_0\to M$ and $p_1:P_1\to\mathrm{ker}p_0$ are projective covers.

Minimal projective presenetations are unique in the following sense: if
$$\xymatrix{
P_1\ar[r]^{p_1} & P_0\ar[r]^{p_0} & M\ar[r] & 0 \\
P'_1\ar[r]^{p'_1} & P'_0\ar[r]^{p'_0} & M\ar[r] & 0 \\
}$$
are both minimal projective presentations of $M$, then this diagram can be completed to the following commutative one:
$$\xymatrix{
P_1\ar[r]^{p_1}\ar[d]^{a} & P_0\ar[r]^{p_0}\ar[d]^{b} & M\ar[r]\ar[d]^{=} & 0 \\
P'_1\ar[r]^{p'_1} & P'_0\ar[r]^{p'_0} & M\ar[r] & 0 \\
}$$
were both $a,b$ are isomorphisms.

It can be shown, that if $R$ is a finite-dimensional algebra over a field $k$, then every finitely generated $R$-module $M$ admits minimal projective presentation (indeed, $R$ is \PMlinkname{semiperfect}{PerfectAndSemiperfectRings} in this case). 
%%%%%
%%%%%
\end{document}

\documentclass[12pt]{article}
\usepackage{pmmeta}
\pmcanonicalname{Span}
\pmcreated{2013-03-22 11:58:18}
\pmmodified{2013-03-22 11:58:18}
\pmowner{mathwizard}{128}
\pmmodifier{mathwizard}{128}
\pmtitle{span}
\pmrecord{22}{30806}
\pmprivacy{1}
\pmauthor{mathwizard}{128}
\pmtype{Definition}
\pmcomment{trigger rebuild}
\pmclassification{msc}{16D10}
\pmclassification{msc}{15A03}
\pmsynonym{linear span}{Span}
%\pmkeywords{linear combination}
%\pmkeywords{span}
\pmrelated{LinearCombination}
\pmrelated{Basis}
\pmrelated{ProofOfGramSchmidtOrthogonalizationProcedure}
\pmrelated{FinitelyGeneratedRModule}
\pmdefines{spanning set}

\endmetadata

\usepackage{amssymb}
\usepackage{amsmath}
\usepackage{amsfonts}
\def\vec#1{\overrightarrow{\mathbf{#1}}}
\def\R{\mathbb{R}}
\def\Sp{\operatorname{Sp}}
\newcommand{\ZZ}{\mathbb{Z}}

\begin{document}
The \emph{span} of a set of vectors $\vec{v}_1,\dots,\vec{v_n}$ of a vector space $V$ over a field $K$ is the set of linear combinations $a_1\vec{v}_1+\dots+a_n\vec{v}_n$ with $a_i\in K$.  It is denoted $\Sp(\vec{v}_1,\dots,\vec{v}_n)$.   More generally, the span of a set $S$ (not necessarily finite) of vectors is the collection of all (finite) linear combinations of elements of $S$.  The span of the empty set is defined to be the singleton consisting of the zero vector $\vec{0}$.

For example, the standard basis vectors $\hat{\imath}$ and $\hat{\jmath}$ span $\R^2$ because every vector of $\R^2$ can be represented as a linear combination of $\hat{\imath}$ and $\hat{\jmath}$. 

$\Sp(\vec{v}_1,\dots,\vec{v}_n)$ is a subspace of $V$ and is the smallest subspace containing $\vec{v}_1,\dots,\vec{v}_n$.

\emph{Span} is both a noun and a verb; a \emph{set of vectors} can span a vector space, and a vector can be \emph{in the span} of a set of vectors.

\textbf{Checking span:} To see whether a vector is \emph{in the span} of other vectors, one can set up an augmented matrix, since if $\vec{u}$ is in the span of $\vec{v}_1,\vec{v}_2$, then $\vec{u} = x_1\vec{v}_1 + x_2\vec{v}_2$. This is a system of linear equations. Thus, if it has a solution, $\vec{u}$ is in the span of $\vec{v}_1,\vec{v}_2$. Note that the solution does not have to be unique for $\vec{u}$ to be in the span.

To see whether a set of vectors \emph{spans} a vector space, you need to check that there are at least as many linearly independent vectors as the dimension of the space. For example, it can be shown that in $\R^n$, $n+1$ vectors are never linearly independent, and $n-1$ vectors never span.

\textbf{Remark}.  We can define the concept of span also for a module $M$ over a ring $R$. 
Given a subset $X\subset M$ we define the module generated by $X$ as the set of all finite linear combinations of elements of $X$.
Be aware that in general there does not exist a linearly independent subset which generates the whole module, i.e. there does not have to exist a basis.
Also, even if $M$ is generated by $n$ elements, it is in general not true that any other set of $n$ linearly independent elements of $M$ spans $M$. 
For example $\ZZ$ is generated by $1$ as a $\ZZ$-module but not by $2$.

%%%%%
%%%%%
%%%%%
\end{document}

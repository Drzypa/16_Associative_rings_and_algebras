\documentclass[12pt]{article}
\usepackage{pmmeta}
\pmcanonicalname{NilAndNilpotentIdeals}
\pmcreated{2013-03-22 13:13:25}
\pmmodified{2013-03-22 13:13:25}
\pmowner{mclase}{549}
\pmmodifier{mclase}{549}
\pmtitle{nil and nilpotent ideals}
\pmrecord{6}{33690}
\pmprivacy{1}
\pmauthor{mclase}{549}
\pmtype{Definition}
\pmcomment{trigger rebuild}
\pmclassification{msc}{16N40}
\pmrelated{KoetheConjecture}
\pmdefines{nil}
\pmdefines{nil ring}
\pmdefines{nil ideal}
\pmdefines{nil right ideal}
\pmdefines{nil left ideal}
\pmdefines{nil subring}
\pmdefines{nilpotent}
\pmdefines{nilpotent element}
\pmdefines{nilpotent ring}
\pmdefines{nilpotent ideal}
\pmdefines{nilpotent right ideal}
\pmdefines{nilpotent left ideal}
\pmdefines{nilpotent subring}
\pmdefines{locally nilpotent}
\pmdefines{locally nilpotent ring}
\pmdefines{locally nilpo}

% this is the default PlanetMath preamble.  as your knowledge
% of TeX increases, you will probably want to edit this, but
% it should be fine as is for beginners.

% almost certainly you want these
\usepackage{amssymb}
\usepackage{amsmath}
\usepackage{amsfonts}

% used for TeXing text within eps files
%\usepackage{psfrag}
% need this for including graphics (\includegraphics)
%\usepackage{graphicx}
% for neatly defining theorems and propositions
%\usepackage{amsthm}
% making logically defined graphics
%%%\usepackage{xypic}

% there are many more packages, add them here as you need them

% define commands here
\begin{document}
An element $x$ of a ring is \emph{nilpotent} if $x^n = 0$ for some positive integer $n$.

A ring $R$ is \emph{nil} if every element in $R$ is nilpotent.  Similarly, a one- or two-sided ideal is called \emph{nil} if each of its elements is nilpotent.

A ring $R$ [resp. a one- or two sided ideal $A$] is \emph{nilpotent} if $R^n = 0$ [resp. $A^n = 0$] for some positive integer $n$.

A ring or an ideal is \emph{locally nilpotent} if every finitely generated subring is nilpotent.

The following implications hold for rings (or ideals):

$$\text{nilpotent} \quad\Rightarrow \text{locally nilpotent} \quad\Rightarrow \text{nil}$$
%%%%%
%%%%%
\end{document}

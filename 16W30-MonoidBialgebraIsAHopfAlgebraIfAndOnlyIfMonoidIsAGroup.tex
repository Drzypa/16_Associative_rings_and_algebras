\documentclass[12pt]{article}
\usepackage{pmmeta}
\pmcanonicalname{MonoidBialgebraIsAHopfAlgebraIfAndOnlyIfMonoidIsAGroup}
\pmcreated{2013-03-22 18:58:51}
\pmmodified{2013-03-22 18:58:51}
\pmowner{joking}{16130}
\pmmodifier{joking}{16130}
\pmtitle{monoid bialgebra is a Hopf algebra if and only if monoid is a group}
\pmrecord{4}{41845}
\pmprivacy{1}
\pmauthor{joking}{16130}
\pmtype{Theorem}
\pmcomment{trigger rebuild}
\pmclassification{msc}{16W30}

% this is the default PlanetMath preamble.  as your knowledge
% of TeX increases, you will probably want to edit this, but
% it should be fine as is for beginners.

% almost certainly you want these
\usepackage{amssymb}
\usepackage{amsmath}
\usepackage{amsfonts}

% used for TeXing text within eps files
%\usepackage{psfrag}
% need this for including graphics (\includegraphics)
%\usepackage{graphicx}
% for neatly defining theorems and propositions
%\usepackage{amsthm}
% making logically defined graphics
%%%\usepackage{xypic}

% there are many more packages, add them here as you need them

% define commands here

\begin{document}
Assume that $H$ is a Hopf algebra with comultiplication $\Delta$, counit $\varepsilon$ and antipode $S$. It is well known, that if $c\in H$ and $\Delta(c)=\sum\limits_{i=1}^{n}a_i\otimes b_i$, then $\sum\limits_{i=1}^{n}S(a_i)b_i=\varepsilon(c)1=\sum\limits_{i=1}^{n}a_iS(b_i)$ (actualy, this condition defines the antipode), where on the left and right side we have multiplication in $H$.

Now let $G$ be a monoid and $k$ a field. It is well known that $kG$ is a bialgebra (please, see parent object for details), but one may ask, when $kG$ is a Hopf algebra? We will try to answer this question.

\textbf{Proposition.} A monoid bialgebra $kG$ is a Hopf algebra if and only if $G$ is a group.

\textit{Proof.} ,,$\Leftarrow$'' If $G$ is a group, then define $S:kG\to kG$ by $S(g)=g^{-1}$. It is easy to check, that $S$ is the antipode, thus $kG$ is a Hopf algebra.

,,$\Rightarrow$'' Assume that $kG$ is a Hopf algebra, i.e. we have the antipode $S:kG\to kG$. Then, for any $g\in G$ we have $S(g)g=gS(g)=1$ (because $\Delta(g)=g\otimes g$ and $\varepsilon(g)=1$). Here $1$ is the identity in both $G$ and $kG$. Of course $S(g)\in kG$, so
$$S(g)=\sum_{h\in G}\lambda_h h.$$
Thus we have 
$$1=\big(\sum_{h\in G}\lambda_h h\big)g=\sum_{h\in G}\lambda_h hg.$$
Of course $G$ is a basis, so this decomposition is unique. Therefore, there exists $g'\in G$ such that $\lambda_{g'}=1$ and $\lambda_{h'}=0$ for $h'\neq g'$. We obtain, that $1=g'g$, thus $g$ is left invertible. Since $g$ was arbitrary it implies that $g$ is invertible. Thus, we've shown that $G$ is a group. $\square$
%%%%%
%%%%%
\end{document}

\documentclass[12pt]{article}
\usepackage{pmmeta}
\pmcanonicalname{Module1}
\pmcreated{2013-03-22 12:01:51}
\pmmodified{2013-03-22 12:01:51}
\pmowner{yark}{2760}
\pmmodifier{yark}{2760}
\pmtitle{module}
\pmrecord{12}{31022}
\pmprivacy{1}
\pmauthor{yark}{2760}
\pmtype{Definition}
\pmcomment{trigger rebuild}
\pmclassification{msc}{16D10}
\pmsynonym{module action}{Module1}
\pmsynonym{left module action}{Module1}
\pmsynonym{right module action}{Module1}
\pmsynonym{unital module}{Module1}
\pmrelated{Module}

\usepackage{amssymb}
\usepackage{amsmath}
\usepackage{amsfonts}
\begin{document}
\PMlinkescapeword{terms}

(This is a definition of modules in terms of ring homomorphisms. You may prefer to read \PMlinkname{the other definition}{Module} instead.)

Let $R$ be a ring, 
and let $M$ be an abelian group.

We say that $M$ is a {\it left $R$-module}
if there exists a ring homomorphism $\phi\colon R \to {\rm End}_{\Bbb{Z}}(M)$ 
from $R$ to the ring of abelian group endomorphisms on $M$
(in which multiplication of endomorphisms is composition,
using left function notation).
We typically denote this function using a multiplication notation:
$$[\phi(r)](m) = r \cdot m = rm.$$

This ring homomorphism defines 
what is called a {\PMlinkescapetext {\it left module action}}
of $R$ upon $M$.

If $R$ is a unital ring 
(i.e. a ring with identity),
then we typically demand
that the ring homomorphism
map the unit $1 \in R$
to the identity endomorphism on $M$,
so that $1 \cdot m = m$ for all $m \in M$.
In this case we may say 
that the module is \emph{unital}.

Typically the abelian group structure on $M$
is expressed in additive terms,
i.e. with operator $+$, 
identity element $0_M$ (or just $0$),
and inverses written in 
the form $-m$ for $m \in M$.

%The fact that $\phi$ is a ring homomorphism implies the following:

%(a) $(r_1 + r_2) \cdot m = r_1 \cdot m + r_2 \cdot m$
%for all $r_1$ and $r_2$ in $R$, and $m$ in $M$;
%
%(b) $(r_1 \cdot r_2) \cdot m = r_1 \cdot (r_2 \cdot m)$
%for all $r_1$ and $r_2$ in $R$, and $m$ in $M$;
%
%(c) $0 \cdot m = 0_M$ for all $m$ in $M$;
%
%(d) $(-r) \cdot m = -(r \cdot m)$
%for all $r$ in $R$, and $m$ in $M$;
%
%(e) $1 \cdot m = m$ for all $m$ in $M$, if $R$ and $M$ are unital.
%
%The fact that elements of $R$ map to endomorphisms on $M$
%implies the following:
%
%(a) $r \cdot (m_1 + m_2) = r \cdot m_1 + r \cdot m_2$
%for all $r$ in $R$, and $m_1$ and $m_2$ in $M$;
%
%(b) $r \cdot 0_M = 0_M$ for all $r$ in $R$;
%
%(c) $r \cdot (-m) = -(r \cdot m)$ for all $r$ in $R$, and $m$ in $M$;
%
%One often defines a module action in terms of lists of properties as above
%instead of appealing to the existence of the ring homomorphism $\phi$.

Right module actions are defined similarly,
only with the elements of $R$ being written
on the right sides of elements of $M$.
In this case we either need to use 
an anti-homomorphism $R \to \operatorname{End}_{\mathbb{Z}}(M)$,
or switch to right notation for writing functions.
%%%%%
%%%%%
%%%%%
\end{document}

\documentclass[12pt]{article}
\usepackage{pmmeta}
\pmcanonicalname{ARingModuloItsJacobsonRadicalIsSemiprimitive}
\pmcreated{2013-03-22 12:49:34}
\pmmodified{2013-03-22 12:49:34}
\pmowner{yark}{2760}
\pmmodifier{yark}{2760}
\pmtitle{a ring modulo its Jacobson radical is semiprimitive}
\pmrecord{13}{33149}
\pmprivacy{1}
\pmauthor{yark}{2760}
\pmtype{Theorem}
\pmcomment{trigger rebuild}
\pmclassification{msc}{16N20}

\endmetadata

\usepackage{amssymb}
\usepackage{amsmath}
\usepackage{amsfonts}

\begin{document}
\PMlinkescapeword{characterizations}

Let $R$ be a ring.  Then $J(R/J(R))=(0)$.

{\it Proof:}\\
We will only prove this in the case where $R$ is a unital ring
(although it is true without this assumption).

Let $[u] \in J(R/J(R))$.
By one of the characterizations of the Jacobson radical,
$1-[r][u]$ is left invertible for all $r \in R$,
so there exists $v \in R$ such that $[v](1-[r][u])=1$.

Then $v(1-ru)=1-a$ for some $a \in J(R)$.
There is a $w \in R$ such that $w(1-a)=1$,
and we have $wv(1-ru)=1$.

Since this holds for all $r \in R$,
it follows that $u \in J(R)$, and therefore $[u]=0$.
%%%%%
%%%%%
\end{document}

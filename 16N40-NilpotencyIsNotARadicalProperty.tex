\documentclass[12pt]{article}
\usepackage{pmmeta}
\pmcanonicalname{NilpotencyIsNotARadicalProperty}
\pmcreated{2013-03-22 14:13:02}
\pmmodified{2013-03-22 14:13:02}
\pmowner{mclase}{549}
\pmmodifier{mclase}{549}
\pmtitle{nilpotency is not a radical property}
\pmrecord{4}{35652}
\pmprivacy{1}
\pmauthor{mclase}{549}
\pmtype{Proof}
\pmcomment{trigger rebuild}
\pmclassification{msc}{16N40}

\endmetadata

% this is the default PlanetMath preamble.  as your knowledge
% of TeX increases, you will probably want to edit this, but
% it should be fine as is for beginners.

% almost certainly you want these
\usepackage{amssymb}
\usepackage{amsmath}
\usepackage{amsfonts}

% used for TeXing text within eps files
%\usepackage{psfrag}
% need this for including graphics (\includegraphics)
%\usepackage{graphicx}
% for neatly defining theorems and propositions
%\usepackage{amsthm}
% making logically defined graphics
%%%\usepackage{xypic}

% there are many more packages, add them here as you need them

% define commands here
\begin{document}
Nilpotency is not a radical property, because a ring does not, in general, contain a largest nilpotent ideal.

Let $k$ be a field, and let $S = k[X_1, X_2, \dotsc]$ be the ring of polynomials over $k$ in infinitely many variables $X_1, X_2, \dots$.
Let $I$ be the ideal of $S$ generated by $\{X_n^{n+1} \mid n \in \mathbb(N)\}$.
Let $R = S/I$.  Note that $R$ is commutative.

For each $n$, let $A_n = \sum_{k=1}^n RX_n$.
Let $A = \bigcup A_n = \sum_{k = 1}^\infty RX_n$.

Then each $A_n$ is nilpotent, since it is the sum of finitely many nilpotent ideals (see proof \PMlinkid{here}{5650}).  But $A$ is nil, but not nilpotent.  Indeed, for any $n$, there is an element $x \in A$ such that $x^n \neq 0$, namely $x = X_n$, and so we cannot have $A^n = 0$.

So $R$ cannot have a largest nilpotent ideal, for this ideal would have to contain all the ideals $A_n$ and therefore $A$.
%%%%%
%%%%%
\end{document}

\documentclass[12pt]{article}
\usepackage{pmmeta}
\pmcanonicalname{HomogeneousPolynomial1}
\pmcreated{2013-03-22 14:53:42}
\pmmodified{2013-03-22 14:53:42}
\pmowner{CWoo}{3771}
\pmmodifier{CWoo}{3771}
\pmtitle{homogeneous polynomial}
\pmrecord{17}{36577}
\pmprivacy{1}
\pmauthor{CWoo}{3771}
\pmtype{Definition}
\pmcomment{trigger rebuild}
\pmclassification{msc}{16R99}
\pmclassification{msc}{13B25}
\pmclassification{msc}{16S36}
\pmclassification{msc}{11E76}
\pmsynonym{polynomial form}{HomogeneousPolynomial1}
\pmrelated{HomogeneousIdeal}
\pmrelated{HomogeneousFunction}
\pmrelated{HomogeneousEquation}
\pmdefines{homogeneous component}
\pmdefines{cubic form}
\pmdefines{linear form}

% this is the default PlanetMath preamble.  as your knowledge
% of TeX increases, you will probably want to edit this, but
% it should be fine as is for beginners.

% almost certainly you want these
\usepackage{amssymb,amscd}
\usepackage{amsmath}
\usepackage{amsfonts}

% used for TeXing text within eps files
%\usepackage{psfrag}
% need this for including graphics (\includegraphics)
%\usepackage{graphicx}
% for neatly defining theorems and propositions
%\usepackage{amsthm}
% making logically defined graphics
%%%\usepackage{xypic}

% there are many more packages, add them here as you need them

% define commands here
\begin{document}
\PMlinkescapeword{homogeneous}
\PMlinkescapeword{degree}

Let $R$ be an associative ring.  A (multivariate) polynomial $f$ over $R$ is said to be \emph{homogeneous of degree} $r$ if it is expressible as an $R$-\PMlinkname{linear combination}{LinearCombination} of monomials of degree $r$: 
$$f(x_1,\ldots,x_n)=\sum_{i=1}^{m}a_i{x_1}^{r_{i1}}\cdots{x_n}^{r_{in}},$$ 
where $r=r_{i1}+\cdots+r_{in}$ for all $i\in\lbrace 1,\ldots,m\rbrace$ and $a_i\in R$.  

A general homogeneous polynomial is also known sometimes as a \emph{polynomial form}.  A homogeneous polynomial of degree 1 is called a \emph{linear form}; a homogeneous polynomial of degree 2 is called a \emph{quadratic form}; and a homogeneous polynomial of degree 3 is called a \emph{cubic form}.

\textbf{Remarks}. 
\begin{enumerate} 
\item If $f$ is a homogeneous polynomial over a ring $R$ with $\operatorname{deg}(f)=r$, then $f(tx_1,\ldots,tx_n)=t^rf(x_1,\ldots,x_n)$.  In fact, a homogeneous function that is also a polynomial is a homogeneous polynomial.
\item Every polynomial $f$ over $R$ can be expressed uniquely as a finite sum of homogeneous polynomials.  
The homogeneous polynomials that make up the polynomial $f$ are called the \emph{homogeneous components} of $f$. 
\item If $f$ and $g$ are homogeneous polynomials of degree $r$ and $s$ over a domain $R$, then $fg$ is homogeneous of degree $r+s$.  From this, one sees that given a domain $R$, the ring $R[\boldsymbol{X}]$ is a graded ring, where $\boldsymbol{X}$ is a finite set of indeterminates.  The condition that $R$ does not have any zero divisors is essential here.  As a counterexample, in $\mathbb{Z}_6[x,y]$, if $f(x,y)=2x+4y$ and $g(x,y)=3x+3y$, then $f(x,y)g(x,y)=0$.
\end{enumerate} 

\textbf{Examples} 
\begin{itemize} 
\item $f(x,y) = x^2+xy+yx+y^2$ is a homogeneous polynomial of degree 2.  Notice the middle two monomials could be combined into the monomial 2xy if the variables are allowed to commute with one another.
\item $f(x) = x^3+1$ is not a homogeneous polynomial. 
\item $f(x,y,z) = x^3+xyz+zyz+3xy^2+x^2-xy+y^2+zy+z^2+xz+y+2x+6$ is a polynomial that is the sum of four homogeneous 
polynomials: $x^3+xyz+zyz+3xy^2$ (with degree 3), $x^2-xy+y^2+zy+z^2+xz$ (degree = 2), $y+2x$ (degree = 1) and 
$6$ (deg = 0). 
\item Every symmetric polynomial can be written as a sum of symmetric homogeneous polynomials.
\end{itemize}
%%%%%
%%%%%
\end{document}

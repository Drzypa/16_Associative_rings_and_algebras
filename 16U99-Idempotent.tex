\documentclass[12pt]{article}
\usepackage{pmmeta}
\pmcanonicalname{Idempotent}
\pmcreated{2013-03-22 13:07:27}
\pmmodified{2013-03-22 13:07:27}
\pmowner{mclase}{549}
\pmmodifier{mclase}{549}
\pmtitle{idempotent}
\pmrecord{11}{33558}
\pmprivacy{1}
\pmauthor{mclase}{549}
\pmtype{Definition}
\pmcomment{trigger rebuild}
\pmclassification{msc}{16U99}
\pmclassification{msc}{20M99}
\pmsynonym{idempotent element}{Idempotent}
\pmrelated{Semilattice}
\pmrelated{Idempotency}
\pmdefines{orthogonal idempotents}
\pmdefines{complete set of orthogonal idempotents}

% this is the default PlanetMath preamble.  as your knowledge
% of TeX increases, you will probably want to edit this, but
% it should be fine as is for beginners.

% almost certainly you want these
\usepackage{amssymb}
\usepackage{amsmath}
\usepackage{amsfonts}

% used for TeXing text within eps files
%\usepackage{psfrag}
% need this for including graphics (\includegraphics)
%\usepackage{graphicx}
% for neatly defining theorems and propositions
%\usepackage{amsthm}
% making logically defined graphics
%%%\usepackage{xypic}

% there are many more packages, add them here as you need them

% define commands here

\newcommand{\isom}{\cong}
\begin{document}
\PMlinkescapeword{complete}

An element $x$ of a ring is called an \emph{idempotent element}, or simply an \emph{idempotent} if $x^2 = x$.

The set of idempotents of a ring can be partially ordered by putting $e \le f$ iff $e = ef = fe$.

The element $0$ is a minimum element in this partial order.  If the ring has an identity element, $1$, then $1$ is a maximum element in this partial order.

Since the above definitions refer only to the multiplicative structure of the ring, they also hold for semigroups (with the proviso, of course, that a semigroup may have neither a zero element nor an identity element).  In the special case of a semilattice, this partial order is the same as the one described in the entry for semilattice.

If a ring has an identity, then $1 - e$ is always an idempotent whenever $e$ is an idempotent, and $e(1-e) = (1-e)e = 0$.

In a ring with an identity, two idempotents $e$ and $f$ are called a \emph{pair of orthogonal idempotents} if $e + f = 1$, and $ef = fe = 0$.  Obviously, this is just a fancy way of saying that $f = 1 - e$.

More generally, a set $\{e_1, e_2, \dots, e_n\}$ of idempotents is called a \emph{complete set of orthogonal idempotents} if $e_i e_j = e_j e_i = 0$ whenever $i \neq j$ and if $1 = e_1 + e_2 + \dots + e_n$.

If $\{e_1, e_2, \dots, e_n\}$ is a complete set of orthogonal idempotents,
and in addition each $e_i$ is in the centre of $R$, then each $Re_i$ is a subring, and $$R \isom Re_1 \times Re_2 \times \dots \times Re_n.$$

Conversely, whenever $R_1 \times R_2 \times \dots \times R_n$ is a direct
product of rings with identities, write $e_i$ for the element of the product
corresponding to the identity element of $R_i$.  Then $\{e_1, e_2, \dots, e_n\}$ is a complete set of central orthogonal idempotents of the product ring.

When a complete set of orthogonal idempotents is not central, there is a more complicated \PMlinkescapetext{decomposition}: see the entry on the Peirce decomposition for the details.
%%%%%
%%%%%
\end{document}

\documentclass[12pt]{article}
\usepackage{pmmeta}
\pmcanonicalname{GoldieRing}
\pmcreated{2013-03-22 14:04:10}
\pmmodified{2013-03-22 14:04:10}
\pmowner{mclase}{549}
\pmmodifier{mclase}{549}
\pmtitle{Goldie ring}
\pmrecord{6}{35427}
\pmprivacy{1}
\pmauthor{mclase}{549}
\pmtype{Definition}
\pmcomment{trigger rebuild}
\pmclassification{msc}{16P60}
\pmsynonym{Goldie}{GoldieRing}
\pmrelated{UniformDimension}
\pmdefines{left Goldie}
\pmdefines{right Goldie}
\pmdefines{left Goldie ring}
\pmdefines{right Goldie ring}

\endmetadata

% this is the default PlanetMath preamble.  as your knowledge
% of TeX increases, you will probably want to edit this, but
% it should be fine as is for beginners.

% almost certainly you want these
\usepackage{amssymb}
\usepackage{amsmath}
\usepackage{amsfonts}

% used for TeXing text within eps files
%\usepackage{psfrag}
% need this for including graphics (\includegraphics)
%\usepackage{graphicx}
% for neatly defining theorems and propositions
%\usepackage{amsthm}
% making logically defined graphics
%%%\usepackage{xypic}

% there are many more packages, add them here as you need them

% define commands here

\DeclareMathOperator{\rann}{r.ann}
\begin{document}
Let $R$ be a ring.  If the set of annihilators
$\{ \rann(x) \mid x \in R\}$
satisifies the ascending chain condition, then
$R$ is said to satisfy the \emph{ascending chain condition on right annihilators}.

A ring $R$ is called a \emph{right Goldie ring} if it satisfies the ascending chain condition on right annihilators and $R_R$ is a module of finite rank.

\emph{Left Goldie ring} is defined similarly.  If the context makes it clear on which side the ring operates, then such a ring is simply called a \emph{Goldie ring}.

A right Noetherian ring is right Goldie.
%%%%%
%%%%%
\end{document}

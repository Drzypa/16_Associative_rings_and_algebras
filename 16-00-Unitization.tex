\documentclass[12pt]{article}
\usepackage{pmmeta}
\pmcanonicalname{Unitization}
\pmcreated{2013-03-22 14:47:36}
\pmmodified{2013-03-22 14:47:36}
\pmowner{rspuzio}{6075}
\pmmodifier{rspuzio}{6075}
\pmtitle{unitization}
\pmrecord{9}{36445}
\pmprivacy{1}
\pmauthor{rspuzio}{6075}
\pmtype{Definition}
\pmcomment{trigger rebuild}
\pmclassification{msc}{16-00}
\pmclassification{msc}{13-00}
\pmclassification{msc}{20-00}
\pmsynonym{minimal unitization}{Unitization}

\endmetadata

% this is the default PlanetMath preamble.  as your knowledge
% of TeX increases, you will probably want to edit this, but
% it should be fine as is for beginners.

% almost certainly you want these
\usepackage{amssymb}
\usepackage{amsmath}
\usepackage{amsfonts}

% used for TeXing text within eps files
%\usepackage{psfrag}
% need this for including graphics (\includegraphics)
%\usepackage{graphicx}
% for neatly defining theorems and propositions
%\usepackage{amsthm}
% making logically defined graphics
%%%\usepackage{xypic}

% there are many more packages, add them here as you need them

% define commands here

\def\A{\bf A}
\def\R{\bf R}
\begin{document}
The operation of unitization allows one to add a unity element to an algebra.  Because of this construction, one can regard any algebra as a subalgebra of an algebra with unity.  If the algebra already has a unity, the operation creates a larger algebra in which the old unity is no longer the unity.

Let $\A$ be an algebra over a ring $\R$ with unity $1$.  Then, as a module, the unitization of $\A$ is the direct sum of $\R$ and $\A$:
\[
\A^+ = \R \oplus \A
\]
The product operation is defined as follows:
\[
(x, a) \cdot (y, b) = (xy, ab + xb + ya)
\]
The unity of $\A^+$ is $(1,0)$.

It is also possible to unitize any ring using this construction if one regards the ring as an algebra over the ring of \PMlinkname{integers}{Integer}.  (See the entry every ring is an integer algebra for details.)  It is worth noting, 
however, that the result of unitizing a ring this way will always be a ring whose unity has zero characteristic.  If one
has a ring of finite characteristic $k$, one can instead regard it as an algebra over $\mathbb{Z}_k$ and unitize
accordingly to obtain a ring of characteristic $k$.

The construction described above is often  called ``minimal unitization''. It is in fact minimal, in the sense that 
every other unitization contains this unitization as a subalgebra.
%%%%%
%%%%%
\end{document}

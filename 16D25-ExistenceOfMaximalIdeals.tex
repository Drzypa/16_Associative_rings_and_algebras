\documentclass[12pt]{article}
\usepackage{pmmeta}
\pmcanonicalname{ExistenceOfMaximalIdeals}
\pmcreated{2013-03-22 13:56:57}
\pmmodified{2013-03-22 13:56:57}
\pmowner{yark}{2760}
\pmmodifier{yark}{2760}
\pmtitle{existence of maximal ideals}
\pmrecord{22}{34713}
\pmprivacy{1}
\pmauthor{yark}{2760}
\pmtype{Theorem}
\pmcomment{trigger rebuild}
\pmclassification{msc}{16D25}
\pmclassification{msc}{13A15}
\pmsynonym{existence of maximal ideals}{ExistenceOfMaximalIdeals}
%\pmkeywords{maximal ideal}
%\pmkeywords{commutative ring}
%\pmkeywords{identity}
%\pmkeywords{axiom of choice}
\pmrelated{ZornsLemma}
\pmrelated{AxiomOfChoice}
\pmrelated{MaximalIdeal}
\pmrelated{ExistenceOfMaximalSubgroups}
\pmrelated{DefinitionOfPrimeIdealByKrull}

\usepackage{amssymb}
\usepackage{amsmath}
\usepackage{amsthm}
\usepackage{amsfonts}

\newtheorem*{thm*}{Theorem}
\newtheorem*{defn*}{Definition}
\newtheorem*{prop*}{Proposition}
\newtheorem*{lemma*}{Lemma}
\newtheorem*{cor*}{Corollary}

\newcommand{\Nats}{\mathbb{N}}
\newcommand{\Ints}{\mathbb{Z}}
\newcommand{\Reals}{\mathbb{R}}
\newcommand{\Complex}{\mathbb{C}}
\newcommand{\Rats}{\mathbb{Q}}
\begin{document}
\PMlinkescapeword{chain}
\PMlinkescapeword{equivalent}
\PMlinkescapeword{identity}
\PMlinkescapeword{necessary}
\PMlinkescapeword{order}
\PMlinkescapeword{property}
\PMlinkescapeword{simple}
\PMlinkescapeword{unity}

\begin{thm*}
Let $\mathcal{R}$ be a unital ring.
Every proper ideal of $\mathcal{R}$ lies in a maximal ideal of $\mathcal{R}$.
\end{thm*}

Applying this theorem to the zero ideal gives the following corollary:

\begin{cor*}
Every unital ring $\mathcal{R}\neq 0$ has a maximal ideal.
\end{cor*}

{\it Proof of theorem}.
This proof is a straightforward application of Zorn's Lemma.
Readers are encouraged to attempt the proof themselves
before reading the details below.

Let $\mathcal{I}$ be a proper ideal of $\mathcal{R}$,
and let $\Sigma$ be the partially ordered set
$$\Sigma=\{ \mathcal{A} \mid \mathcal{A} \hbox{ is an ideal of }
\mathcal{R},\hbox{ and }
\mathcal{I}\subseteq\mathcal{A}\neq \mathcal{R}\}$$
ordered by inclusion.

Note that $\mathcal{I}\in\Sigma$, so $\Sigma$ is non-empty.

In order to apply Zorn's Lemma
we need to prove that every non-empty \PMlinkname{chain}{TotalOrder}
in $\Sigma$ has an upper bound in $\Sigma$.
Let $\{\mathcal{A}_{\alpha}\}$ be a non-empty chain of ideals in $\Sigma$,
so for all indices $\alpha,\beta$ we have
$$\mathcal{A}_{\alpha}\subseteq \mathcal{A}_{\beta}\quad \text{ or }\quad \mathcal{A}_{\beta}\subseteq
\mathcal{A}_{\alpha}.$$
We claim that $\mathcal{B}$ defined by
$$\mathcal{B}=\bigcup_{\alpha}\mathcal{A}_{\alpha}$$
is a suitable upper bound.
\begin{itemize}
\item $\mathcal{B}$ is an ideal. Indeed, let $a,b\in\mathcal{B}$,
so there exist $\alpha,\beta$ such that $a\in
\mathcal{A}_{\alpha}$, $b\in\mathcal{A}_{\beta}$. Since these two
ideals are in a totally ordered chain we have
$$\mathcal{A}_{\alpha}\subseteq \mathcal{A}_{\beta}\quad \text{ or }\quad \mathcal{A}_{\beta}\subseteq
\mathcal{A}_{\alpha}$$ Without loss of generality, we assume
$\mathcal{A}_{\alpha}\subseteq \mathcal{A}_{\beta}$. Then both
$a,b\in \mathcal{A}_{\beta}$, and $\mathcal{A}_{\beta}$ is an
ideal of the ring $\mathcal{R}$. Thus $a+b\in
\mathcal{A}_{\beta}\subseteq \mathcal{B}$.

Similarly, let $r\in \mathcal{R}$ and $b\in \mathcal{B}$. As
above, there exists $\beta$ such that $b\in \mathcal{A}_{\beta}$.
Since $\mathcal{A}_{\beta}$ is an ideal we have
$$r\cdot b \in \mathcal{A}_{\beta}\subseteq\mathcal{B}$$
and
$$b\cdot r \in \mathcal{A}_{\beta}\subseteq\mathcal{B}.$$
Therefore, $\mathcal{B}$ is an ideal.

\item $\mathcal{B}\neq \mathcal{R}$, otherwise $1$ would belong to
$\mathcal{B}$, so there would be an $\alpha$ such that $1\in
\mathcal{A}_{\alpha}$ so $\mathcal{A}_{\alpha}=\mathcal{R}$. But
this is impossible because we assumed $\mathcal{A}_{\alpha}\in
\Sigma$ for all indices $\alpha$.

\item $\mathcal{I}\subseteq\mathcal{B}$.
Indeed, the chain is non-empty, 
so there is some $\mathcal{A}_\alpha$ in the chain,
and we have $\mathcal{I}\subseteq\mathcal{A}_\alpha\subseteq\mathcal{B}$.
\end{itemize}
Therefore $\mathcal{B}\in\Sigma$. Hence every chain in $\Sigma$
has an upper bound in $\Sigma$ and we can apply Zorn's Lemma to
deduce the existence of $\mathcal{M}$, a maximal element (with
respect to inclusion) in $\Sigma$.
By definition of the set $\Sigma$,
this $\mathcal{M}$ must be a maximal ideal of $\mathcal{R}$
containing $\mathcal{I}$. QED

Note that the above proof never makes use of 
the associativity of ring multiplication,
and the result therefore holds also in non-associative rings.
The result cannot, however, be generalized to rings without unity.

Note also that the use of the Axiom of Choice (in the form of Zorn's Lemma)
is necessary,
as there are models of ZF in which the above theorem and corollary fail.
%%%%%
%%%%%
\end{document}

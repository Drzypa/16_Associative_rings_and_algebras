\documentclass[12pt]{article}
\usepackage{pmmeta}
\pmcanonicalname{FiniteDimensionalModulesOverAlgebra}
\pmcreated{2013-03-22 19:16:35}
\pmmodified{2013-03-22 19:16:35}
\pmowner{joking}{16130}
\pmmodifier{joking}{16130}
\pmtitle{finite dimensional modules over algebra}
\pmrecord{4}{42208}
\pmprivacy{1}
\pmauthor{joking}{16130}
\pmtype{Definition}
\pmcomment{trigger rebuild}
\pmclassification{msc}{16S99}
\pmclassification{msc}{20C99}
\pmclassification{msc}{13B99}

% this is the default PlanetMath preamble.  as your knowledge
% of TeX increases, you will probably want to edit this, but
% it should be fine as is for beginners.

% almost certainly you want these
\usepackage{amssymb}
\usepackage{amsmath}
\usepackage{amsfonts}

% used for TeXing text within eps files
%\usepackage{psfrag}
% need this for including graphics (\includegraphics)
%\usepackage{graphicx}
% for neatly defining theorems and propositions
%\usepackage{amsthm}
% making logically defined graphics
%%%\usepackage{xypic}

% there are many more packages, add them here as you need them

% define commands here

\begin{document}
Assume that $k$ is a field, $A$ is a $k$-algebra and $M$ is a $A$-module over $k$. In particular $M$ is a $A$-module and a vector space over $k$, thus we may speak about $M$ being finitely generated as $A$-module and finite dimensional as a vector space. These two concepts are related as follows:

\textbf{Proposition.} Assume that $A$ and $M$ are both unital and additionaly $A$ is finite dimensional. Then $M$ is finite dimensional vector space if and only if $M$ is finitely generated $A$-module.

\textit{Proof.} ,,$\Rightarrow$'' Of course if $M$ is finite dimensional, then there exists basis 
$$\{x_1,\ldots,x_n\}\subset M.$$
Thus every element of $M$ can be (uniquely) expressed in the form
$$\sum_{i=1}^{n}\lambda_i\cdot x_i$$
which is equal to 
$$\sum_{i=1}^{n}(\lambda_i\cdot 1)\cdot x_i$$
since $M$ and $A$ are unital. This completes this implication, because $\lambda_i\cdot 1\in A$ for all $i$.

,,$\Leftarrow$'' Assume that $M$ is finitely generated $A$-module. In particular there is a subset 
$$\{x_1,\ldots,x_n\}\subset M$$
such that every element of $M$ is of the form
$$\sum_{i=1}^{n}a_i\cdot x_i$$
with all $a_i\in A$. Let $m\in M$ be with the decomposition as above. Now $A$ is finite dimensional, so there is a subset 
$$\{y_1,\ldots,y_t\}\subset A$$
which is a $k$-basis of $A$. In particular for each $i$ we have
$$a_i=\sum_{j=1}^{t}\lambda_{ij}\cdot y_j$$
with $\lambda_{ij}\in k$. Thus we obtain
$$m=\sum_{i=1}^{n}a_i\cdot x_i = \sum_{i=1}^{n}\big(\sum_{j=1}^{t}\lambda_{ij}\cdot y_j\big)\cdot x_i=$$
$$=\sum_{i=1}^{n}\sum_{j=1}^{t}\lambda_{ij}\cdot(y_j\cdot x_i)$$
which shows, that all $y_j\cdot x_i\in M$ together make a set of generators of $M$ over $k$ (note that $y_j$ and $x_i$ are independent on $m$). Since it is finite, then $M$ is finite dimensional and the proof is complete. $\square$
%%%%%
%%%%%
\end{document}

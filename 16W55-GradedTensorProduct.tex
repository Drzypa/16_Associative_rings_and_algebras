\documentclass[12pt]{article}
\usepackage{pmmeta}
\pmcanonicalname{GradedTensorProduct}
\pmcreated{2013-03-22 12:45:44}
\pmmodified{2013-03-22 12:45:44}
\pmowner{mathcam}{2727}
\pmmodifier{mathcam}{2727}
\pmtitle{graded tensor product}
\pmrecord{8}{33070}
\pmprivacy{1}
\pmauthor{mathcam}{2727}
\pmtype{Definition}
\pmcomment{trigger rebuild}
\pmclassification{msc}{16W55}
\pmrelated{TensorProduct}
\pmdefines{super tensor product}

% this is the default PlanetMath preamble.  as your knowledge
% of TeX increases, you will probably want to edit this, but
% it should be fine as is for beginners.

% almost certainly you want these
\usepackage{amssymb}
\usepackage{amsmath}
\usepackage{amsfonts}

% used for TeXing text within eps files
%\usepackage{psfrag}
% need this for including graphics (\includegraphics)
%\usepackage{graphicx}
% for neatly defining theorems and propositions
%\usepackage{amsthm}
% making logically defined graphics
%%%\usepackage{xypic} 

% there are many more packages, add them here as you need them

% define commands here
\begin{document}
If $A$ and $B$ are $\mathbb{Z}$-graded algebras, we define the \emph{graded tensor product} (or \emph{super tensor product}) $A \otimes_{su} B$ to be the ordinary tensor product as graded modules, but with multiplication - called the {\em super product} - defined by
$$(a \otimes b)(a' \otimes b') = (-1)^{(\text{deg \ } b)(\text{deg \ } a')}aa' \otimes bb'$$
where $a,a',b,b'$ are homogeneous. The super tensor product of $A$ and $B$ is itself a graded algebra, as we grade the super tensor product of $A$ and $B$ as follows:

$$ (A \otimes_{su} B)^n = \coprod_{p,q \text{ : } p + q = n} A^p \otimes B^q $$
%%%%%
%%%%%
\end{document}

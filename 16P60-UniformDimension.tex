\documentclass[12pt]{article}
\usepackage{pmmeta}
\pmcanonicalname{UniformDimension}
\pmcreated{2013-03-22 14:02:59}
\pmmodified{2013-03-22 14:02:59}
\pmowner{mclase}{549}
\pmmodifier{mclase}{549}
\pmtitle{uniform dimension}
\pmrecord{7}{35401}
\pmprivacy{1}
\pmauthor{mclase}{549}
\pmtype{Definition}
\pmcomment{trigger rebuild}
\pmclassification{msc}{16P60}
\pmrelated{GoldieRing}

% this is the default PlanetMath preamble.  as your knowledge
% of TeX increases, you will probably want to edit this, but
% it should be fine as is for beginners.

% almost certainly you want these
\usepackage{amssymb}
\usepackage{amsmath}
\usepackage{amsfonts}

% used for TeXing text within eps files
%\usepackage{psfrag}
% need this for including graphics (\includegraphics)
%\usepackage{graphicx}
% for neatly defining theorems and propositions
%\usepackage{amsthm}
% making logically defined graphics
%%%\usepackage{xypic}

% there are many more packages, add them here as you need them

% define commands here

\newcommand{\udim}{\operatorname{u-dim}}
\begin{document}
Let $M$ be a module over a ring $R$, and suppose that $M$ contains no infinite direct sums of non-zero submodules. (This is the same as saying that $M$ is a module of finite rank.)

Then there exists an integer $n$ such that $M$ contains an essential submodule $N$ where $$N = U_1 \oplus U_2 \oplus \dots \oplus U_n$$
is a direct sum of $n$ uniform submodules.

This number $n$ does not depend on the choice of $N$ or the decomposition into uniform submodules.

We call $n$ the \emph{uniform dimension} of $M$.  Sometimes this is written $\udim M = n$.

If $R$ is a field $K$, and $M$ is a finite-dimensional vector space over $K$, then $\udim M = \dim_KM$.

$\udim M = 0$ if and only if $M = 0$.
%%%%%
%%%%%
\end{document}

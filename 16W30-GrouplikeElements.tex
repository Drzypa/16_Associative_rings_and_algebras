\documentclass[12pt]{article}
\usepackage{pmmeta}
\pmcanonicalname{GrouplikeElements}
\pmcreated{2013-03-22 18:58:37}
\pmmodified{2013-03-22 18:58:37}
\pmowner{joking}{16130}
\pmmodifier{joking}{16130}
\pmtitle{grouplike elements}
\pmrecord{5}{41840}
\pmprivacy{1}
\pmauthor{joking}{16130}
\pmtype{Definition}
\pmcomment{trigger rebuild}
\pmclassification{msc}{16W30}

% this is the default PlanetMath preamble.  as your knowledge
% of TeX increases, you will probably want to edit this, but
% it should be fine as is for beginners.

% almost certainly you want these
\usepackage{amssymb}
\usepackage{amsmath}
\usepackage{amsfonts}

% used for TeXing text within eps files
%\usepackage{psfrag}
% need this for including graphics (\includegraphics)
%\usepackage{graphicx}
% for neatly defining theorems and propositions
%\usepackage{amsthm}
% making logically defined graphics
%%%\usepackage{xypic}

% there are many more packages, add them here as you need them

% define commands here

\begin{document}
Let $(C,\Delta,\varepsilon)$ be a coalgebra over a field $k$.

\textbf{Definition.} The element $g\in C$ is called \textit{grouplike} iff $g\neq 0$ and $\Delta(g)=g\otimes g$. The set of all grouplike elements in a coalgebra $C$ is denoted by $G(C)$.

\textbf{Properties}. $0)$ The set $G(C)$ can be empty, but (for example) if $C$ can be turned into a bialgebra, then $G(C)\neq\emptyset$. In particular Hopf algebras always have grouplike elements.

$1)$ If $g\in G(C)$, then it follows from the counit property that $\varepsilon(g)=1$.

$2)$ It can be shown that the set $G(C)$ is linearly independent.
%%%%%
%%%%%
\end{document}

\documentclass[12pt]{article}
\usepackage{pmmeta}
\pmcanonicalname{IdealOfAnAlgebra}
\pmcreated{2013-03-22 18:09:00}
\pmmodified{2013-03-22 18:09:00}
\pmowner{asteroid}{17536}
\pmmodifier{asteroid}{17536}
\pmtitle{ideal of an algebra}
\pmrecord{6}{40706}
\pmprivacy{1}
\pmauthor{asteroid}{17536}
\pmtype{Definition}
\pmcomment{trigger rebuild}
\pmclassification{msc}{16D25}
\pmsynonym{left ideal of an algebra}{IdealOfAnAlgebra}
\pmsynonym{right ideal of an algebra}{IdealOfAnAlgebra}
\pmsynonym{two-sided ideal of an algebra}{IdealOfAnAlgebra}

% this is the default PlanetMath preamble.  as your knowledge
% of TeX increases, you will probably want to edit this, but
% it should be fine as is for beginners.

% almost certainly you want these
\usepackage{amssymb}
\usepackage{amsmath}
\usepackage{amsfonts}

% used for TeXing text within eps files
%\usepackage{psfrag}
% need this for including graphics (\includegraphics)
%\usepackage{graphicx}
% for neatly defining theorems and propositions
%\usepackage{amsthm}
% making logically defined graphics
%%%\usepackage{xypic}

% there are many more packages, add them here as you need them

% define commands here

\begin{document}
\PMlinkescapephrase{left ideal}
\PMlinkescapephrase{right ideal}
\PMlinkescapephrase{two-sided ideal}


Let $A$ be an algebra over a ring $R$.

{\bf Definition -} A \emph{left ideal} of $A$ is a subalgebra $I \subseteq A$ such that $ax \in I$ whenever $a \in A$ and $ x \in I$.

Equivalently, a left ideal of $A$ is a subset $I \subset A$ such that
\begin{enumerate}
\item $x - y \in I$, for all $x, y \in I$.
\item $rx \in I$, for all $r \in R$ and $x \in I$.
\item $ax \in I$, for all $a \in A$ and $x \in I$
\end{enumerate}

Similarly one can define a \emph{right ideal} by replacing condition 3 by: $xa \in I$ whenever $a \in A$ and $x \in I$.

A \emph{two-sided ideal} of $A$ is a left ideal which is also a right ideal. Usually the word "\PMlinkescapetext{ideal}" by itself means two-sided ideal. Of course, all these notions coincide when $A$ is commutative.

\subsubsection{Remark}

Since an algebra is also a ring, one might think of borrowing the definition of ideal from ring \PMlinkescapetext{theory}. The problem is that condition 2 would not be in general satisfied (unless the algebra is unital).
%%%%%
%%%%%
\end{document}

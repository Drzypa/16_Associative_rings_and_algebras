\documentclass[12pt]{article}
\usepackage{pmmeta}
\pmcanonicalname{ProjectiveEquivalence}
\pmcreated{2013-03-22 14:50:13}
\pmmodified{2013-03-22 14:50:13}
\pmowner{CWoo}{3771}
\pmmodifier{CWoo}{3771}
\pmtitle{projective equivalence}
\pmrecord{5}{36505}
\pmprivacy{1}
\pmauthor{CWoo}{3771}
\pmtype{Definition}
\pmcomment{trigger rebuild}
\pmclassification{msc}{16E10}
\pmclassification{msc}{18G20}
\pmclassification{msc}{18G10}
\pmdefines{projectively equivalent}
\pmdefines{Schanuel's Lemma}

\endmetadata

% this is the default PlanetMath preamble.  as your knowledge
% of TeX increases, you will probably want to edit this, but
% it should be fine as is for beginners.

% almost certainly you want these
\usepackage{amssymb,amscd}
\usepackage{amsmath}
\usepackage{amsfonts}

% used for TeXing text within eps files
%\usepackage{psfrag}
% need this for including graphics (\includegraphics)
%\usepackage{graphicx}
% for neatly defining theorems and propositions
%\usepackage{amsthm}
% making logically defined graphics
%%\usepackage{xypic}

% there are many more packages, add them here as you need them

% define commands here
\begin{document}
Let $R$ be a ring with 1.  Two $R$-modules $A$ and $B$ are said to be \emph{projectively equivalent} $A\sim B$ if there exist 
two projective $R$-modules $P$ and $Q$ such that $$A\oplus P\cong B\oplus Q.$$
\par
\textbf{Remarks.} 
\begin{enumerate}
\item Projective equivalence is an equivalence relation.
\item Any projective module is projectively equivalent to the zero module.
\item \textbf{(Schanuel's Lemma)}.  Given two short exact sequences:
\begin{center}
$\xymatrix{0\ar[r]&B_1\ar[r]&P\ar[r]&A_1\ar[r]&0}$
\end{center}
\begin{center}
$\xymatrix{0\ar[r]&B_2\ar[r]&Q\ar[r]&A_2\ar[r]&0}$
\end{center}
with $A_1\sim A_2$, then $B_1\sim B_2$.
\item Schanuel's Lemma can be generalized.  Given two projective resolutions:
\begin{center}
$\xymatrix{\ldots\ar[r]^{p_3}&P_2\ar[r]^{p_2}&P_1\ar[r]^{p_1}&P_0\ar[r]^{p_0}&A_1\ar[r]&0}$
\end{center}
\begin{center}
$\xymatrix{\ldots\ar[r]^{q_3}&Q_2\ar[r]^{q_2}&Q_1\ar[r]^{q_1}&Q_0\ar[r]^{q_0}&A_2\ar[r]&0}$
\end{center}
with $A_1\sim A_2$, then $\operatorname{Ker}(p_n)\sim\operatorname{Ker}(q_n)$ for all $n\geq0$
\item The concept of projective equivalence between two modules can be generalized to any abelian categories having enough projectives.
\end{enumerate}
%%%%%
%%%%%
\end{document}

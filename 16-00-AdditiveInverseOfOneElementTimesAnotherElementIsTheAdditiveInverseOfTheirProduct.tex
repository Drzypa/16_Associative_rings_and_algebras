\documentclass[12pt]{article}
\usepackage{pmmeta}
\pmcanonicalname{AdditiveInverseOfOneElementTimesAnotherElementIsTheAdditiveInverseOfTheirProduct}
\pmcreated{2013-03-22 15:43:40}
\pmmodified{2013-03-22 15:43:40}
\pmowner{cvalente}{11260}
\pmmodifier{cvalente}{11260}
\pmtitle{additive inverse of one element times another element is the additive inverse of their product}
\pmrecord{8}{37677}
\pmprivacy{1}
\pmauthor{cvalente}{11260}
\pmtype{Theorem}
\pmcomment{trigger rebuild}
\pmclassification{msc}{16-00}
\pmclassification{msc}{20-00}
\pmclassification{msc}{13-00}

% this is the default PlanetMath preamble.  as your knowledge
% of TeX increases, you will probably want to edit this, but
% it should be fine as is for beginners.

% almost certainly you want these
\usepackage{amssymb}
\usepackage{amsmath}
\usepackage{amsfonts}

% used for TeXing text within eps files
%\usepackage{psfrag}
% need this for including graphics (\includegraphics)
%\usepackage{graphicx}
% for neatly defining theorems and propositions
%\usepackage{amsthm}
% making logically defined graphics
%%%\usepackage{xypic}

% there are many more packages, add them here as you need them

% define commands here
\begin{document}
Let $R$ be a ring. For all $x,y \in R$

$(-x)\cdot y = x\cdot (-y) = -(x\cdot y)$

All we need to prove is that $(-x)\cdot y + x\cdot y = x\cdot (-y) + x\cdot y = 0$

Now: $(-x)\cdot y + x\cdot y = ((-x) + x)\cdot y$ by distributivity.

Since $(-x)+x=0$ by definition and for all $y$, $0\cdot y=0$
 we get:

$(-x)\cdot y + x\cdot y = 0\cdot y = 0$ and thus $(-x)\cdot y = - (x\cdot y)$


For $x\cdot (-y)$, use the previous properties of rings to show that

$x\cdot (-y) + x\cdot y = x\cdot ((-y) +y) = x\cdot 0 = 0$

and thus $x \cdot (-y) = - (x\cdot y)$
%%%%%
%%%%%
\end{document}

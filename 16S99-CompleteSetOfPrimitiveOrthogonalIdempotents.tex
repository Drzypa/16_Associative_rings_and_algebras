\documentclass[12pt]{article}
\usepackage{pmmeta}
\pmcanonicalname{CompleteSetOfPrimitiveOrthogonalIdempotents}
\pmcreated{2013-03-22 19:17:38}
\pmmodified{2013-03-22 19:17:38}
\pmowner{joking}{16130}
\pmmodifier{joking}{16130}
\pmtitle{complete set of primitive orthogonal idempotents}
\pmrecord{4}{42228}
\pmprivacy{1}
\pmauthor{joking}{16130}
\pmtype{Definition}
\pmcomment{trigger rebuild}
\pmclassification{msc}{16S99}
\pmclassification{msc}{20C99}
\pmclassification{msc}{13B99}

\endmetadata

% this is the default PlanetMath preamble.  as your knowledge
% of TeX increases, you will probably want to edit this, but
% it should be fine as is for beginners.

% almost certainly you want these
\usepackage{amssymb}
\usepackage{amsmath}
\usepackage{amsfonts}

% used for TeXing text within eps files
%\usepackage{psfrag}
% need this for including graphics (\includegraphics)
%\usepackage{graphicx}
% for neatly defining theorems and propositions
%\usepackage{amsthm}
% making logically defined graphics
%%%\usepackage{xypic}

% there are many more packages, add them here as you need them

% define commands here

\begin{document}
Let $A$ be a unital algebra over a field $k$. Recall that $e\in A$ is an idempotent iff $e^2=e$. If $e_1,e_2\in A$ are idempotents, then we will say that they are \textbf{orthogonal} iff $e_1e_2=e_2e_1=0$. Furthermore an idempotent $e\in A$ is called \textbf{primitive} iff $e$ cannot be written as a sum $e=e_1+e_2$ where both $e_1,e_2\in A$ are nonzero idempotents. An idempotent is called \textbf{trivial} iff it is either $0$ or $1$.

Now assume that $A$ is an algebra such that
$$A=M_1\oplus M_2$$
as right modules and $1=m_1+m_2$ for some $m_1\in M_1$, $m_2\in M_2$. Then $m_1$, $m_2$ are orthogonal idempotents in $A$ and $M_1=m_1A$, $M_2=m_2A$. Furthermore $M_i$ is indecomposable (as a right module) if and only if $m_i$ is primitive. This can be easily generalized to any number (but finite) of summands.

If $A$ is additionally finite-dimensional, then
$$A=P_1\oplus\cdots\oplus P_n$$
for some (unique up to isomorphism) right (ideals) indecomposable modules $P_i$. It follows from the preceding that
$$P_i=e_iA$$
for some $e_i\in A$ and $\{e_1,\ldots,e_n\}$ is a set of pairwise orthogonal, primitive idempotents. This set is called \textbf{the complete set of primitive orthogonal idempotents of $A$}.
%%%%%
%%%%%
\end{document}

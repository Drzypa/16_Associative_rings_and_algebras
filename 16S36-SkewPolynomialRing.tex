\documentclass[12pt]{article}
\usepackage{pmmeta}
\pmcanonicalname{SkewPolynomialRing}
\pmcreated{2013-03-22 11:49:35}
\pmmodified{2013-03-22 11:49:35}
\pmowner{antizeus}{11}
\pmmodifier{antizeus}{11}
\pmtitle{skew polynomial ring}
\pmrecord{7}{30373}
\pmprivacy{1}
\pmauthor{antizeus}{11}
\pmtype{Definition}
\pmcomment{trigger rebuild}
\pmclassification{msc}{16S36}

\usepackage{amssymb}
\usepackage{amsmath}
\usepackage{amsfonts}
\usepackage{graphicx}
%%%%\usepackage{xypic}
\begin{document}
If $(\sigma, \delta)$ is a left skew derivation on $R$,
then we can construct the
{\it (left) skew polynomial ring}
$R[\theta;\sigma,\delta]$, which is made up of
polynomials in an indeterminate $\theta$
and left-hand coefficients from $R$,
with multiplication satisfying the relation
$$\theta \cdot r = \sigma(r) \cdot \theta + \delta(r)$$
for all $r$ in $R$.
%%%%%
%%%%%
%%%%%
%%%%%
\end{document}

\documentclass[12pt]{article}
\usepackage{pmmeta}
\pmcanonicalname{OrthogonalIdempotentsOfTheGroupRing}
\pmcreated{2013-03-22 14:12:42}
\pmmodified{2013-03-22 14:12:42}
\pmowner{mathcam}{2727}
\pmmodifier{mathcam}{2727}
\pmtitle{orthogonal idempotents of the group ring}
\pmrecord{9}{35646}
\pmprivacy{1}
\pmauthor{mathcam}{2727}
\pmtype{Definition}
\pmcomment{trigger rebuild}
\pmclassification{msc}{16S34}

\endmetadata

% this is the default PlanetMath preamble.  as your knowledge
% of TeX increases, you will probably want to edit this, but
% it should be fine as is for beginners.

% almost certainly you want these
\usepackage{amssymb}
\usepackage{amsmath}
\usepackage{amsfonts}
\usepackage{amsthm}

% used for TeXing text within eps files
%\usepackage{psfrag}
% need this for including graphics (\includegraphics)
%\usepackage{graphicx}
% for neatly defining theorems and propositions
%\usepackage{amsthm}
% making logically defined graphics
%%%\usepackage{xypic}

% there are many more packages, add them here as you need them

% define commands here

\newcommand{\mc}{\mathcal}
\newcommand{\mb}{\mathbb}
\newcommand{\mf}{\mathfrak}
\newcommand{\ol}{\overline}
\newcommand{\ra}{\rightarrow}
\newcommand{\la}{\leftarrow}
\newcommand{\La}{\Leftarrow}
\newcommand{\Ra}{\Rightarrow}
\newcommand{\nor}{\vartriangleleft}
\newcommand{\Gal}{\text{Gal}}
\newcommand{\GL}{\text{GL}}
\newcommand{\Z}{\mb{Z}}
\newcommand{\R}{\mb{R}}
\newcommand{\Q}{\mb{Q}}
\newcommand{\C}{\mb{C}}
\newcommand{\<}{\langle}
\renewcommand{\>}{\rangle}
\begin{document}
Let $G$ be a finite abelian group, let $L$ be any field containing the $|G|$-th roots of unity, and let $\hat{G}$ denote the character group of $G$ with values in $L$.  For any character $\chi\in \hat{G}$, we define $\varepsilon_\chi$, the corresponding \emph{orthogonal idempotent of the group ring} $L[G]$, by
\begin{align*}
\varepsilon_\chi=\frac{1}{|G|}\sum_{g\in G} \chi(g)g^{-1}.
\end{align*}

The following equalities hold:
\begin{itemize}
\item $\varepsilon_\chi^2=\varepsilon_\chi$ for all $\chi$
\item $\varepsilon_\chi\varepsilon_\psi=0$ for any $\chi\neq\psi$
\item $\sum_{\chi\in\hat{G}}\varepsilon_\chi=1$
\item $\varepsilon_\chi\cdot g=\chi(g)\varepsilon_\chi$
\end{itemize}

These orthogonal idempotents are used to decompose modules over $L[G]$:  If $M$ is such a module, then $M=\oplus_\chi (\varepsilon_\chi M)$.
%%%%%
%%%%%
\end{document}

\documentclass[12pt]{article}
\usepackage{pmmeta}
\pmcanonicalname{ModulefiniteExtensionsAreIntegral}
\pmcreated{2013-03-22 17:01:23}
\pmmodified{2013-03-22 17:01:23}
\pmowner{rm50}{10146}
\pmmodifier{rm50}{10146}
\pmtitle{module-finite extensions are integral}
\pmrecord{9}{39308}
\pmprivacy{1}
\pmauthor{rm50}{10146}
\pmtype{Theorem}
\pmcomment{trigger rebuild}
\pmclassification{msc}{16D10}
\pmclassification{msc}{13C05}
\pmclassification{msc}{13B02}
\pmrelated{RingFiniteIntegralExtensionsAreModuleFinite}

\endmetadata

% this is the default PlanetMath preamble.  as your knowledge
% of TeX increases, you will probably want to edit this, but
% it should be fine as is for beginners.

% almost certainly you want these
\usepackage{amssymb}
\usepackage{amsmath}
\usepackage{amsfonts}

% used for TeXing text within eps files
%\usepackage{psfrag}
% need this for including graphics (\includegraphics)
%\usepackage{graphicx}
% for neatly defining theorems and propositions
\usepackage{amsthm}
% making logically defined graphics
%%%\usepackage{xypic}

% there are many more packages, add them here as you need them
\newcommand{\Nats}{\mathbb{N}}
\newcommand{\Ints}{\mathbb{Z}}
\newcommand{\Reals}{\mathbb{R}}
\newcommand{\Complex}{\mathbb{C}}
\newcommand{\Rats}{\mathbb{Q}}
\newcommand{\Gal}{\operatorname{Gal}}
\newcommand{\Cl}{\operatorname{Cl}}
\newcommand{\ol}{\overline}
\newcommand{\Leg}[2]{\left(\frac{#1}{#2}\right)}
\renewcommand{\frak}[1]{\mathfrak{#1}}
%
\newtheorem{thm}{Theorem}
\newtheorem{cor}[thm]{Corollary}
\newtheorem{lem}[thm]{Lemma}
\newtheorem{prop}[thm]{Proposition}
\newtheorem{ax}{Axiom}
% define commands here

\begin{document}
\textbf{Theorem} Suppose $B\subset A$ is module-finite. Then $A$ is integral over $B$.

\textbf{Proof.} 
Choose $u\in A$.

For clarity, assume $A$ is spanned by two elements $\omega_1,\omega_2$. The proof given clearly generalizes to the case where a spanning set for $A$ has more than two elements.

Write
\begin{eqnarray*}
&u\omega_1&=b_{11}\omega_1+b_{12}\omega_2\\
&u\omega_2&=b_{21}\omega_1+b_{22}\omega_2
\end{eqnarray*}

Consider
\[C=\left(\begin{array}{cc}
u-b_{11}&-b_{12}\\
-b_{21}&u-b_{22}\end{array}\right)\]
and let $C^{\mathrm{adj}}$ be the adjugate of $C$.
Then $C\left(\begin{array}{c}\omega_1\\
\omega_2\end{array}\right)=0$, so
$C^{\mathrm{adj}}C\left(\begin{array}{c}\omega_1\\
\omega_2\end{array}\right)=0$.

Now, $C^{\mathrm{adj}}C$ is a diagonal matrix with $\det C$ on the diagonal, so
\[\left(\begin{array}{cc}f(u)&0\\
0&f(u)\end{array}\right)\left(\begin{array}{c}\omega_1\\
\omega_2\end{array}\right)=\left(\begin{array}{c}0\\
0\end{array}\right)\]
where $f\in B[x]$ is monic.

But neither $\omega_1$ nor $\omega_2$ is zero, so $f(u)$ must be.

Note that, as with the field case, the converse is not true. For example, the algebraic integers are integral but not finite over $\Ints$.
%%%%%
%%%%%
\end{document}

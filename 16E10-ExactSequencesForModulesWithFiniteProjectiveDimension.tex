\documentclass[12pt]{article}
\usepackage{pmmeta}
\pmcanonicalname{ExactSequencesForModulesWithFiniteProjectiveDimension}
\pmcreated{2013-03-22 19:04:55}
\pmmodified{2013-03-22 19:04:55}
\pmowner{joking}{16130}
\pmmodifier{joking}{16130}
\pmtitle{exact sequences for modules with finite projective dimension}
\pmrecord{4}{41970}
\pmprivacy{1}
\pmauthor{joking}{16130}
\pmtype{Corollary}
\pmcomment{trigger rebuild}
\pmclassification{msc}{16E10}
\pmclassification{msc}{18G20}
\pmclassification{msc}{18G10}

\endmetadata

% this is the default PlanetMath preamble.  as your knowledge
% of TeX increases, you will probably want to edit this, but
% it should be fine as is for beginners.

% almost certainly you want these
\usepackage{amssymb}
\usepackage{amsmath}
\usepackage{amsfonts}

% used for TeXing text within eps files
%\usepackage{psfrag}
% need this for including graphics (\includegraphics)
%\usepackage{graphicx}
% for neatly defining theorems and propositions
%\usepackage{amsthm}
% making logically defined graphics
%%%\usepackage{xypic}

% there are many more packages, add them here as you need them

% define commands here

\begin{document}
\textbf{Proposition.} Let $R$ be a ring and $M$ be a (left) $R$-module, such that $\mbox{proj dim}(M)=n<\infty$. If
$$0\rightarrow K\rightarrow P_{n}\rightarrow\cdots\rightarrow P_{0}\rightarrow M\rightarrow 0$$
is an exact sequence of $R$-modules, such that each $P_{i}$ is projective, then $K$ is projective.

\textit{Proof.} Since $\mbox{proj dim}(M)=n<\infty$, then there exists exact sequence of $R$-modules
$$0\rightarrow P'_{n}\rightarrow\cdots\rightarrow P'_{0}\rightarrow M\rightarrow 0,$$
Note that sequences
$$P_{n}\rightarrow\cdots\rightarrow P_{0}\rightarrow M\rightarrow 0;$$
$$P'_{n}\rightarrow\cdots\rightarrow P_{0}\rightarrow M\rightarrow 0,$$
are projective resolutions of $M$. Let $\delta:P_{n}\rightarrow P_{n-1}$ and $\beta:P'_{n}\rightarrow P'_{n-1}$ be maps take from these resolutions. Then generalized Schanuel's lemma implies that $\mathrm{ker}\delta$ and $\mathrm{ker}\beta$ are projectively equivalent. But $\mathrm{ker}\delta\simeq K$ and $\mathrm{ker}\beta=0$. This means, that there are projective modules $P,Q$ such that
$$K\oplus P\simeq Q.$$
Therefore $K$ is a direct summand of a free module (since $Q$ is), which completes the proof. $\square$
%%%%%
%%%%%
\end{document}

\documentclass[12pt]{article}
\usepackage{pmmeta}
\pmcanonicalname{MaximalIdealIsPrime}
\pmcreated{2013-03-22 17:37:59}
\pmmodified{2013-03-22 17:37:59}
\pmowner{pahio}{2872}
\pmmodifier{pahio}{2872}
\pmtitle{maximal ideal is prime}
\pmrecord{8}{40054}
\pmprivacy{1}
\pmauthor{pahio}{2872}
\pmtype{Theorem}
\pmcomment{trigger rebuild}
\pmclassification{msc}{16D25}
\pmclassification{msc}{13A15}
\pmrelated{SumOfIdeals}
\pmrelated{MaximumIdealIsPrimeGeneralCase}
\pmrelated{CriterionForMaximalIdeal}

% this is the default PlanetMath preamble.  as your knowledge
% of TeX increases, you will probably want to edit this, but
% it should be fine as is for beginners.

% almost certainly you want these
\usepackage{amssymb}
\usepackage{amsmath}
\usepackage{amsfonts}

% used for TeXing text within eps files
%\usepackage{psfrag}
% need this for including graphics (\includegraphics)
%\usepackage{graphicx}
% for neatly defining theorems and propositions
 \usepackage{amsthm}
% making logically defined graphics
%%%\usepackage{xypic}

% there are many more packages, add them here as you need them

% define commands here

\theoremstyle{definition}
\newtheorem*{thmplain}{Theorem}

\begin{document}
\textbf{Theorem.}  In a commutative ring with non-zero unity, any maximal ideal is a prime ideal.

{\em Proof.}\, Let $\mathfrak{m}$ be a maximal ideal of such a ring $R$ and let the ring product $rs$ belong to $\mathfrak{m}$ but e.g. \,$r \notin \mathfrak{m}$.  The maximality of $\mathfrak{m}$ implies that\, 
$\mathfrak{m}\!+\!(r) = R = (1)$.\, Thus there exists an element \,$m \in \mathfrak{m}$\, and an element\, $x \in R$\, such that\, $m\!+\!xr = 1$.\, Now $m$ and $rs$ belong to $\mathfrak{m}$, whence
       $$s = 1s = (m\!+\!xr)s = sm\!+\!x(rs) \in \mathfrak{m}.$$ 
So we can say that along with $rs$, at least one of its \PMlinkname{factors}{Product} belongs to $\mathfrak{m}$, and therefore $\mathfrak{m}$ is a prime ideal of $R$.
%%%%%
%%%%%
\end{document}

\documentclass[12pt]{article}
\usepackage{pmmeta}
\pmcanonicalname{TheHamiltonianRingIsNotAComplexAlgebra}
\pmcreated{2013-03-22 16:01:57}
\pmmodified{2013-03-22 16:01:57}
\pmowner{Algeboy}{12884}
\pmmodifier{Algeboy}{12884}
\pmtitle{The Hamiltonian ring is not a complex algebra}
\pmrecord{10}{38076}
\pmprivacy{1}
\pmauthor{Algeboy}{12884}
\pmtype{Result}
\pmcomment{trigger rebuild}
\pmclassification{msc}{16W99}

\endmetadata

\usepackage{latexsym}
\usepackage{amssymb}
\usepackage{amsmath}
\usepackage{amsfonts}
\usepackage{amsthm}

%%\usepackage{xypic}

%-----------------------------------------------------

%       Standard theoremlike environments.

%       Stolen directly from AMSLaTeX sample

%-----------------------------------------------------

%% \theoremstyle{plain} %% This is the default

\newtheorem{thm}{Theorem}

\newtheorem{coro}[thm]{Corollary}

\newtheorem{lem}[thm]{Lemma}

\newtheorem{lemma}[thm]{Lemma}

\newtheorem{prop}[thm]{Proposition}

\newtheorem{conjecture}[thm]{Conjecture}

\newtheorem{conj}[thm]{Conjecture}

\newtheorem{defn}[thm]{Definition}

\newtheorem{remark}[thm]{Remark}

\newtheorem{ex}[thm]{Example}



%\countstyle[equation]{thm}



%--------------------------------------------------

%       Item references.

%--------------------------------------------------


\newcommand{\exref}[1]{Example-\ref{#1}}

\newcommand{\thmref}[1]{Theorem-\ref{#1}}

\newcommand{\defref}[1]{Definition-\ref{#1}}

\newcommand{\eqnref}[1]{(\ref{#1})}

\newcommand{\secref}[1]{Section-\ref{#1}}

\newcommand{\lemref}[1]{Lemma-\ref{#1}}

\newcommand{\propref}[1]{Prop\-o\-si\-tion-\ref{#1}}

\newcommand{\corref}[1]{Cor\-ol\-lary-\ref{#1}}

\newcommand{\figref}[1]{Fig\-ure-\ref{#1}}

\newcommand{\conjref}[1]{Conjecture-\ref{#1}}


% Normal subgroup or equal.

\providecommand{\normaleq}{\unlhd}

% Normal subgroup.

\providecommand{\normal}{\lhd}

\providecommand{\rnormal}{\rhd}
% Divides, does not divide.

\providecommand{\divides}{\mid}

\providecommand{\ndivides}{\nmid}


\providecommand{\union}{\cup}

\providecommand{\bigunion}{\bigcup}

\providecommand{\intersect}{\cap}

\providecommand{\bigintersect}{\bigcap}










\begin{document}
The \PMlinkname{Hamiltonian algebra}{QuaternionAlgebra2} $\mathbb{H}$ contains isomorphic copies of the real $\mathbb{R}$ and complex $\mathbb{C}$ numbers.  However, the reals are a central subalgebra of $\mathbb{H}$ which makes $\mathbb{H}$ into a real algebra.  This makes identifying $\mathbb{R}$ in $\mathbb{H}$ canonical: $1\in \mathbb{H}$ determines a unique embedding $\mathbb{R}\rightarrow \mathbb{H}:r\mapsto r1$.  Yet $\mathbb{H}$ is not a complex algebra.  The goal presently is to outline some of the incongruities of $\mathbb{C}=\langle 1,i\rangle$ and $\mathbb{H}=\langle 1,\hat{\imath},\hat{\jmath},\hat{k}\rangle$ which may be obscured by the notational overlap of the letter $i$.

\begin{prop}
There are no proper finite dimensional division rings over algebraically closed fields.
\end{prop}
\begin{proof}
Let $D$ be a finite dimensional division ring over an algebraically closed field $K$.  This means that $K$ is a central subalgebra of $D$.  Let $a\in D$ and consider
$K(a)$.  Since $K$ is central in $D$, $K(a)$ is commutative, and so $K(a)$ is a field extension of $K$.  But as $D$ is a finite dimensional $K$ space, so is $K(a)$.  As any finite dimensional extension of $K$ is algebraic, $K(a)$ is an algebraic extension.  Yet $K$ is algebraically closed so $K(a)=K$.  Thus $a\in K$ so in fact $D=K$.
\end{proof}
\begin{itemize}
\item In particular, this proposition proves $\mathbb{H}$ is not a complex algebra.
\item
Alternatively, from the Wedderburn-Artin theorem we know the only semisimple complex algebra of dimension 2 is $\mathbb{C}\oplus\mathbb{C}$.  This has proper ideals and so it cannot be the division ring $\mathbb{H}$.
\item
It is also evident that the usual, notationally driven, embedding of $\mathbb{C}$ into $\mathbb{H}$ is non-central.  That is, $\mathbb{C}$ embeds as $a+bi\mapsto a+b\hat{i}$, into $\mathbb{H}=\langle 1, \hat{\imath},\hat{\jmath},\hat{k}\rangle$.  This is not central:
\[(1+\hat{\imath})\hat{\jmath}=\hat{\jmath}+\hat{k}\neq \hat{\jmath}(1+\hat{\imath})=\hat{\jmath}-\hat{k}.\]
\item
Further evidence of the incompatiblity of $\mathbb{H}$ and $\mathbb{C}$ comes from considering polynomials.  If $x^2+1$ is considered as a polynomial over $\mathbb{C}[x]$ then it has exactly two roots $i, -i$ as expected.  However, if it is considered as a polynomial over $\mathbb{H}[x]$ we arrive at 6 obvious roots: $\{\hat{\imath},-\hat{\imath},\hat{\jmath},-\hat{\jmath},\hat{k},-\hat{k}\}$.  But indeed, given any $q\in\mathbb{H}$, $q\neq 0$, then $q\hat{\imath}q^{-1}$ is also a root.  Thus there are an infinite number of roots to $x^2+1$.  Therefore declaring  $\hat{\imath}=\sqrt{-1}$ can be greatly misleading.  Such a conflict does not arise for polynomials with real roots since $\mathbb{R}$ is a central subalgebra.
\end{itemize}

%%%%%
%%%%%
\end{document}

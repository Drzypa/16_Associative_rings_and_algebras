\documentclass[12pt]{article}
\usepackage{pmmeta}
\pmcanonicalname{JacobsonRadicalOfAModuleCategoryAndItsPower}
\pmcreated{2013-03-22 19:02:23}
\pmmodified{2013-03-22 19:02:23}
\pmowner{joking}{16130}
\pmmodifier{joking}{16130}
\pmtitle{Jacobson radical of a module category and its power}
\pmrecord{6}{41916}
\pmprivacy{1}
\pmauthor{joking}{16130}
\pmtype{Definition}
\pmcomment{trigger rebuild}
\pmclassification{msc}{16N20}

% this is the default PlanetMath preamble.  as your knowledge
% of TeX increases, you will probably want to edit this, but
% it should be fine as is for beginners.

% almost certainly you want these
\usepackage{amssymb}
\usepackage{amsmath}
\usepackage{amsfonts}

% used for TeXing text within eps files
%\usepackage{psfrag}
% need this for including graphics (\includegraphics)
%\usepackage{graphicx}
% for neatly defining theorems and propositions
%\usepackage{amsthm}
% making logically defined graphics
%%%\usepackage{xypic}

% there are many more packages, add them here as you need them

% define commands here

\begin{document}
Assume that $k$ is a field and $A$ is a $k$-algebra. The category of (left) $A$-modules will be denoted by $\mathrm{Mod}(A)$ and $\mathrm{Hom}_{A}(X,Y)$ will denote the set of all $A$-homomorphisms between $A$-modules $X$ and $Y$. Of course $\mathrm{Hom}_{A}(X,Y)$ is an $A$-module itself and $\mathrm{End}_{A}(X)=\mathrm{Hom}_{A}(X,X)$ is a $k$-algebra (even $A$-algebra) with composition as a multiplication.

Let $X$ and $Y$ be $A$-modules. Define
$$\mathrm{rad}_{A}(X,Y)=\{f\in\mathrm{Hom}_{A}(X,Y)\ |\ \forall_{g\in\mathrm{Hom}_{A}(Y,X)}\ 1_{X}-gf \mbox{ is invertible in }\mathrm{End}_{A}(X)\}.$$
\textbf{Definition.} \textit{The Jacobson radical} of a category $\mathrm{Mod}(A)$ is defined as a class
$$\mathrm{rad}\ \mathrm{Mod}(A)=\bigcup_{X,Y\in\mathrm{Mod}(A)}\ \mathrm{rad}_{A}(X,Y). \square$$

\textbf{Properties.} $1)$ The Jacobson radical is an ideal in $\mathrm{Mod}(A)$, i.e. for any $X,Y,Z\in\mathrm{Mod}(A)$, for any $f\in\mathrm{rad}_{A}(X,Y)$, any $h\in\mathrm{Hom}_{A}(Y,Z)$ and any $g\in\mathrm{Hom}_{A}(Z,X)$ we have $hf\in\mathrm{rad}_{A}(X,Z)$ and $fg\in\mathrm{rad}_{A}(Z,X)$. Additionaly $\mathrm{rad}_{A}(X,Y)$ is an $A$-submodule of $\mathrm{Hom}_{A}(X,Y)$.

$2)$ For any $A$-module $X$ we have $\mathrm{rad}_{A}(X,X)=\mathrm{rad}(\mathrm{End}_{A}(X))$, where on the right side we have the classical Jacobson radical.

$3)$ If $X,Y$ are both indecomposable $A$-modules such that both $\mathrm{End}_{A}(X)$ and $\mathrm{End}_{A}(Y)$ are local algebras (in particular, if $X$ and $Y$ are finite dimensional), then 
$$\mathrm{rad}_{A}(X,Y)=\{f\in\mathrm{Hom}_{A}(X,Y)\ |\ f\mbox{ is not an isomorphism}\}.$$
In particular, if $X$ and $Y$ are not isomorphic, then $\mathrm{rad}_{A}(X,Y)=\mathrm{Hom}_{A}(X,Y)$. $\square$

Let $n\in\mathbb{N}$ and let $f\in\mathrm{Hom}_{A}(X,Y)$. Assume there is a sequence of $A$-modules $X=X_0,X_1,\ldots,X_{n-1},X_n=Y$ and for any $0\leq i\leq n-1$ we have an $A$-homomorphism $f_i\in\mathrm{rad}_{A}(X_i,X_{i+1})$ such that $f=f_{n-1}f_{n-2}\cdots f_{1}f_{0}$. Then we will say that $f$ is \textit{$n$-factorizable through Jacobson radical}.

\textbf{Definition.} \textit{The $n$-th power of a Jacobson radical} of a category $\mathrm{Mod}(A)$ is defined as a class
$$\mathrm{rad}^{n}\ \mathrm{Mod}(A)=\bigcup_{X,Y\in\mathrm{Mod}(A)}\ \mathrm{rad}^{n}_{A}(X,Y),$$
where $\mathrm{rad}^{n}_{A}(X,Y)$ is an $A$-submodule of $\mathrm{Hom}_{A}(X,Y)$ generated by all homomorphisms $n$-factorizable through Jacobson radical. Additionaly define 
$$\mathrm{rad}^{\infty}_{A}(X,Y)=\bigcap_{n=1}^{\infty}\mathrm{rad}^{n}_{A}(X,Y)\ \square$$

\textbf{Properties.} $0)$ Obviously $\mathrm{rad}_{A}(X,Y)=\mathrm{rad}^{1}_{A}(X,Y)$ and for any $n\in\mathbb{N}$ we have $$\mathrm{rad}_{A}^{n}(X,Y)\supseteq\mathrm{rad}_{A}^{\infty}(X,Y).$$
$1)$ Of course each $\mathrm{rad}^{n}_{A}(X,Y)$ is an $A$-submodule of $\mathrm{Hom}_{A}(X,Y)$ and we have following sequence of inclusions:
$$\mathrm{Hom}_{A}(X,Y)\supseteq\mathrm{rad}^{1}_{A}(X,Y)\supseteq\mathrm{rad}^{2}_{A}(X,Y)\supseteq\mathrm{rad}^{3}_{A}(X,Y)\supseteq\cdots$$
$2)$ If both $X$ and $Y$ are finite dimensional, then there exists $n\in\mathbb{N}$ such that
$$\mathrm{rad}^{\infty}_{A}(X,Y)=\mathrm{rad}^{n}_{A}(X,Y).$$

%%%%%
%%%%%
\end{document}

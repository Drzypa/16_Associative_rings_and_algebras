\documentclass[12pt]{article}
\usepackage{pmmeta}
\pmcanonicalname{HopfAlgebra}
\pmcreated{2013-03-22 13:06:08}
\pmmodified{2013-03-22 13:06:08}
\pmowner{mhale}{572}
\pmmodifier{mhale}{572}
\pmtitle{Hopf algebra}
\pmrecord{12}{33524}
\pmprivacy{1}
\pmauthor{mhale}{572}
\pmtype{Definition}
\pmcomment{trigger rebuild}
\pmclassification{msc}{16W30}
\pmrelated{GroupoidAndGroupRepresentationsRelatedToQuantumSymmetries}
\pmrelated{QuantumGroups}
\pmrelated{QuantumGroupoids2}
\pmrelated{WeakHopfCAlgebra2}
\pmrelated{WeakHopfCAlgebra}
\pmrelated{LocallyCompactQuantumGroupsUniformContinuity}
\pmrelated{FiniteQuantumGroup}
\pmdefines{antipode}

\endmetadata

\usepackage{amssymb}
\usepackage{amsmath}
\usepackage{amsfonts}
\usepackage{amsthm}

% used for TeXing text within eps files
%\usepackage{psfrag}
% need this for including graphics (\includegraphics)
%\usepackage{graphicx}
% making logically defined graphics
\usepackage[all]{xy}

% my maths package

\newcommand*{\Nset}{\mathbb{N}}
\newcommand*{\Zset}{\mathbb{Z}}
\newcommand*{\Qset}{\mathbb{Q}}
\newcommand*{\Rset}{\mathbb{R}}
\newcommand*{\Cset}{\mathbb{C}}
\newcommand*{\Hset}{\mathbb{H}}
\newcommand*{\Oset}{\mathbb{O}}
\newcommand*{\Bset}{\mathbb{B}}
\newcommand*{\Kset}{\mathbb{K}}
\newcommand*{\Sset}{\mathbb{S}}
\newcommand*{\Tset}{\mathbb{T}}
\newcommand*{\GLgrp}{\mathrm{GL}}
\newcommand*{\SLgrp}{\mathrm{SL}}
\newcommand*{\Ogrp}{\mathrm{O}}
\newcommand*{\SOgrp}{\mathrm{SO}}
\newcommand*{\Ugrp}{\mathrm{U}}
\newcommand*{\SUgrp}{\mathrm{SU}}
\newcommand*{\e}{\mathop{\mathrm{e}}\nolimits}
\newcommand*{\im}{\mathord{\mathrm{i}}}
\newcommand*{\identity}{\mathord{\mathrm{1\!\!\!\:I}}}
\newcommand*{\tr}{\mathop{\mathrm{tr}}}
\newcommand*{\Tr}{\mathop{\mathrm{Tr}}}
\renewcommand*{\d}{\mathrm{d}}
\newcommand*{\deriv}[2]{\frac{\d #1}{\d #2}}
\newcommand*{\pderiv}[2]{\frac{\partial #1}{\partial #2}}
\newcommand*{\fderiv}[2]{\frac{\delta #1}{\delta #2}}

% my noncommutative geometry package

\newcommand*{\algebra}[1][A]{\mathord{\mathcal{#1}}}
\newcommand*{\hilbert}[1][H]{\mathord{\mathcal{#1}}}
\newcommand*{\hilbmod}[1][E]{\mathord{\mathcal{#1}}}
\newcommand*{\Matrix}[2]{\mathord{\mathrm{M}_{#1}(#2)}}
\newcommand*{\dixmier}{\mathop{\mathrm{Tr}_\omega}}
\newcommand*{\Res}{\mathop{\mathrm{Res}}}
\newcommand*{\Wres}{\mathop{\mathrm{Wres}}}
\newcommand*{\Aut}{\mathop{\mathrm{Aut}}\nolimits}
\newcommand*{\Inn}{\mathop{\mathrm{Inn}}\nolimits}
\newcommand*{\Out}{\mathop{\mathrm{Out}}\nolimits}
\newcommand*{\Diff}{\mathop{\mathrm{Diff}}\nolimits}
\newcommand*{\Ker}{\mathop{\mathrm{Ker}}\nolimits}
\newcommand*{\Coker}{\mathop{\mathrm{Coker}}\nolimits}
\newcommand*{\Img}{\mathop{\mathrm{Im}}\nolimits}
\newcommand*{\End}{\mathop{\mathrm{End}}\nolimits}
\newcommand*{\spin}{\mathop{\mathrm{spin}}\nolimits}
\newcommand*{\Ind}{\mathop{\mathrm{Ind}}\nolimits}
\newcommand*{\KK}{\mathit{KK}}
\newcommand*{\HH}{\mathit{HH}}
\newcommand*{\HC}{\mathit{HC}}
\newcommand*{\ch}{\mathop{\mathrm{ch}}\nolimits}

% my category theory package

\newcommand*{\mathcat}[1]{\mathord{\mathbf{#1}}}
\newcommand*{\id}{\mathrm{id}}
\newcommand*{\op}{\mathrm{op}}
\newcommand*{\boxprod}{\mathbin{\square}}

% my environments

\newtheoremstyle{inlinedefn}{}{0pt}{}{}{\bfseries}{.}{0.5em}{}
\theoremstyle{inlinedefn}
\newtheorem{definition}{Definition}

\newtheoremstyle{break}{\baselineskip}{\baselineskip}{\itshape}{}{\bfseries}{}{\newline}{}
\theoremstyle{break}
\newtheorem{example}{Example}

% misc commands

\newcommand*{\defn}[1]{\textbf{#1}}
\begin{document}
A \textbf{Hopf algebra} is a bialgebra $A$ over a field $\Kset$ with a $\Kset$-linear map $S : A \to A$,
called the \textbf{antipode}, such that
\begin{equation}
m\circ(S\otimes\id)\circ\Delta = \eta\circ\varepsilon = m\circ(\id\otimes S)\circ\Delta,
\end{equation}
where $m : A\otimes A \to A$ is the multiplication map $m(a\otimes b) = ab$ and $\eta : \Kset \to A$ is the unit map $\eta(k) = k\identity$.

In \PMlinkescapetext{terms} of a commutative diagram:
\[\begin{xy}
\xymatrix@R=20pt@C=70pt{
& *+<10pt>\txt{$A$} \ar_{\Delta}[dl] \ar_{\varepsilon}[dd] \ar^{\Delta}[dr] & \\
*+<10pt>\txt{$A\otimes A$} \ar_{S\otimes\id}[dd] & & *+<10pt>\txt{$A\otimes A$} \ar^{\id\otimes S}[dd] \\
& *+<10pt>\txt{$\Kset$} \ar_{\eta}[dd] & \\
*+<10pt>\txt{$A\otimes A$} \ar_{m}[dr] & & *+<10pt>\txt{$A\otimes A$} \ar^{m}[dl] \\
& *+<10pt>\txt{$A$} &
}\end{xy}\]

The category of commutative Hopf algebras is anti-equivalent to the category of affine group schemes.
The prime spectrum of a commutative Hopf algebra is an affine group scheme of multiplicative units.
And going in the opposite direction, the algebra of natural transformations from an affine group scheme
to its \PMlinkname{affine 1-space}{AffineSpace} is a commutative Hopf algebra,
with coalgebra structure given by dualising the group structure of the affine group scheme.
Further, a commutative Hopf algebra is a cogroup object in the category of commutative algebras.

% Examples

\begin{example}[Algebra of functions on a finite group]
Let $A = C(G)$ be the algebra of complex-valued functions on a finite group $G$
and identify $C(G\times G)$ with $A \otimes A$.
Then, $A$ is a Hopf algebra with comultiplication $(\Delta(f))(x,y) = f(xy)$,
counit $\varepsilon(f) = f(e)$,
and antipode $(S(f))(x) = f(x^{-1})$.
\end{example}

\begin{example}[Group algebra of a finite group]
Let $A = \Cset G$ be the complex group algebra of a finite group $G$.
Then, $A$ is a Hopf algebra with comultiplication $\Delta(g) = g \otimes g$,
counit $\varepsilon(g) = 1$,
and antipode $S(g) = g^{-1}$.
\end{example}

The above two examples are dual to one another.
Define a bilinear form $C(G) \otimes \Cset G \to \Cset$ by
$\langle f,x \rangle = f(x)$.
Then,
\begin{eqnarray*}
\langle fg, x \rangle & = & \langle f \otimes g, \Delta(x) \rangle, \\
\langle 1, x \rangle & = & \varepsilon(x), \\
\langle \Delta(f), x \otimes y \rangle & = & \langle f, xy \rangle, \\
\varepsilon(f) & = & \langle f, e \rangle, \\
\langle S(f), x \rangle & = & \langle f, S(x) \rangle, \\
\langle f^*, x \rangle & = & \overline{\langle f, S(x)^* \rangle}.
\end{eqnarray*}

\begin{example}[Polynomial functions on a Lie group]
Let $A = \mathrm{Poly}(G)$ be the algebra of complex-valued polynomial functions on a complex Lie group $G$ and identify $\mathrm{Poly}(G\times G)$ with $A \otimes A$.
Then, $A$ is a Hopf algebra with comultiplication $(\Delta(f))(x,y) = f(xy)$,
counit $\varepsilon(f) = f(e)$,
and antipode $(S(f))(x) = f(x^{-1})$.
\end{example}

\begin{example}[Universal enveloping algebra of a Lie algebra]
Let $A = \mathcal{U}(\mathfrak{g})$ be the universal enveloping algebra of a complex Lie algebra $\mathfrak{g}$.
Then, $A$ is a Hopf algebra with comultiplication $\Delta(X) = X\otimes1+1\otimes X$,
counit $\varepsilon(X) = 0$,
and antipode $S(X) = -X$.
\end{example}

The above two examples are dual to one another (if $\mathfrak{g}$ is the Lie algebra of $G$).
Define a bilinear form $\mathrm{Poly}(G) \otimes \mathcal{U}(\mathfrak{g}) \to \Cset$ by
$\langle f,X \rangle = \deriv{}{t}\big|_{t=0} f(\exp(tX))$.
%%%%%
%%%%%
\end{document}

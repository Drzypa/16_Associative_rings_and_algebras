\documentclass[12pt]{article}
\usepackage{pmmeta}
\pmcanonicalname{Unit}
\pmcreated{2013-03-22 11:56:28}
\pmmodified{2013-03-22 11:56:28}
\pmowner{drini}{3}
\pmmodifier{drini}{3}
\pmtitle{unit}
\pmrecord{15}{30676}
\pmprivacy{1}
\pmauthor{drini}{3}
\pmtype{Definition}
\pmcomment{trigger rebuild}
\pmclassification{msc}{16B99}
\pmsynonym{unital}{Unit}
%\pmkeywords{Ring}
%\pmkeywords{Factorization}
\pmrelated{Associates}
\pmrelated{Prime}
\pmrelated{Ring}
\pmrelated{UnitsOfQuadraticFields}
\pmdefines{algebraic unit}

\usepackage{graphicx}
%%%%\usepackage{xypic} 
\usepackage{bbm}
\newcommand{\Z}{\mathbbmss{Z}}
\newcommand{\C}{\mathbbmss{C}}
\newcommand{\R}{\mathbbmss{R}}
\newcommand{\Q}{\mathbbmss{Q}}
\newcommand{\mathbb}[1]{\mathbbmss{#1}}
\newcommand{\figura}[1]{\begin{center}\includegraphics{#1}\end{center}}
\newcommand{\figuraex}[2]{\begin{center}\includegraphics[#2]{#1}\end{center}}
\begin{document}
Let $R$ be a ring with multiplicative identity $1$. We say that $u\in R$ is an unit (or unital) if $u$ divides $1$ (denoted $u \mid 1$). That is, there exists an $r\in R$  such that $1=ur=ru$.

Notice that $r$ will be the multiplicative inverse (in the ring) of $u$, so we can characterize the units as those elements of the ring having multiplicative inverses.

In the special case  that $R$ is the ring of integers of an algebraic number field $K$, the units of $R$ are sometimes called the {\em algebraic units} of $K$ (and also the units of $K$).\, They are determined by Dirichlet's unit theorem.
%%%%%
%%%%%
%%%%%
\end{document}

\documentclass[12pt]{article}
\usepackage{pmmeta}
\pmcanonicalname{Noetherian}
\pmcreated{2013-03-22 12:26:53}
\pmmodified{2013-03-22 12:26:53}
\pmowner{antizeus}{11}
\pmmodifier{antizeus}{11}
\pmtitle{noetherian}
\pmrecord{5}{32575}
\pmprivacy{1}
\pmauthor{antizeus}{11}
\pmtype{Definition}
\pmcomment{trigger rebuild}
\pmclassification{msc}{16P40}
\pmsynonym{left noetherian}{Noetherian}
\pmsynonym{right noetherian}{Noetherian}
\pmrelated{Artinian}
\pmrelated{Noetherian}
\pmrelated{HollowMatrixRings}

\endmetadata

\usepackage{amssymb}
\usepackage{amsmath}
\usepackage{amsfonts}
\begin{document}
A module $M$ is {\it noetherian} if it satisfies the following equivalent conditions:
\begin{itemize}
\item the ascending chain condition holds for submodules of $M$ ;
\item every nonempty family of submodules of $M$ has a maximal element;
\item every submodule of $M$ is finitely generated.
\end{itemize}

A ring $R$ is {\it left noetherian} 
if it is noetherian as a left module over itself 
(i.e. if $_RR$ is a \PMlinkescapetext{noetherian module}), 
and {\it right noetherian} 
if it is noetherian as a right module over itself 
(i.e. if $R_R$ is an \PMlinkescapetext{noetherian module}), 
and simply {\it noetherian} 
if both conditions hold.
%%%%%
%%%%%
\end{document}

\documentclass[12pt]{article}
\usepackage{pmmeta}
\pmcanonicalname{LengthOfAModule}
\pmcreated{2013-03-22 14:35:32}
\pmmodified{2013-03-22 14:35:32}
\pmowner{CWoo}{3771}
\pmmodifier{CWoo}{3771}
\pmtitle{length of a module}
\pmrecord{11}{36156}
\pmprivacy{1}
\pmauthor{CWoo}{3771}
\pmtype{Definition}
\pmcomment{trigger rebuild}
\pmclassification{msc}{16D10}
\pmclassification{msc}{13C15}
\pmsynonym{finite-length module}{LengthOfAModule}
%\pmkeywords{length}
\pmdefines{finite length}

% this is the default PlanetMath preamble.  as your knowledge
% of TeX increases, you will probably want to edit this, but
% it should be fine as is for beginners.

% almost certainly you want these
\usepackage{amssymb}
\usepackage{amsmath}
\usepackage{amsfonts}
\usepackage{amsthm}

% used for TeXing text within eps files
%\usepackage{psfrag}
% need this for including graphics (\includegraphics)
%\usepackage{graphicx}
% for neatly defining theorems and propositions
%\usepackage{amsthm}
% making logically defined graphics
%%%\usepackage{xypic}

% there are many more packages, add them here as you need them

% define commands here

\newcommand{\mc}{\mathcal}
\newcommand{\mb}{\mathbb}
\newcommand{\mf}{\mathfrak}
\newcommand{\ol}{\overline}
\newcommand{\ra}{\rightarrow}
\newcommand{\la}{\leftarrow}
\newcommand{\La}{\Leftarrow}
\newcommand{\Ra}{\Rightarrow}
\newcommand{\nor}{\vartriangleleft}
\newcommand{\Gal}{\text{Gal}}
\newcommand{\GL}{\text{GL}}
\newcommand{\Z}{\mb{Z}}
\newcommand{\R}{\mb{R}}
\newcommand{\Q}{\mb{Q}}
\newcommand{\C}{\mb{C}}
\newcommand{\<}{\langle}
\renewcommand{\>}{\rangle}
\begin{document}
Let $A$ be a ring and let $M$ be an $A$-module.  If there is a finite sequence of submodules of $M$
\begin{align*}
M=M_0\supset M_1\supset \cdots \supset M_n=0
\end{align*}
such that each quotient module $M_i/M_{i+1}$ is simple, then $n$ is necessarily unique by the \PMlinkname{Jordan-H\"older theorem}{JordanHolderDecomposition} for modules.  We define the above number $n$ to be the \emph{length} of $M$.  If such a finite sequence does not exist, then the length of $M$ is defined to be $\infty$.

If $M$ has finite length, then $M$ satisfies both the ascending and descending chain conditions.

A ring $A$ is said to have \emph{finite length} if there is an $A$-module whose length is finite.
%%%%%
%%%%%
\end{document}

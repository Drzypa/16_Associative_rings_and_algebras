\documentclass[12pt]{article}
\usepackage{pmmeta}
\pmcanonicalname{IdempotentClassifications}
\pmcreated{2013-03-22 16:48:43}
\pmmodified{2013-03-22 16:48:43}
\pmowner{Algeboy}{12884}
\pmmodifier{Algeboy}{12884}
\pmtitle{idempotent classifications}
\pmrecord{9}{39046}
\pmprivacy{1}
\pmauthor{Algeboy}{12884}
\pmtype{Definition}
\pmcomment{trigger rebuild}
\pmclassification{msc}{16U99}
\pmclassification{msc}{20M99}
%\pmkeywords{primitive idempotent}
\pmdefines{division idempotent}
\pmdefines{local idempotent}

\usepackage{latexsym}
\usepackage{amssymb}
\usepackage{amsmath}
\usepackage{amsfonts}
\usepackage{amsthm}
\usepackage{enumerate}

%%\usepackage{xypic}

%-----------------------------------------------------

%       Standard theoremlike environments.

%       Stolen directly from AMSLaTeX sample

%-----------------------------------------------------

%% \theoremstyle{plain} %% This is the default

\newtheorem{thm}{Theorem}

\newtheorem{coro}[thm]{Corollary}

\newtheorem{lem}[thm]{Lemma}

\newtheorem{lemma}[thm]{Lemma}

\newtheorem{prop}[thm]{Proposition}

\newtheorem{conjecture}[thm]{Conjecture}

\newtheorem{conj}[thm]{Conjecture}

\newtheorem{defn}[thm]{Definition}

\newtheorem{remark}[thm]{Remark}

\newtheorem{ex}[thm]{Example}



%\countstyle[equation]{thm}



%--------------------------------------------------

%       Item references.

%--------------------------------------------------


\newcommand{\exref}[1]{Example-\ref{#1}}

\newcommand{\thmref}[1]{Theorem-\ref{#1}}

\newcommand{\defref}[1]{Definition-\ref{#1}}

\newcommand{\eqnref}[1]{(\ref{#1})}

\newcommand{\secref}[1]{Section-\ref{#1}}

\newcommand{\lemref}[1]{Lemma-\ref{#1}}

\newcommand{\propref}[1]{Prop\-o\-si\-tion-\ref{#1}}

\newcommand{\corref}[1]{Cor\-ol\-lary-\ref{#1}}

\newcommand{\figref}[1]{Fig\-ure-\ref{#1}}

\newcommand{\conjref}[1]{Conjecture-\ref{#1}}


% Normal subgroup or equal.

\providecommand{\normaleq}{\unlhd}

% Normal subgroup.

\providecommand{\normal}{\lhd}

\providecommand{\rnormal}{\rhd}
% Divides, does not divide.

\providecommand{\divides}{\mid}

\providecommand{\ndivides}{\nmid}


\providecommand{\union}{\cup}

\providecommand{\bigunion}{\bigcup}

\providecommand{\intersect}{\cap}

\providecommand{\bigintersect}{\bigcap}










\begin{document}
An a unital ring $R$, an idempotent $e\in R$ is called a \emph{division idempotent}
if $eRe=\{ere:r\in R\}$, with the product of $R$, forms a division ring.
If instead $eRe$ is a local ring -- here this means a ring with a unique maximal
ideal $\mathfrak{m}$ where $eRe/\mathfrak{m}$ a division ring -- then 
$e$ is called a \emph{local idempotent}.

\begin{lemma}
Any integral domain $R$ has only the trivial idempotents $0$ and $1$.  In particular, every division ring has only trivial idempotents.
\end{lemma}
\begin{proof}
Suppose $e\in R$ with $e\neq 0$ and $e^2=e=1e$.  Then by cancellation $e=1$.
\end{proof}

The integers are an integral domain which is not a division ring and they
serve as a counter-example to many conjectures about idempotents of general
rings as we will explore below.  However, the first important result is to
show the hierarchy of idempotents.

\begin{thm}
Every local ring $R$ has only trivial idempotents $0$ and $1$.
\end{thm}
\begin{proof}
Let $\mathfrak{m}$ be the unique maximal ideal of $R$.  Then $\mathfrak{m}$
is the Jacobson radical of $R$.  Now suppose $e\in \mathfrak{m}$ is an 
idempotent.  Then $1-e$ must be left invertible (following the
\PMlinkname{element characterization of Jacobson radicals}{JacobsonRadical}).  So there exists some 
$u\in R$ such that $1=u(1-e)$.  However, this produces
\[e=u(1-e)e=u(e-e^2)=u(e-e)=0.\]
Thus every non-trivial idempotent $e\in R$ lies outside $\mathfrak{m}$.
As $R/\mathfrak{m}$ is a division ring, the only idempotents are $0$ and $1$.
Thus if $e\in R$, $e\neq 0$ is an idempotent then it projects to an
idempotent of $R/\mathfrak{m}$ and as $e\notin \mathfrak{m}$ it follows
$e$ projects onto $1$ so that $e=1+z$ for some $z\in\mathfrak{m}$.  As 
$e^2=e$ we find $0=z+z^2$ (often called an anti-idempotent).  Once again 
as $z\in\mathfrak{m}$ we know there exists a $u\in R$ such that $1=u(1+z)$
and $z=u(1+z)z=u(z+z^2)=0$ so indeed $e=1$.
\end{proof}

\begin{coro}
Every division idempotent is a local idempotent, and every local idempotent
is a primitive idempotent.
\end{coro}

\begin{ex}
Let $R$ be a unital ring.  Then in $M_n(R)$ the standard idempotents 
are the matrices
\[E_{ii}=\begin{bmatrix} 
0 & \\
  & \ddots & \\
  &        & 1 \\
  &        &   & \ddots\\
  &        &    &       & 0
\end{bmatrix},\qquad 1\leq i\leq n.\]
\begin{enumerate}[(i)]
\item If $R$ has only trivial idempotents (i.e.: $0$ and $1$) then each
$E_{ii}$ is a primitive idempotent of $M_n(R)$.
\item If $R$ is a local ring then each $E_{ii}$ is a local idempotent.
\item If $R$ is a division ring then each $E_{ii}$ is a division idempotent.
\end{enumerate}
When $R=\mathbb{R}\oplus\mathbb{R}$ then (i) is not satisfied and consequently
neither are (ii) and (iii).  When $R=\mathbb{Z}$ then (i) is satisfied but not
(ii) nor (iii).  When $R=\mathbb{R}[[x]]$ -- the formal power series
ring over $\mathbb{R}$ -- then (i) and (ii) are satisfied but not (iii).
Finally when $R=\mathbb{R}$ then all three are satisfied.
\end{ex}


A consequence of the Wedderburn-Artin theorems classifies all Artinian simple 
rings as matrix rings over a division ring.  Thus the primitive idempotents of an Artinian ring are all local idempotents.  Without the Artinian assumption this may fail as we have already seen with $\mathbb{Z}$.


%%%%%
%%%%%
\end{document}

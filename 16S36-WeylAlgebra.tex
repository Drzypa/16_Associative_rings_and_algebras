\documentclass[12pt]{article}
\usepackage{pmmeta}
\pmcanonicalname{WeylAlgebra}
\pmcreated{2013-03-22 15:27:19}
\pmmodified{2013-03-22 15:27:19}
\pmowner{GrafZahl}{9234}
\pmmodifier{GrafZahl}{9234}
\pmtitle{Weyl algebra}
\pmrecord{5}{37305}
\pmprivacy{1}
\pmauthor{GrafZahl}{9234}
\pmtype{Definition}
\pmcomment{trigger rebuild}
\pmclassification{msc}{16S36}
\pmclassification{msc}{16S32}
\pmrelated{HeisenbergAlgebra}
\pmrelated{UniversalEnvelopingAlgebra}

% this is the default PlanetMath preamble.  as your knowledge
% of TeX increases, you will probably want to edit this, but
% it should be fine as is for beginners.

% almost certainly you want these
\usepackage{amssymb}
\usepackage{amsmath}
\usepackage{amsfonts}

% used for TeXing text within eps files
%\usepackage{psfrag}
% need this for including graphics (\includegraphics)
%\usepackage{graphicx}
% for neatly defining theorems and propositions
\usepackage{amsthm}
% making logically defined graphics
%%%\usepackage{xypic}

% there are many more packages, add them here as you need them

% define commands here
\newcommand{\<}{\langle}
\renewcommand{\>}{\rangle}
\newcommand{\Bigcup}{\bigcup\limits}
\newcommand{\DirectSum}{\bigoplus\limits}
\newcommand{\Prod}{\prod\limits}
\newcommand{\Sum}{\sum\limits}
\newcommand{\h}{\widehat}
\newcommand{\mbb}{\mathbb}
\newcommand{\mbf}{\mathbf}
\newcommand{\mc}{\mathcal}
\newcommand{\mmm}[9]{\left(\begin{array}{rrr}#1&#2&#3\\#4&#5&#6\\#7&#8&#9\end{array}\right)}
\newcommand{\mf}{\mathfrak}
\newcommand{\ol}{\overline}

% Math Operators/functions
\DeclareMathOperator{\Aut}{Aut}
\DeclareMathOperator{\End}{End}
\DeclareMathOperator{\Frob}{Frob}
\DeclareMathOperator{\cwe}{cwe}
\DeclareMathOperator{\id}{id}
\DeclareMathOperator{\mult}{mult}
\DeclareMathOperator{\we}{we}
\DeclareMathOperator{\wt}{wt}
\begin{document}
\subsection*{Abstract definition}

Let $F$ be a field and $V$ be an $F$-vector space with basis
$\{P_i\}_{i\in I}\cup\{Q_i\}_{i\in I}$, where $I$ is some non-empty
index set. Let $T$ be the tensor algebra of $V$ and let
$J$ be the ideal in $T$ generated by the set
$\{P_i\otimes Q_j-Q_j\otimes P_i-\delta_{ij}\}_{i,j\in I}$ where
$\delta$ is the Kronecker delta symbol. Then the quotient $T/J$ is the
\emph{$|I|$-th Weyl algebra}.

\subsection*{A more concrete definition}

If the field $F$ has characteristic zero we have the following more
concrete definition. Let $R:=F[\{X_i\}_{i\in I}]$ be the polynomial
ring over $F$ in indeterminates $X_i$ labeled by $I$. For any $i\in
I$, let $\partial_i$ denote the partial differential operator with
respect to $X_i$. Then the $|I|$-th Weyl algebra is the set $W$ of all
differential operators of the form
\begin{equation*}
D=\Sum_{|\alpha|\leq n}f_\alpha\partial^\alpha
\end{equation*}
where the summation variable $\alpha$ is a multi-index with $|I|$
entries, $n$ is the degree of $D$, and $f_\alpha\in R$. The algebra
structure is defined by the usual operator multiplication, where the
coefficients $f_\alpha\in R$ are identified with the operators of left
multiplication with them for conciseness of notation. Since the
derivative of a polynomial is again a polynomial, it is clear that $W$
is closed under that multiplication.

The equivalence of these definitions can be seen by replacing the
generators $Q_i$ with left multiplication by the indeterminates $X_i$,
the generators $P_i$ with the partial differential operator
$\partial_i$, and the tensor product with operator multiplication, and
observing that $\partial_iX_j-X_j\partial_i=\delta_{ij}$. If, however,
the characteristic $p$ of $F$ is positive, the resulting homomorphism
to $W$ is not injective, since for example the expressions
$\partial_i^p$ and $X_i^n$ commute, while $P_i^{\otimes p}$ and
$Q_i^{\otimes n}$ do not.
%%%%%
%%%%%
\end{document}

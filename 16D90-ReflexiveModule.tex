\documentclass[12pt]{article}
\usepackage{pmmeta}
\pmcanonicalname{ReflexiveModule}
\pmcreated{2013-03-22 19:22:38}
\pmmodified{2013-03-22 19:22:38}
\pmowner{CWoo}{3771}
\pmmodifier{CWoo}{3771}
\pmtitle{reflexive module}
\pmrecord{9}{42331}
\pmprivacy{1}
\pmauthor{CWoo}{3771}
\pmtype{Definition}
\pmcomment{trigger rebuild}
\pmclassification{msc}{16D90}
\pmclassification{msc}{16D80}
\pmdefines{torsionless}
\pmdefines{reflexive}

\endmetadata

\usepackage{amssymb,amscd}
\usepackage{amsmath}
\usepackage{amsfonts}
\usepackage{mathrsfs}

% used for TeXing text within eps files
%\usepackage{psfrag}
% need this for including graphics (\includegraphics)
%\usepackage{graphicx}
% for neatly defining theorems and propositions
\usepackage{amsthm}
% making logically defined graphics
%%\usepackage{xypic}
\usepackage{pst-plot}

% define commands here
\newcommand*{\abs}[1]{\left\lvert #1\right\rvert}
\newtheorem{prop}{Proposition}
\newtheorem{thm}{Theorem}
\newtheorem{ex}{Example}
\newcommand{\real}{\mathbb{R}}
\newcommand{\pdiff}[2]{\frac{\partial #1}{\partial #2}}
\newcommand{\mpdiff}[3]{\frac{\partial^#1 #2}{\partial #3^#1}}

\begin{document}
Let $R$ be a ring, and $M$ a right $R$-module.  Then its dual, $M^*$, is given by $\hom(M,R)$, and has the structure of a left module over $R$.  The dual of that, $M^{**}$, is in turn a right $R$-module.  Fix any $m\in M$.  Then for any $f\in M^*$, the mapping $$f \mapsto f(m)$$ is a left $R$-module homomorphism from $M^*$ to $R$.  In other words, the mapping is an element of $M^{**}$.  We call this mapping $\hat{m}$, since it only depends on $m$.  For any $m\in M$, the mapping $$m\mapsto \hat{m}$$ is a then a right $R$-module homomorphism from $M$ to $M^{**}$.  Let us call it $\theta$.

\textbf{Definition}.  Let $R$, $M$, and $\theta$ be given as above.  If $\theta$ is injective, we say that $M$ is \emph{torsionless}.  If $\theta$ is in addition an isomorphism, we say that $M$ is \emph{reflexive}.  A torsionless module is sometimes referred to as being semi-reflexive.

An obvious example of a reflexive module is any vector space over a field (similarly, a right vector space over a division ring).

Some of the properties of torsionless and reflexive modules are
\begin{itemize}
\item any free module is torsionless.
\item any direct sum of torsionless modules is torsionless; any submodule of a torsionless module is torsionless.
\item based on the two properties above, any projective module is torsionless.
\item $R$ is reflexive.
\item any finite direct sum of reflexive modules is reflexive; any direct summand of a reflexive module is reflexive.
\item based on the two immediately preceding properties, any finitely generated projective module is reflexive.
\end{itemize}

%%%%%
%%%%%
\end{document}

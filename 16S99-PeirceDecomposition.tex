\documentclass[12pt]{article}
\usepackage{pmmeta}
\pmcanonicalname{PeirceDecomposition}
\pmcreated{2013-03-22 14:39:17}
\pmmodified{2013-03-22 14:39:17}
\pmowner{mclase}{549}
\pmmodifier{mclase}{549}
\pmtitle{Peirce decomposition}
\pmrecord{7}{36246}
\pmprivacy{1}
\pmauthor{mclase}{549}
\pmtype{Definition}
\pmcomment{trigger rebuild}
\pmclassification{msc}{16S99}

\endmetadata

% this is the default PlanetMath preamble.  as your knowledge
% of TeX increases, you will probably want to edit this, but
% it should be fine as is for beginners.

% almost certainly you want these
\usepackage{amssymb}
\usepackage{amsmath}
\usepackage{amsfonts}

% used for TeXing text within eps files
%\usepackage{psfrag}
% need this for including graphics (\includegraphics)
%\usepackage{graphicx}
% for neatly defining theorems and propositions
%\usepackage{amsthm}
% making logically defined graphics
%%%\usepackage{xypic}

% there are many more packages, add them here as you need them

% define commands here

\newcommand{\isom}{\cong}
\begin{document}
Let $e$ be an idempotent of a ring $R$, not necessarily with an identity.
For any subset $X$ of $R$, we introduce the notations:

$$(1-e)X = \{ x - ex \mid x \in X\}$$

and

$$X(1-e) = \{x - xe \mid x \in X\}.$$

If it happens that $R$ has an identity element, then $1-e$ is a legitimate element of
$R$, and this notation agrees with the usual product of an element and a set.

It is easy to see that $Xe \cap X(1-e) = 0 = eX \cap (1-e)X$ for any set $X$ which contains $0$.

Applying this first on the right with $X = R$ and then on the left with $X = Re$ and $X = R(1-e)$,
we obtain:

$$R = eRe \oplus eR(1-e) \oplus (1-e)Re \oplus (1-e)R(1-e).$$

This is called the {\em Peirce Decompostion} of $R$ with respect to $e$.

Note that $eRe$ and $(1-e)R(1-e)$ are subrings, $eR(1-e)$ is an $eRe$-$(1-e)R(1-e)$-bimodule,
and $(1-e)Re$ is a $(1-e)R(1-e)$-$eRe$-bimodule.

This is an example of a generalized matrix ring:

$$R \isom
  \begin{pmatrix}
    eRe & eR(1-e) \\
    (1-e)Re & (1-e)R(1-e) \\
  \end{pmatrix}
$$

More generally, if $R$ has an identity element,
and $e_1, e_2, \dots, e_n$ is a complete set of orthogonal idempotents,
then

$$R \isom
  \begin{pmatrix}
    e_1Re_1 & e_1Re_2 & \dots & e_1Re_n \\
    e_2Re_1 & e_2Re_2 & \dots & e_2Re_n \\
    \vdots & \vdots & \ddots & \vdots \\
    e_nRe_1 & e_nRe_2 & \dots & e_nRe_n
  \end{pmatrix}
$$

is a generalized matrix ring.
%%%%%
%%%%%
\end{document}

\documentclass[12pt]{article}
\usepackage{pmmeta}
\pmcanonicalname{FundamentalIsomorphismTheoremForCoalgebras}
\pmcreated{2013-03-22 18:49:30}
\pmmodified{2013-03-22 18:49:30}
\pmowner{joking}{16130}
\pmmodifier{joking}{16130}
\pmtitle{fundamental isomorphism theorem for coalgebras}
\pmrecord{4}{41628}
\pmprivacy{1}
\pmauthor{joking}{16130}
\pmtype{Theorem}
\pmcomment{trigger rebuild}
\pmclassification{msc}{16W30}

\endmetadata

% this is the default PlanetMath preamble.  as your knowledge
% of TeX increases, you will probably want to edit this, but
% it should be fine as is for beginners.

% almost certainly you want these
\usepackage{amssymb}
\usepackage{amsmath}
\usepackage{amsfonts}

% used for TeXing text within eps files
%\usepackage{psfrag}
% need this for including graphics (\includegraphics)
%\usepackage{graphicx}
% for neatly defining theorems and propositions
%\usepackage{amsthm}
% making logically defined graphics
%%%\usepackage{xypic}

% there are many more packages, add them here as you need them

% define commands here

\begin{document}
Let $(C,\Delta,\varepsilon)$ and $(D,\Delta',\varepsilon')$ be coalgebras. Recall, that if $D_0\subseteq D$ is a subcoalgebra, then $(D_0,\Delta'_{| D_0},\varepsilon'_{| D_0})$ is a coalgebra. On the other hand, if $I\subseteq C$ is a coideal, then there is a canonical coalgebra structure on $C/I$ (please, see \PMlinkname{this entry}{SubcoalgebrasAndCoideals} for more details).

\textbf{Theorem.} If $f:C\to D$ is a coalgebra homomorphism, then $\mathrm{ker}(f)$ is a coideal, $\mathrm{im}(f)$ is a subcoalgebra and a mapping $f':C/\mathrm{ker}(f)\to\mathrm{im}(f)$ defined by $f'\big( c+\mathrm{ker}(f)\big) =f(c)$ is a well defined coalgebra isomorphism.
%%%%%
%%%%%
\end{document}

\documentclass[12pt]{article}
\usepackage{pmmeta}
\pmcanonicalname{LyingOverTheorem}
\pmcreated{2013-03-22 19:15:42}
\pmmodified{2013-03-22 19:15:42}
\pmowner{pahio}{2872}
\pmmodifier{pahio}{2872}
\pmtitle{Lying-Over Theorem}
\pmrecord{6}{42191}
\pmprivacy{1}
\pmauthor{pahio}{2872}
\pmtype{Theorem}
\pmcomment{trigger rebuild}
\pmclassification{msc}{16D99}
\pmclassification{msc}{13C99}
\pmdefines{lie over}

% this is the default PlanetMath preamble.  as your knowledge
% of TeX increases, you will probably want to edit this, but
% it should be fine as is for beginners.

% almost certainly you want these
\usepackage{amssymb}
\usepackage{amsmath}
\usepackage{amsfonts}

% used for TeXing text within eps files
%\usepackage{psfrag}
% need this for including graphics (\includegraphics)
%\usepackage{graphicx}
% for neatly defining theorems and propositions
 \usepackage{amsthm}
% making logically defined graphics
%%%\usepackage{xypic}

% there are many more packages, add them here as you need them

% define commands here

\theoremstyle{definition}
\newtheorem*{thmplain}{Theorem}

\begin{document}
Let $\mathfrak{o}$ be a subring of a commutative ring $\mathfrak{O}$ with nonzero unity and integral over 
$\mathfrak{o}$.\, If $\mathfrak{a}$ is an ideal of $\mathfrak{o}$ and $\mathfrak{A}$ an ideal of $\mathfrak{O}$ such that
$$\mathfrak{A\cap o \;=\; a},$$
then $\mathfrak{A}$ is said to \emph{lie over} $\mathfrak{a}$.\\


\textbf{Theorem.}\, If $\mathfrak{p}$ is a prime ideal of a ring $\mathfrak{o}$ which is a subring of a commutative ring $\mathfrak{O}$ with nonzero unity and integral over $\mathfrak{o}$, then there exists a prime ideal $\mathfrak{P}$ of 
$\mathfrak{O}$ lying over $\mathfrak{p}$.\, If the prime ideals $\mathfrak{P}$ and $\mathfrak{Q}$ both lie over 
$\mathfrak{p}$ and\, $\mathfrak{P \,\subseteq\, Q}$,\, then\, $\mathfrak{P \,=\, Q}$.



\begin{thebibliography}{8}
\bibitem{LM}{\sc M. Larsen \& P. McCarthy}: {\em Multiplicative theory of ideals}.\, Academic Press, New York (1971).
\bibitem{J}{\sc P. Jaffard}: {\em Les syst\`emes d'id\'eaux}.\, Dunod, Paris (1960).
\end{thebibliography}
%%%%%
%%%%%
\end{document}

\documentclass[12pt]{article}
\usepackage{pmmeta}
\pmcanonicalname{SurjectiveHomomorphismBetweenUnitaryRings}
\pmcreated{2013-03-22 19:10:22}
\pmmodified{2013-03-22 19:10:22}
\pmowner{pahio}{2872}
\pmmodifier{pahio}{2872}
\pmtitle{surjective homomorphism between unitary rings}
\pmrecord{6}{42080}
\pmprivacy{1}
\pmauthor{pahio}{2872}
\pmtype{Theorem}
\pmcomment{trigger rebuild}
\pmclassification{msc}{16B99}
\pmclassification{msc}{13B10}
\pmrelated{IsomorphismSwappingZeroAndUnity}

% this is the default PlanetMath preamble.  as your knowledge
% of TeX increases, you will probably want to edit this, but
% it should be fine as is for beginners.

% almost certainly you want these
\usepackage{amssymb}
\usepackage{amsmath}
\usepackage{amsfonts}

% used for TeXing text within eps files
%\usepackage{psfrag}
% need this for including graphics (\includegraphics)
%\usepackage{graphicx}
% for neatly defining theorems and propositions
 \usepackage{amsthm}
% making logically defined graphics
%%%\usepackage{xypic}

% there are many more packages, add them here as you need them

% define commands here

\theoremstyle{definition}
\newtheorem*{thmplain}{Theorem}

\begin{document}
\textbf{Theorem.}\, Let $f$ be a surjective homomorphism from a unitary ring $R$ to another unitary ring $R\,'$.\, Then
\begin{itemize}
\item $f(1) \;=\; 1',$
\item $f(a^{-1}) \;=\; (f(a))^{-1}$\; for all elements $a$ belonging to the group of units of $R$.
\end{itemize}



\emph{Proof.}\, $1^\circ$.\, In a ring, the \PMlinkescapetext{multiplicative} identity element is unique, whence it suffices to show that $f(1)$ has the properties required for the unity of the ring $R\,'$.\, When $a'$ is an arbitrary element of this ring, there is by the surjectivity an element $a$ of $R$ such that\, $f(a) = a'$.\, Thus we have
$$f(1)a' \;=\; f(1)f(a) \;=\; f(1a) \;=\; f(a) \;=\; a', \quad 
  a'f(1) \;=\; f(a)f(1) \;=\; f(a1) \;=\; f(a) \;=\; a'.$$
$2^\circ$.\, Let $a$ be a unit of $R$.\, Then
$$f(a)f(a^{-1}) \;=\; f(aa^{-1}) \;=\; f(1) \;=\; 1', \quad f(a^{-1})f(a) \;=\; f(a^{-1}a) \;=\; f(1) \;=\; 1',$$
whence $f(a^{-1})$ is a multiplicative inverse of $f(a)$.

%%%%%
%%%%%
\end{document}

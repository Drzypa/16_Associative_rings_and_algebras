\documentclass[12pt]{article}
\usepackage{pmmeta}
\pmcanonicalname{AdditiveInverseOfTheZeroInARing}
\pmcreated{2013-03-22 15:45:13}
\pmmodified{2013-03-22 15:45:13}
\pmowner{aplant}{12431}
\pmmodifier{aplant}{12431}
\pmtitle{additive inverse of the zero in a ring}
\pmrecord{9}{37707}
\pmprivacy{1}
\pmauthor{aplant}{12431}
\pmtype{Definition}
\pmcomment{trigger rebuild}
\pmclassification{msc}{16B70}

\endmetadata

% this is the default PlanetMath preamble.  as your knowledge
% of TeX increases, you will probably want to edit this, but
% it should be fine as is for beginners.

% almost certainly you want these
\usepackage{amssymb}
\usepackage{amsmath}
\usepackage{amsfonts}

% used for TeXing text within eps files
%\usepackage{psfrag}
% need this for including graphics (\includegraphics)
%\usepackage{graphicx}
% for neatly defining theorems and propositions
%\usepackage{amsthm}
% making logically defined graphics
%%%\usepackage{xypic}

% there are many more packages, add them here as you need them

% define commands here
\begin{document}
In any ring $R$, the additive identity is unique and usually denoted by $0$.  It is called the zero or {\em neutral element} of the ring and it satisfies the zero property under multiplication.  The additive inverse of the zero must be zero itself.  For suppose otherwise: that there is some nonzero $c \in R$ so that $0 + c = 0$.  For any element $a \in R$ we have $a + 0 = a$ since $0$ is the additive identity.  Now, because addition is associative we have
\begin{eqnarray*}
0 & = & a + 0 \\ 
 & = & a + (0 + c) \\
 & = & (a + 0) + c \\
 & = & a + c.
\end{eqnarray*}
Since $a$ is any arbitrary element in the ring, this would imply that (nonzero) $c$ is an additive identity, contradicting the uniqueness of the additive identity.  And so our suppostition that $0$ has a nonzero inverse cannot be true.  So the additive inverse of the zero is zero itself.  We can write this as $-0 = 0$, where the $-$ sign means ``additive inverse".

Yes, for sure, there are other ways to come to this result, and we encourage you to have a bit of fun describing your own reasons for why the additive inverse of the zero of the ring must be zero itself.

For example, since $0$ is the neutral element of the ring this means that $0 + 0 = 0$.  From this it immediately follows that $-0 = 0$.
%%%%%
%%%%%
\end{document}

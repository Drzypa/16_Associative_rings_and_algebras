\documentclass[12pt]{article}
\usepackage{pmmeta}
\pmcanonicalname{ExampleOfKlein4ring}
\pmcreated{2013-03-22 17:41:59}
\pmmodified{2013-03-22 17:41:59}
\pmowner{Algeboy}{12884}
\pmmodifier{Algeboy}{12884}
\pmtitle{example of Klein 4-ring}
\pmrecord{9}{40142}
\pmprivacy{1}
\pmauthor{Algeboy}{12884}
\pmtype{Example}
\pmcomment{trigger rebuild}
\pmclassification{msc}{16B99}
\pmclassification{msc}{20-00}
\pmrelated{NonCommutativeRingsOfOrderFour}

\usepackage{latexsym}
\usepackage{amssymb}
\usepackage{amsmath}
\usepackage{amsfonts}
\usepackage{amsthm}

%%\usepackage{xypic}

%-----------------------------------------------------

%       Standard theoremlike environments.

%       Stolen directly from AMSLaTeX sample

%-----------------------------------------------------

%% \theoremstyle{plain} %% This is the default

\newtheorem{thm}{Theorem}

\newtheorem{coro}[thm]{Corollary}

\newtheorem{lem}[thm]{Lemma}

\newtheorem{lemma}[thm]{Lemma}

\newtheorem{prop}[thm]{Proposition}

\newtheorem{conjecture}[thm]{Conjecture}

\newtheorem{conj}[thm]{Conjecture}

\newtheorem{defn}[thm]{Definition}

\newtheorem{remark}[thm]{Remark}

\newtheorem{ex}[thm]{Example}



%\countstyle[equation]{thm}



%--------------------------------------------------

%       Item references.

%--------------------------------------------------


\newcommand{\exref}[1]{Example-\ref{#1}}

\newcommand{\thmref}[1]{Theorem-\ref{#1}}

\newcommand{\defref}[1]{Definition-\ref{#1}}

\newcommand{\eqnref}[1]{(\ref{#1})}

\newcommand{\secref}[1]{Section-\ref{#1}}

\newcommand{\lemref}[1]{Lemma-\ref{#1}}

\newcommand{\propref}[1]{Prop\-o\-si\-tion-\ref{#1}}

\newcommand{\corref}[1]{Cor\-ol\-lary-\ref{#1}}

\newcommand{\figref}[1]{Fig\-ure-\ref{#1}}

\newcommand{\conjref}[1]{Conjecture-\ref{#1}}


% Normal subgroup or equal.

\providecommand{\normaleq}{\unlhd}

% Normal subgroup.

\providecommand{\normal}{\lhd}

\providecommand{\rnormal}{\rhd}
% Divides, does not divide.

\providecommand{\divides}{\mid}

\providecommand{\ndivides}{\nmid}


\providecommand{\union}{\cup}

\providecommand{\bigunion}{\bigcup}

\providecommand{\intersect}{\cap}

\providecommand{\bigintersect}{\bigcap}










\begin{document}
The Klein $4$-ring $K$ can be represented by as a left ideal of $2\times 2$-matrices
over the field with two elements $\mathbb{Z}_2$.  Doing so helps to explain
some of the unnatural properties of this nonunital ring and is an example of how 
many nonunital rings can often be understood as very natural subobjects of unital 
rings.

\begin{equation}
K=\left\{ \begin{bmatrix} x & 0\\ y & 0\end{bmatrix} : x,y\in\mathbb{Z}_2\right\}
\end{equation}
To match the product with with the table, use
\begin{align*}
a & := \begin{bmatrix} 1 & 0 \\ 0 & 0 \end{bmatrix},\\
b & := \begin{bmatrix} 0 & 0\\ 1 & 0 \end{bmatrix},\\
c & := \begin{bmatrix} 1 & 0 \\ 1 & 0 \end{bmatrix}.
\end{align*}

Here the properties of the abstract multiplication table of $K$ (given 
\PMlinkname{here}{Klein4Ring}) can be
seen as rather natural properties of unital rings.  That is, the elements
$a$ and $c$ are idempotents in the ring $M_2(\mathbb{Z}_2)$ and so they
behave similar to identities, and $b$ is nilpotent so that its annihilating
property is expected as well.

The second noncommutative nonunital ring of order 4 is the transpose of these matrices,
that is, a right ideal of $M_2(\mathbb{Z}_2)$.

Viewed in this way we recognize the Klein $4$-ring as part of an infinite family
of similar nonunital rings of left/right ideals of a unital ring.  Some authors
prefer to treat such objects only as ideals and not as rings so that the properties 
are always given the background of a more familiar structure.
%%%%%
%%%%%
\end{document}

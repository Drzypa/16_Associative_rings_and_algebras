\documentclass[12pt]{article}
\usepackage{pmmeta}
\pmcanonicalname{ZeroModule}
\pmcreated{2013-03-22 12:01:42}
\pmmodified{2013-03-22 12:01:42}
\pmowner{antizeus}{11}
\pmmodifier{antizeus}{11}
\pmtitle{zero module}
\pmrecord{6}{31005}
\pmprivacy{1}
\pmauthor{antizeus}{11}
\pmtype{Definition}
\pmcomment{trigger rebuild}
\pmclassification{msc}{16D10}
\pmsynonym{zero submodule}{ZeroModule}
\pmrelated{ZeroIdeal}

\endmetadata

\usepackage{amssymb}
\usepackage{amsmath}
\usepackage{amsfonts}
\usepackage{graphicx}
%%%\usepackage{xypic}
\begin{document}
Let $R$ be a ring.

The abelian group which contains only an identity element (zero) 
gains a trivial $R$-module structure, 
which we call the {\PMlinkescapetext {\it zero module}}.

Every $R$-module $M$ has an zero element 
and thus a submodule consisting of that element.
This is called the {\it zero submodule} of $M$.
%%%%%
%%%%%
%%%%%
\end{document}

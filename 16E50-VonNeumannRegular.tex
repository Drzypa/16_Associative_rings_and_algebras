\documentclass[12pt]{article}
\usepackage{pmmeta}
\pmcanonicalname{VonNeumannRegular}
\pmcreated{2013-03-22 12:56:18}
\pmmodified{2013-03-22 12:56:18}
\pmowner{CWoo}{3771}
\pmmodifier{CWoo}{3771}
\pmtitle{von Neumann regular}
\pmrecord{13}{33295}
\pmprivacy{1}
\pmauthor{CWoo}{3771}
\pmtype{Definition}
\pmcomment{trigger rebuild}
\pmclassification{msc}{16E50}
\pmdefines{von Neumann regular ring}
\pmdefines{regular ring}
\pmdefines{pseudoinverse}

% this is the default PlanetMath preamble.  as your knowledge
% of TeX increases, you will probably want to edit this, but
% it should be fine as is for beginners.

% almost certainly you want these
\usepackage{amssymb}
\usepackage{amsmath}
\usepackage{amsfonts}

% used for TeXing text within eps files
%\usepackage{psfrag}
% need this for including graphics (\includegraphics)
%\usepackage{graphicx}
% for neatly defining theorems and propositions
%\usepackage{amsthm}
% making logically defined graphics
%%%\usepackage{xypic}

% there are many more packages, add them here as you need them

% define commands here
\def\sse{\subseteq}
\def\bigtimes{\mathop{\mbox{\Huge $\times$}}}
\def\impl{\Rightarrow}
\begin{document}
An element $a$ of a ring $R$ is said to be \emph{von Neumann regular} if there
exists $b\in R$ such that $aba=a$. Such an element $b$ is known as a \emph{\PMlinkescapetext{pseudoinverse}} of $a$.  

For example, any unit in a ring is von Neumann regular.  Also, any idempotent element is von Neumann regular.  For a non-unit, non-idempotent von Nuemann regular element, take $M_2(\mathbb{R})$, the ring of $2\times 2$ matrices over $\mathbb{R}$.  Then  
\begin{center}
$\begin{pmatrix}
2 & 0 \\
0 & 0
\end{pmatrix}=
\begin{pmatrix}
2 & 0 \\
0 & 0
\end{pmatrix}
\begin{pmatrix}
\frac{1}{2} & 0 \\
0 & 0
\end{pmatrix}
\begin{pmatrix}
2 & 0 \\
0 & 0
\end{pmatrix}$
\end{center}
is von Neumann regular.  In fact, we can replace $2$ with any non-zero $r\in \mathbb{R}$ and the resulting matrix is also von Neumann regular.  There are several ways to generalize this example.  One way is take a central idempotent $e$ in any ring $R$, and any $rs=f$ with $ef=e$.  Then $re$ is von Neumann regular, with $s,se$ and $sf$ all as pseudoinverses.  In another generalization, we have two rings $R,S$ where $R$ is an algebra over $S$.  Take any idempotent $e\in R$, and any invertible element $s\in S$ such that $s$ commutes with $e$.  Then $se$ is von Neumann regular.

A ring $R$ is said to be a \emph{von Neumann regular ring} (or simply
a \emph{regular ring}, if the \PMlinkescapetext{meaning} is clear from context)
if every element of $R$ is von Neumann regular.

For example, any division ring is von Neumann regular, and so is any ring of matrices over a division ring.  In general, any semisimple ring is von Neumann regular.

\textbf{Remark}.  Note that \emph{regular ring} in the sense of von Neumann should not be confused with \emph{regular ring} in the sense of \PMlinkescapetext{commutative algebra}, which is a Noetherian ring whose localization at every prime ideal is a regular local ring.
%%%%%
%%%%%
\end{document}

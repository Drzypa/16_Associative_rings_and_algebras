\documentclass[12pt]{article}
\usepackage{pmmeta}
\pmcanonicalname{ExampleOfReducibleAndIrreducibleGmodules}
\pmcreated{2013-03-22 16:37:50}
\pmmodified{2013-03-22 16:37:50}
\pmowner{rm50}{10146}
\pmmodifier{rm50}{10146}
\pmtitle{example of reducible and irreducible $G$-modules}
\pmrecord{6}{38831}
\pmprivacy{1}
\pmauthor{rm50}{10146}
\pmtype{Example}
\pmcomment{trigger rebuild}
\pmclassification{msc}{16D60}
\pmdefines{augmentation}

% this is the default PlanetMath preamble.  as your knowledge
% of TeX increases, you will probably want to edit this, but
% it should be fine as is for beginners.

% almost certainly you want these
\usepackage{amssymb}
\usepackage{amsmath}
\usepackage{amsfonts}

% used for TeXing text within eps files
%\usepackage{psfrag}
% need this for including graphics (\includegraphics)
%\usepackage{graphicx}
% for neatly defining theorems and propositions
%\usepackage{amsthm}
% making logically defined graphics
%%%\usepackage{xypic}

% there are many more packages, add them here as you need them

% define commands here

\begin{document}
\PMlinkescapeword{stable}
Let $G=S_r$, the permutation group on $r$ elements, and $N=k^r$ where $k$ is an arbitrary field. Consider the permutation representation of $G$ on $N$ given by
\[\sigma(a_1,\ldots,a_r)=(a_{\sigma(1)},\ldots,a_{\sigma(r)}),\ \sigma\in S_r, a_i\in k\]

If $r>1$, we can define two submodules of $N$, called the \emph{trace} and \emph{augmentation}, as
\begin{gather*}
N'=\{(a,a,\ldots,a)\}\\
N''=\{(a_1,a_2,\ldots,a_r)\ \bigl\rvert\ \sum a_i=0 \bigr .\}
\end{gather*}
Clearly both $N'$ and $N''$ are stable under the action of $G$ and thus in fact form submodules of $N$.

If the characteristic of $k$ divides $r$, then obviously $N''\supset N'$. Otherwise, $N''$ is a simple (irreducible) $G$-module. For suppose $N''$ has a nontrivial submodule $M$, and choose a nonzero $u\in M$. Then some pair of coordinates of $u$ are unequal, for if not, then $u=(a,\ldots,a)$ and then $u\not\in N''$ because of the restriction on the characteristic of $k$ forces $ra\neq 0$. So apply a suitable element of $G$ to get another element of $M$, $v=(b_1,b_2,\ldots,b_r)$ where $b_1\neq b_2$ (note here that we use the fact that $M$ is a submodule and thus is stable under the action of $G$).

But now $(1 2)v - ev = (b_1-b_2,b_2-b_1,0,\ldots,0)$ is also in $M$, so $w=(1,-1,0,\ldots,0)\in M$. It is obvious that by multiplying $w$ by elements of $k$ and by permuting, we can obtain any element of $N''$ and thus $M=N''$. Thus $N''$ is simple.

It is also obvious that $N=N'\oplus N''$.
%%%%%
%%%%%
\end{document}

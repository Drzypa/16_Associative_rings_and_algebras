\documentclass[12pt]{article}
\usepackage{pmmeta}
\pmcanonicalname{ZeroRing}
\pmcreated{2013-03-22 13:30:19}
\pmmodified{2013-03-22 13:30:19}
\pmowner{Wkbj79}{1863}
\pmmodifier{Wkbj79}{1863}
\pmtitle{zero ring}
\pmrecord{26}{34086}
\pmprivacy{1}
\pmauthor{Wkbj79}{1863}
\pmtype{Definition}
\pmcomment{trigger rebuild}
\pmclassification{msc}{16U99}
\pmclassification{msc}{13M05}
\pmclassification{msc}{13A99}
\pmrelated{ZeroVectorSpace}
\pmrelated{Unity}

% this is the default PlanetMath preamble.  as your knowledge
% of TeX increases, you will probably want to edit this, but
% it should be fine as is for beginners.

% almost certainly you want these
\usepackage{amssymb}
\usepackage{amsmath}
\usepackage{amsfonts}

% used for TeXing text within eps files
%\usepackage{psfrag}
% need this for including graphics (\includegraphics)
%\usepackage{graphicx}
% for neatly defining theorems and propositions
%\usepackage{amsthm}
% making logically defined graphics
%%%\usepackage{xypic}

% there are many more packages, add them here as you need them

% define commands here
\begin{document}
A ring is a {\sl zero ring\/} if the product of any two elements is the additive identity (or zero).

Zero rings are commutative under multiplication.  For if $Z$ is a zero ring, 
$0_Z$ is its additive identity, and $x,y \in Z$, then $xy=0_Z=yx.$

Every zero ring is a nilpotent ring.  For if $Z$ is a zero ring, then $Z^2=\{0_Z\}$.

Since every subring of a ring must contain its zero element, every subring of a ring is an ideal, and a zero ring has no prime ideals.

The simplest zero ring is ${\mathbb Z}_1=\{0\}$.  Up to \PMlinkname{isomorphism}{RingIsomorphism}, this is the only zero ring that has a multiplicative identity.

Zero rings exist in \PMlinkescapetext{abundance}.  They can be constructed from any ring.  If $R$ is a ring, then

$$\left\{ \left. \left( \begin{array}{cc}
r & -r \\
r & -r \end{array} \right) \right| r \in R \right\}$$

considered as a subring of ${\mathbf M}_{2\operatorname{x}2}(R)$ (with standard matrix addition and multiplication) is a zero ring.  Moreover, the cardinality of this subset of ${\mathbf M}_{2\operatorname{x}2}(R)$ is the same as that of $R$.

Moreover, zero rings can be constructed from any abelian group.  If $G$ is a group with identity $e_G$, it can be made into a zero ring by declaring its addition to be its group operation and defining its multiplication by $a \cdot b=e_G$ for any $a,b \in G$.

Every finite zero ring can be written as a direct product of cyclic rings, which must also be zero rings themselves.  This follows from the \PMlinkname{fundamental theorem of finite abelian groups}{FundamentalTheoremOfFinitelyGeneratedAbelianGroups}.  Thus, if $p_1, \ldots , p_m$ are distinct primes, $a_1, \ldots , a_m$ are positive integers, and $\displaystyle n= \prod_{j=1}^m {p_j}^{a_j}$, then the number of zero rings of \PMlinkname{order}{Order} $n$ is $\displaystyle \prod_{j=1}^m p(a_j)$, where $p$ denotes the \PMlinkname{partition function}{PartitionFunction2}.
%%%%%
%%%%%
\end{document}

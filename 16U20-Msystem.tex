\documentclass[12pt]{article}
\usepackage{pmmeta}
\pmcanonicalname{Msystem}
\pmcreated{2013-03-22 17:29:09}
\pmmodified{2013-03-22 17:29:09}
\pmowner{CWoo}{3771}
\pmmodifier{CWoo}{3771}
\pmtitle{$m$-system}
\pmrecord{11}{39873}
\pmprivacy{1}
\pmauthor{CWoo}{3771}
\pmtype{Definition}
\pmcomment{trigger rebuild}
\pmclassification{msc}{16U20}
\pmclassification{msc}{13B30}
\pmsynonym{m-system}{Msystem}
\pmrelated{MultiplicativelyClosed}
\pmrelated{NSystem}
\pmrelated{PrimeIdeal}

\usepackage{amssymb,amscd}
\usepackage{amsmath}
\usepackage{amsfonts}
\usepackage{mathrsfs}

% used for TeXing text within eps files
%\usepackage{psfrag}
% need this for including graphics (\includegraphics)
%\usepackage{graphicx}
% for neatly defining theorems and propositions
\usepackage{amsthm}
% making logically defined graphics
%%\usepackage{xypic}
\usepackage{pst-plot}
\usepackage{psfrag}

% define commands here
\newtheorem{prop}{Proposition}
\newtheorem{thm}{Theorem}
\newtheorem{ex}{Example}
\newcommand{\real}{\mathbb{R}}
\newcommand{\pdiff}[2]{\frac{\partial #1}{\partial #2}}
\newcommand{\mpdiff}[3]{\frac{\partial^#1 #2}{\partial #3^#1}}
\begin{document}
\PMlinkescapeword{closed subset}

Let $R$ be a ring.  A subset $S$ of $R$ is called an $m$-{\em system} if 
\begin{itemize}
\item $S\ne \varnothing$, and
\item for every two elements $x,\,y\in S$, there is an element $r\in R$ such that $xry\in S$.
\end{itemize}

$m$-Systems are a generalization of multiplicatively closet subsets in a ring.  Indeed, every multiplicatively closed subset of $R$ is an $m$-system: any $x,y\in S$, then $xy\in S$, hence $xyy \in S$.  However, the converse is not true.  For example, the set $$\lbrace r^n\mid r\in R \mbox{ and } n \mbox{ is an odd positive integer}\rbrace$$ is an $m$-system, but not multiplicatively closed in general (unless, for example, if $r=1$).

\textbf{Remarks}.  $m$-Systems and prime ideals of a ring are intimately related.  Two basic relationships between the two notions are
\begin{enumerate}
\item An ideal $P$ in a ring $R$ is a prime ideal iff $R-P$ is an $m$-system.
\begin{proof}  $P$ is prime iff $xRy\subseteq P$ implies $x$ or $y\in P$, iff $x,y\in R-P$ implies that there is $r\in R$ with $xry\notin P$ iff $R-P$ is an $m$-system.
\end{proof}
\item Given an $m$-system $S$ of $R$ and an ideal $I$ with $I\cap S=\varnothing$.  Then there exists a prime ideal $P\subseteq R$ with the property that $P$ contains $I$ and $P\cap S = \varnothing$, and $P$ is the largest among all ideals with this property. 
\begin{proof}  Let $\mathcal{C}$ be the collection of all ideals containing $I$ and disjoint from $S$.  First, $I\in \mathcal{C}$.  Second, any chain $K$ of ideals in $\mathcal{C}$, its union $\bigcup K$ is also in $\mathcal{C}$.  So Zorn's lemma applies.  Let $P$ be a maximal element in $\mathcal{C}$.  We want to show that $P$ is prime.  Suppose otherwise.  In other words, $aRb\subseteq P$ with $a,b\notin P$.  Then $\langle P,a\rangle$ and $\langle P,b\rangle$ both have non-empty intersections with $S$. Let $$c=p+fag \in \langle P,a\rangle \cap S\quad \mbox{ and }\quad d=q+hbk \in \langle P,b\rangle \cap S,$$ where $p,q\in P$ and $f,g,h,k\in R$. Then there is $r\in R$ such that $crd\in S$. But this implies that $$crd = (p+fag)r(q+hbk)= p(rq+rhbk)+(fagr)q+f\big(a(grh)b\big)k \in P$$ as well, contradicting $P\cap S=\varnothing$.  Therefore, $P$ is prime.
\end{proof}
\end{enumerate}
$m$-Systems are also used to define the non-commutative version of the radical of an ideal of a ring.
%%%%%
%%%%%
\end{document}

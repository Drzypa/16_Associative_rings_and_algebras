\documentclass[12pt]{article}
\usepackage{pmmeta}
\pmcanonicalname{WelldefinednessOfProductOfFinitelyGeneratedIdeals}
\pmcreated{2013-03-22 19:12:56}
\pmmodified{2013-03-22 19:12:56}
\pmowner{pahio}{2872}
\pmmodifier{pahio}{2872}
\pmtitle{well-definedness of product of finitely generated ideals}
\pmrecord{6}{42135}
\pmprivacy{1}
\pmauthor{pahio}{2872}
\pmtype{Theorem}
\pmcomment{trigger rebuild}
\pmclassification{msc}{16D25}
\pmrelated{WellDefined}
\pmrelated{ProductOfIdeals}
\pmrelated{ProductOfFinitelyGeneratedIdeals}

\endmetadata

% this is the default PlanetMath preamble.  as your knowledge
% of TeX increases, you will probably want to edit this, but
% it should be fine as is for beginners.

% almost certainly you want these
\usepackage{amssymb}
\usepackage{amsmath}
\usepackage{amsfonts}

% used for TeXing text within eps files
%\usepackage{psfrag}
% need this for including graphics (\includegraphics)
%\usepackage{graphicx}
% for neatly defining theorems and propositions
 \usepackage{amsthm}
% making logically defined graphics
%%%\usepackage{xypic}

% there are many more packages, add them here as you need them

% define commands here

\theoremstyle{definition}
\newtheorem*{thmplain}{Theorem}

\begin{document}
Ler $R$ be of a commutative ring with nonzero unity.\, If
\begin{align}
\mathfrak{a} \;=\; (a_1,\,\ldots,\,a_m) \;=\; (\alpha_1,\,\ldots,\,\alpha_\mu)
\end{align}
and
\begin{align}
\mathfrak{b} \;=\; (b_1,\,\ldots,\,b_n) \;=\, (\beta_1,\,\ldots,\,\beta_\nu)
\end{align}
are two finitely generated ideals of $R$, both with two \PMlinkescapetext{generating systems}, then the ideals
$$\mathfrak{c} \;:=\; (a_1b_1,\,\ldots,\,a_ib_j,\,\ldots,\,a_mb_n)$$
and
$$\mathfrak{d} \;:=\; (\alpha_1\beta_1,\,\ldots,\,\alpha_i\beta_j,\,\ldots,\,\alpha_\mu\beta_\nu)$$
are equal.\\


\emph{Proof.}\, By (1) and (2), for every $i,\,j$, there are elements $r_{ik},\,s_{jl}$ of $R$ such that
\begin{align}
a_i \;=\; r_{i1}\alpha_1\!+\ldots+r_{i\mu}\alpha_\mu, \quad b_j \;=\; s_{j1}\beta_1\!+\ldots+s_{j\nu}\beta_\nu.
\end{align}
Multiplying the equations (3) we see that
$$a_ib_j \;=\; 
(r_{i1}s_{j1})(\alpha_1\beta_1)\!+\!(r_{i2}s_{j1})(\alpha_2\beta_1)\!+\ldots+\!(r_{i\mu}s_{j\nu})(\alpha_\mu\beta_\nu),$$
whence the generators $a_ib_j$ of $\mathfrak{c}$ belong to $\mathfrak{d}$ and consecuently\, 
$\mathfrak{c} \subseteq \mathfrak{d}$.\, The reverse containment is seen similarly.

%%%%%
%%%%%
\end{document}

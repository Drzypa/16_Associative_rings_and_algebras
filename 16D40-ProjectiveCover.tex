\documentclass[12pt]{article}
\usepackage{pmmeta}
\pmcanonicalname{ProjectiveCover}
\pmcreated{2013-03-22 12:10:08}
\pmmodified{2013-03-22 12:10:08}
\pmowner{antizeus}{11}
\pmmodifier{antizeus}{11}
\pmtitle{projective cover}
\pmrecord{6}{31380}
\pmprivacy{1}
\pmauthor{antizeus}{11}
\pmtype{Definition}
\pmcomment{trigger rebuild}
\pmclassification{msc}{16D40}

\endmetadata

\usepackage{amssymb}
\usepackage{amsmath}
\usepackage{amsfonts}
\usepackage{graphicx}
%%%\usepackage{xypic}
\begin{document}
Let $X$ and $P$ be modules.
We say that $P$ is a {\it projective cover} of $X$
if $P$ is a projective module
and there exists an epimorphism $p \colon P \to X$
such that $\operatorname{ker} p$ is a superfluous submodule of $P$.

Equivalently, $P$ is an projective cover of $X$
if $P$ is projective,
and there is an epimorphism $p \colon P \to X$,
and if $g \colon P' \to X$ is an epimorphism
from a projective module $P'$ to $X$,
then there exists an epimorphism $h \colon P' \to P$
such that $ph = g$.
$$
\xymatrix{
  &
  P'
        \ar[d]^g
        \ar@{-->}[dl]_h
  \\
  P
        \ar[r]_p
  &
  X
        \ar[r]
        \ar[d]
  &
  0
  \\
  &
  0
}
$$
%%%%%
%%%%%
%%%%%
\end{document}

\documentclass[12pt]{article}
\usepackage{pmmeta}
\pmcanonicalname{PrimeRing}
\pmcreated{2013-03-22 11:51:05}
\pmmodified{2013-03-22 11:51:05}
\pmowner{Wkbj79}{1863}
\pmmodifier{Wkbj79}{1863}
\pmtitle{prime ring}
\pmrecord{10}{30413}
\pmprivacy{1}
\pmauthor{Wkbj79}{1863}
\pmtype{Definition}
\pmcomment{trigger rebuild}
\pmclassification{msc}{16U99}
\pmclassification{msc}{16N60}
\pmrelated{ZeroIdeal}

\endmetadata

\usepackage{amssymb}
\usepackage{amsmath}
\usepackage{amsfonts}
\usepackage{graphicx}
%%%%\usepackage{xypic}
\begin{document}
A ring $R$ is said to be a \emph{prime ring} if the zero ideal is a prime ideal.

If a prime ring $R$ is commutative, then it is a cancellation ring.  If in \PMlinkescapetext{addition} $R$ has a multiplicative identity $1 \neq 0$, then it is an integral domain.
%%%%%
%%%%%
%%%%%
%%%%%
\end{document}

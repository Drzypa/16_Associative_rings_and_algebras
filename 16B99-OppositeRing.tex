\documentclass[12pt]{article}
\usepackage{pmmeta}
\pmcanonicalname{OppositeRing}
\pmcreated{2013-03-22 11:51:14}
\pmmodified{2013-03-22 11:51:14}
\pmowner{antizeus}{11}
\pmmodifier{antizeus}{11}
\pmtitle{opposite ring}
\pmrecord{7}{30416}
\pmprivacy{1}
\pmauthor{antizeus}{11}
\pmtype{Definition}
\pmcomment{trigger rebuild}
\pmclassification{msc}{16B99}
\pmclassification{msc}{17A01}
\pmrelated{DualCategory}
\pmrelated{NonCommutativeRingsOfOrderFour}

\endmetadata

\usepackage{amssymb}
\usepackage{amsmath}
\usepackage{amsfonts}
\usepackage{graphicx}
%%%%\usepackage{xypic}
\begin{document}
If $R$ is a ring, then we may construct the {\it opposite ring} $R^{op}$ which has the same underlying abelian group structure, but with multiplication in the opposite order: the product of $r_1$ and $r_2$ in $R^{op}$ is $r_2 r_1$.
\par
If $M$ is a left $R$-module, then it can be made into a right $R^{op}$-module, where a module element $m$, when multiplied on the right by an element $r$ of $R^{op}$, yields the $rm$ that we have with our left $R$-module action on $M$.  Similarly, right $R$-modules can be made into left $R^{op}$-modules.
\par
If $R$ is a commutative ring, then it is equal to its own opposite ring.
\par
Similar constructions occur in the opposite group and opposite category.
%%%%%
%%%%%
%%%%%
%%%%%
\end{document}

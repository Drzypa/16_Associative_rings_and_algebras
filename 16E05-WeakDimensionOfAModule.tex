\documentclass[12pt]{article}
\usepackage{pmmeta}
\pmcanonicalname{WeakDimensionOfAModule}
\pmcreated{2013-03-22 19:18:40}
\pmmodified{2013-03-22 19:18:40}
\pmowner{joking}{16130}
\pmmodifier{joking}{16130}
\pmtitle{weak dimension of a module}
\pmrecord{4}{42248}
\pmprivacy{1}
\pmauthor{joking}{16130}
\pmtype{Derivation}
\pmcomment{trigger rebuild}
\pmclassification{msc}{16E05}

\endmetadata

% this is the default PlanetMath preamble.  as your knowledge
% of TeX increases, you will probably want to edit this, but
% it should be fine as is for beginners.

% almost certainly you want these
\usepackage{amssymb}
\usepackage{amsmath}
\usepackage{amsfonts}

% used for TeXing text within eps files
%\usepackage{psfrag}
% need this for including graphics (\includegraphics)
%\usepackage{graphicx}
% for neatly defining theorems and propositions
%\usepackage{amsthm}
% making logically defined graphics
%%\usepackage{xypic}

% there are many more packages, add them here as you need them

% define commands here

\begin{document}
Assume that $R$ is a ring. We will consider right $R$-modules.

\textbf{Definition 1.} We will say that an $R$-module $M$ is of \textbf{weak dimension} at most $n\in\mathbb{N}$ iff there exists a short exact sequence
$$\xymatrix{
0\ar[r] & F_n\ar[r] & F_{n-1}\ar[r] & \cdots\ar[r] & F_1\ar[r] & F_0\ar[r] & M\ar[r] & 0
}$$
such that each $F_i$ is a flat module. In this case we write $\mathrm{wd}_{R}M\leqslant n$ (also we say that $M$ is of finite weak dimension). If such short exact sequence does not exist, then the weak dimension is defined as infinity, $\mathrm{wd}_R M=\infty$.

\textbf{Definition 2.} We will say that an $R$-module $M$ is of weak dimension $n\in\mathbb{N}$ iff $\mathrm{wd}_{R}M\leqslant n$ but $\mathrm{wd}_R M\not\leqslant n-1$.

The weak dimension measures how far an $R$-module is from being flat. Let as gather some known facts about the weak dimension:

\textbf{Proposition 1.} Assume that $M$ is a right $R$-module. Then $\mathrm{wd}_RM=n$ for some $n\in\mathbb{N}$ if and only if for any left $R$-module $N$ we have
$$\mathrm{Tor}_{n+1}^R(M,N)=0$$
and there exists a left $R$-module $N'$ such that
$$\mathrm{Tor}_{n}^R(M,N')\neq 0,$$
where $\mathrm{Tor}$ denotes the Tor functor.

Since every projective module is flat, then we can state simple observation:

\textbf{Proposition 2.} Assume that $M$ is a right $R$-module. Then
$$\mathrm{wd}_RM\leqslant\mathrm{pd}_RM,$$
where $\mathrm{pd}_RM$ denotes the projective dimension of $M$.

Generally these two dimension may differ.
%%%%%
%%%%%
\end{document}

\documentclass[12pt]{article}
\usepackage{pmmeta}
\pmcanonicalname{ExampleOfInjectiveModule}
\pmcreated{2013-03-22 17:43:40}
\pmmodified{2013-03-22 17:43:40}
\pmowner{Glotzfrosch}{19314}
\pmmodifier{Glotzfrosch}{19314}
\pmtitle{example of injective module}
\pmrecord{5}{40175}
\pmprivacy{1}
\pmauthor{Glotzfrosch}{19314}
\pmtype{Example}
\pmcomment{trigger rebuild}
\pmclassification{msc}{16D50}

% this is the default PlanetMath preamble.  as your knowledge
% of TeX increases, you will probably want to edit this, but
% it should be fine as is for beginners.

% almost certainly you want these
\usepackage{amssymb}
\usepackage{amsmath}
\usepackage{amsfonts}
\usepackage{amsthm}

% used for TeXing text within eps files
%\usepackage{psfrag}
% need this for including graphics (\includegraphics)
%\usepackage{graphicx}
% for neatly defining theorems and propositions
%\usepackage{amsthm}
% making logically defined graphics
%%%\usepackage{xypic}

% there are many more packages, add them here as you need them

% define commands here

\begin{document}
In the category of unitary $\mathbb{Z}$-modules (which is the category of Abelian groups), every divisible Group is injective, i.e. every Group $G$ such that for any $g \in G$ and $n \in \mathbb{N}$, there is a $h \in G$ such that $nh = g$. For example, $\mathbb{Q}$ and $\mathbb{Q}/\mathbb{Z}$ are divisible, and therefore injective.

\begin{proof}
We have to show that, if $G$ is a divisible Group, $\varphi: U \to G$ is any homomorphism, and $U$ is a subgroup of a Group $H$, there is a homomorphism $\psi: H \to G$ such that the restriction $\psi|_U = \varphi$. In other words, we want to extend $\varphi$ to a homomorphism $H \to G$.

Let $\mathcal{D}$ be the set of pairs $(K, \psi)$ such that $K$ is a subgroup of $G$ containing $U$ and $\psi: K \to G$ is a homomorphism with $\psi|_U = \varphi$. Then $\mathcal{D}$ ist non-empty since it contains $(U, \varphi)$, and it is partially ordered by
\[ (K, \psi) \leq (K', \psi') :\Longleftrightarrow K \subseteq K' \text{ and } \psi'|_K = \psi. \]

For any ascending chain
\[ (K_1, \psi_1) \leq (K_2, \psi_2) \leq \dots, \]
in $\mathcal{D}$, the pair $(\bigcup_{i \in \mathbb{N}} K_i, \bigcup_{i \in \mathbb{N}} \psi_i)$ is in $\mathcal{D}$, and it is an upper bound for this chain. Therefore, by Zorn's Lemma, $\mathcal{D}$ contains a maximal element $(M, \chi)$.

It remains to show that $M = H$. Suppose the opposite, and let $h \in H \setminus M$. Let $\langle h \rangle$ denote the subgroup of $H$ generated by $h$. If $\langle h \rangle \cap M = \{0\}$, the sum $M + \langle h \rangle$ is in fact a direct sum, and we can extend $\chi$ to $M + \langle h \rangle$ by choosing an arbitrary image of $h$ in $G$ and extending linearly. This contradicts the maximality of $(M, \chi)$.

Let us therefore suppose $\langle h \rangle \cap M$ contains an element $nh$, with $n \in \mathbb{N}$ minimal. Since $nh \in M$, and $\chi$ is defined on $M$, $\chi(nh)$ exists, and furthermore, since $G$ is divisible, there is a $g \in G$ such that $ng = \chi(nh)$. It is now easy to see that we can extend $\chi$ to $M + \langle h \rangle$ by defining $\chi(h) := g$, in contradiction to the maximality of $(M, \chi)$.

Therefore, $M = H$. This proves the statement.
\end{proof}
%%%%%
%%%%%
\end{document}

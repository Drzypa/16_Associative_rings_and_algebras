\documentclass[12pt]{article}
\usepackage{pmmeta}
\pmcanonicalname{ExampleOfARightNoetherianRingThatIsNotLeftNoetherian}
\pmcreated{2013-03-22 14:16:15}
\pmmodified{2013-03-22 14:16:15}
\pmowner{CWoo}{3771}
\pmmodifier{CWoo}{3771}
\pmtitle{example of a right noetherian ring that is not left noetherian}
\pmrecord{18}{35721}
\pmprivacy{1}
\pmauthor{CWoo}{3771}
\pmtype{Example}
\pmcomment{trigger rebuild}
\pmclassification{msc}{16P40}

% this is the default PlanetMath preamble.  as your knowledge
% of TeX increases, you will probably want to edit this, but
% it should be fine as is for beginners.

% almost certainly you want these
\usepackage{amssymb}
\usepackage{amsmath}
\usepackage{amsfonts}

% used for TeXing text within eps files
%\usepackage{psfrag}
% need this for including graphics (\includegraphics)
%\usepackage{graphicx}
% for neatly defining theorems and propositions
%\usepackage{amsthm}
% making logically defined graphics
%%%\usepackage{xypic}

% there are many more packages, add them here as you need them

% define commands here

\newtheorem{theorem}{Theorem}
\newtheorem{defn}{Definition}
\newtheorem{prop}{Proposition}
\newtheorem{lemma}{Lemma}
\newtheorem{cor}{Corollary}
\begin{document}
\PMlinkescapeword{right}

This example, due to Lance Small, is briefly described in \emph{Noncommutative Rings}, by I. N. Herstein, published by the Mathematical Association of America, 1968.

Let $R$ be the ring of all $2\times2$ matrices
\(
\begin{pmatrix}
a & b\\
0 & c
\end{pmatrix}
\)
such that $a$ is an integer and $b,c$ are rational. The claim is that $R$ is right noetherian but not left noetherian.


It is relatively straightforward to show that $R$ is not left noetherian. For each natural number $n$, let

\[
I_n = \{
\begin{pmatrix}
0 & \frac{m}{2^n}\\
0  & 0
\end{pmatrix} 
\mid m\in \mathbb{Z}\}.
\]


Verify that each $I_n$ is a left ideal in $R$ and that $I_0\subsetneq I_1\subsetneq I_2\subsetneq \cdots$.

It is a bit harder to show that $R$ is right noetherian. The approach given here uses 
the fact that a ring is right noetherian if all of its right ideals are finitely generated.

Let $I$ be a right ideal in $R$. We show that $I$ is finitely generated by 
checking all possible cases. In the first case, we assume that every matrix in $I$ has 
a zero in its upper left entry. In the second case, we assume that 
there is some matrix in $I$ that has a nonzero upper left entry. The second case splits
into two subcases: either every matrix in $I$ has a zero in its lower right entry or some matrix in $I$ has a nonzero lower right entry.

CASE 1: Suppose that for all matrices in $I$, the upper left entry is zero. Then every element of $I$ has the form
\[
\begin{pmatrix}
0 & y\\
0 & z
\end{pmatrix}
\text{ for some }y, z\in \mathbb{Q}.
\]

Note that for any $c\in\mathbb{Q}$ and any
\(
\begin{pmatrix}
0 & y\\
0 & z
\end{pmatrix}
\in I
\), we have
\(
\begin{pmatrix}
0 & cy\\
0 & cz
\end{pmatrix}
\in I
\) since
\[
\begin{pmatrix}
0 & y\\
0 & z
\end{pmatrix}
\begin{pmatrix}
0 & 0\\
0 & c
\end{pmatrix}
=
\begin{pmatrix}
0 & cy\\
0 & cz
\end{pmatrix}
\]
and $I$ is a right ideal in $R$. So $I$ looks like a rational vector space.

Indeed, note that
\(
V = \{(y,z)\in\mathbb{Q}^2\mid
\begin{pmatrix}
0 & y\\
0 & z
\end{pmatrix}\in I\}
\)
is a subspace of the two dimensional vector space $\mathbb{Q}^2$. So in $V$ there exist two 
(not necessarily linearly independent) vectors $(y_1,z_1)$ and $(y_2,z_2)$ 
which span $V$.

Now, an arbitrary element 
\(
\begin{pmatrix}
0 & y\\
0 & z
\end{pmatrix}
\)
in $I$ corresponds to the vector $(y,z)$ in $V$ and $
(y,z)=(c_1y_1+c_2y_2,c_1z_1+c_2z_2)$ for some $c_1, c_2\in\mathbb{Q}$. Thus
\[
\begin{pmatrix}
0 & y\\
0 & z
\end{pmatrix}
=
\begin{pmatrix}
0 & c_1y_1+c_2y_2\\
0 & c_1z_1+c_2z_2
\end{pmatrix}
=
\begin{pmatrix}
0 & y_1\\
0 & z_1
\end{pmatrix}
\begin{pmatrix}
0 & 0\\
0 & c_1
\end{pmatrix}
+
\begin{pmatrix}
0 & y_2\\
0 & z_2
\end{pmatrix}
\begin{pmatrix}
0 & 0\\
0 & c_2
\end{pmatrix}
\]
and it follows that $I$ is finitely generated 
by the set
\(\{
\begin{pmatrix}
0 & y_1\\
0 & z_1
\end{pmatrix},
\begin{pmatrix}
0 & y_2\\
0 & z_2
\end{pmatrix}
\}
\)
as a right ideal in $R$.

CASE 2: Suppose that some matrix in $I$ has a nonzero upper left entry. Then there is a least positive 
integer $n$ occurring as the upper left entry of a matrix in $I$. It follows that every element 
of $I$ can be put into the form
\[
\begin{pmatrix}
kn & y\\
0 & z
\end{pmatrix}
\text{ for some }k\in\mathbb{Z};\ y, z\in \mathbb{Q}.
\]

By definition of $n$, there is a matrix of the form
\(
\begin{pmatrix}
n & b\\
0 & c
\end{pmatrix}
\)
in $I$. Since $I$ is a right ideal in $R$ and since
\(
\begin{pmatrix}
n & b\\
0 & c
\end{pmatrix}
\begin{pmatrix}
1 & 0\\
0 & 0
\end{pmatrix}
=
\begin{pmatrix}
n & 0\\
0 & 0
\end{pmatrix},
\)
it follows that 
\(
\begin{pmatrix}
n & 0\\
0 & 0
\end{pmatrix}
\)
is in $I$. Now break off into two subcases.

\emph{case} 2.1: Suppose that every matrix in $I$ has a zero in its lower right entry. Then 
an arbitrary element of $I$ has the form
\[
\begin{pmatrix}
kn & y\\
0 & 0
\end{pmatrix}
\text{ for some }k\in\mathbb{Z}, y\in\mathbb{Q}.
\]
Note that 
\(
\begin{pmatrix}
kn & y\\
0 & 0
\end{pmatrix}
=
\begin{pmatrix}
n & 0\\
0 & 0
\end{pmatrix}
\begin{pmatrix}
k & \frac{y}{n}\\
0 & 0
\end{pmatrix}
\). Hence, 
\(
\begin{pmatrix}
n & 0\\
0 & 0
\end{pmatrix}
\) generates $I$ as a right ideal in $R$.

\emph{case} 2.2: Suppose that some matrix in $I$ has a nonzero lower right entry. That is, in $I$ 
we have a matrix
\[
\begin{pmatrix}
mn & y_1\\
0 & z_1
\end{pmatrix}
\text{ for some }m\in\mathbb{Z};\ y_1, z_1\in \mathbb{Q};\ z_1\neq 0.
\]
Since
\(
\begin{pmatrix}
n & 0\\
0 & 0
\end{pmatrix}
\in I,
\) it follows that
\(
\begin{pmatrix}
n & y_1\\
0 & z_1
\end{pmatrix}
\in I.
\)
Let 
\(
\begin{pmatrix}
kn & y\\
0 & z
\end{pmatrix}
\) 
be an arbitrary element of $I$. Since
\(
\begin{pmatrix}
kn & y\\
0 & z
\end{pmatrix}
=
\begin{pmatrix}
n & y_1\\
0 & z_1
\end{pmatrix}
\begin{pmatrix}
k & \frac{1}{n}(y-\frac{y_1z}{z_1})\\
0 & \frac{z}{z_1}
\end{pmatrix},
\) it follows that
\(
\begin{pmatrix}
n & y_1\\
0 & z_1
\end{pmatrix}
\)
generates $I$ as a right ideal in $R$.

In all cases, $I$ is a finitely generated.
%%%%%
%%%%%
\end{document}

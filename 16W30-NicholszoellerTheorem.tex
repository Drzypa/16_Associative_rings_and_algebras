\documentclass[12pt]{article}
\usepackage{pmmeta}
\pmcanonicalname{NicholszoellerTheorem}
\pmcreated{2013-03-22 18:58:34}
\pmmodified{2013-03-22 18:58:34}
\pmowner{joking}{16130}
\pmmodifier{joking}{16130}
\pmtitle{nichols-zoeller theorem}
\pmrecord{6}{41839}
\pmprivacy{1}
\pmauthor{joking}{16130}
\pmtype{Theorem}
\pmcomment{trigger rebuild}
\pmclassification{msc}{16W30}

\endmetadata

% this is the default PlanetMath preamble.  as your knowledge
% of TeX increases, you will probably want to edit this, but
% it should be fine as is for beginners.

% almost certainly you want these
\usepackage{amssymb}
\usepackage{amsmath}
\usepackage{amsfonts}

% used for TeXing text within eps files
%\usepackage{psfrag}
% need this for including graphics (\includegraphics)
%\usepackage{graphicx}
% for neatly defining theorems and propositions
%\usepackage{amsthm}
% making logically defined graphics
%%%\usepackage{xypic}

% there are many more packages, add them here as you need them

% define commands here

\begin{document}
Let $H$ be a Hopf algebra over a field $k$ with an antipode $S$. We will say that $K\subseteq H$ is a \textit{Hopf subalgebra} if $K$ is both subalgebra and subcoalgebra of underlaying algebra and coalgebra structures of $H$, and additionaly $S(K)\subseteq K$. In particular a Hopf subalgebra $K\subseteq H$ is an algebra over $k$, so $H$ may be regarded as a $K$-module.

\textbf{The Nichols-Zoeller Theorem.} If $K\subseteq H$ is a Hopf subalgebra of a Hopf algebra $H$, then $H$ is free as a $K$-module. In particular, if $H$ is finite dimensional, then $\mathrm{dim}_{k}K$ divides $\mathrm{dim}_{k}H$.

\textbf{Remark 1.} This theorem shows that Hopf algebras are very similar to groups, because this is a Hopf analogue of the Lagrange Theorem.

\textbf{Remark 2.} Generally this theorem does not need to hold if $H$ is only an algebra. For example, consider $H=\mathbb{M}_{n}(k)$ the matrix algebra, where $n\geq 2$ and let $T\subseteq H$ be the upper triangular matrix subalgebra. It is well known that $\mathrm{dim}_{k}H=n^2$ and $\mathrm{dim}_{k}T=\frac{n(n+1)}{2}$. Of course $\frac{n(n+1)}{2}$ does not divide $n^2$ for $n\geq 2$. Thus the Nichols-Zoeller Theorem does not hold for algebras.
%%%%%
%%%%%
\end{document}

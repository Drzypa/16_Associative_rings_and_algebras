\documentclass[12pt]{article}
\usepackage{pmmeta}
\pmcanonicalname{NilpotentIdeal}
\pmcreated{2013-03-22 12:01:27}
\pmmodified{2013-03-22 12:01:27}
\pmowner{antizeus}{11}
\pmmodifier{antizeus}{11}
\pmtitle{nilpotent ideal}
\pmrecord{6}{30994}
\pmprivacy{1}
\pmauthor{antizeus}{11}
\pmtype{Definition}
\pmcomment{trigger rebuild}
\pmclassification{msc}{16D25}

\usepackage{amssymb}
\usepackage{amsmath}
\usepackage{amsfonts}
\usepackage{graphicx}
%%%\usepackage{xypic}
\begin{document}
A left (right) ideal $I$ of a ring $R$ is a {\it nilpotent ideal} if $I^n = 0$ for some positive integer $n$.  Here $I^n$ denotes a product of ideals -- $I \cdot I \cdots I$.
%%%%%
%%%%%
%%%%%
\end{document}

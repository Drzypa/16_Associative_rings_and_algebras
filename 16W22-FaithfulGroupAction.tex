\documentclass[12pt]{article}
\usepackage{pmmeta}
\pmcanonicalname{FaithfulGroupAction}
\pmcreated{2013-03-22 14:02:23}
\pmmodified{2013-03-22 14:02:23}
\pmowner{rspuzio}{6075}
\pmmodifier{rspuzio}{6075}
\pmtitle{faithful group action}
\pmrecord{8}{35259}
\pmprivacy{1}
\pmauthor{rspuzio}{6075}
\pmtype{Definition}
\pmcomment{trigger rebuild}
\pmclassification{msc}{16W22}
\pmclassification{msc}{20M30}

\endmetadata

% this is the default PlanetMath preamble.  as your knowledge
% of TeX increases, you will probably want to edit this, but
% it should be fine as is for beginners.

% almost certainly you want these
\usepackage{amssymb}
\usepackage{amsmath}
\usepackage{amsfonts}

% used for TeXing text within eps files
%\usepackage{psfrag}
% need this for including graphics (\includegraphics)
%\usepackage{graphicx}
% for neatly defining theorems and propositions
%\usepackage{amsthm}
% making logically defined graphics
%%%\usepackage{xypic}

% there are many more packages, add them here as you need them

% define commands here
\begin{document}
Let $A$ be a $G$-set, that is, a set acted upon by a group $G$ with action
$\psi:G\times A\to A$.  Then for any $g\in G$, the map $m_g\colon A\to A$ defined by
\[m_g(x)= \psi(g,x)\]
is a permutation of $A$ (in other words, a bijective function from $A$ to itself) and so an element of $S_A$.
We can even get an homomorphism from $G$ to $S_A$ by the rule $g\mapsto m_g$.

If for any pair $g,h\in G$  $g\neq h$ we have
$m_g\neq m_h$, in other words, the homomorphism $g\to m_g$ being injective, we say that the action is faithful.
%%%%%
%%%%%
\end{document}

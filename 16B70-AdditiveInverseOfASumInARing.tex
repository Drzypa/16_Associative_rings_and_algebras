\documentclass[12pt]{article}
\usepackage{pmmeta}
\pmcanonicalname{AdditiveInverseOfASumInARing}
\pmcreated{2013-03-22 15:45:02}
\pmmodified{2013-03-22 15:45:02}
\pmowner{rspuzio}{6075}
\pmmodifier{rspuzio}{6075}
\pmtitle{additive inverse of a sum in a ring}
\pmrecord{11}{37703}
\pmprivacy{1}
\pmauthor{rspuzio}{6075}
\pmtype{Theorem}
\pmcomment{trigger rebuild}
\pmclassification{msc}{16B70}

\endmetadata

% this is the default PlanetMath preamble.  as your knowledge
% of TeX increases, you will probably want to edit this, but
% it should be fine as is for beginners.

% almost certainly you want these
\usepackage{amssymb}
\usepackage{amsmath}
\usepackage{amsfonts}

% used for TeXing text within eps files
%\usepackage{psfrag}
% need this for including graphics (\includegraphics)
%\usepackage{graphicx}
% for neatly defining theorems and propositions
%\usepackage{amsthm}
% making logically defined graphics
%%%\usepackage{xypic}

% there are many more packages, add them here as you need them

% define commands here
\begin{document}
Let $R$ be a ring with elements $a, b \in R$.  
Suppose we want to find the inverse of the element $(a + b) \in R$.  
(Note that we call the element $(a+b)$ the sum of $a$ and $b$.)  
So we want the unique element $c \in R$ so that $(a + b) + c = 0$.  
Actually, let's put $c = (-a) + (-b)$ where $(-a) \in R$ is the additive inverse of $a$ and $(-b) \in R$ is the additive inverse of $b$.  
Because addition in the ring is both associative and commutative we see that
\begin{eqnarray*}
(a + b) + ((-a) + (-b)) & = & (a + (-a)) +(b + (-b))\\
 & = & 0 + 0 = 0
\end{eqnarray*}
since $(-a) \in R$ is the additive inverse of $a$ and $(-b) \in R$ is the additive inverse of $b$.  
Since additive inverses are unique this means that the additive inverse of $(a + b)$ must be $(-a) + (-b)$.  We write this as
\[ -(a + b) = (-a) + (-b). \]

It is important to note that we {\em cannot} just distribute the minus sign across the sum because this would imply that $-1 \in R$ which is not the case if our ring is not with unity.
%%%%%
%%%%%
\end{document}
